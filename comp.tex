\documentclass[11pt]{article}
\usepackage{william}
\usepackage{comment}
\title{Collected Problems: Computational}
\author{William Dai}
\date{Started June 30, 2020}

\excludecomment{sol}
\begin{document}
\maketitle

\section{Introduction}
I use the following scheme: 1 point is roughly AMC 10 8-14 level. 2 points is roughly AMC 10 \# 15-17 level, 3 points are AMC 10 \# 18-21 level, 4 points are AMC 10 \# 22-23 level and 5 points are \# 24-25 level. Furthermore, 6 points are \#6-8 AIME level, 7 points are \# 9-11 AIME level,   8 points are \#12-13 AIME level, 9 points are \#14-15 AIME level, and 10 points are a hypothetical \#16-18 AIME level. These are usually olympiad-style problems that require several major lemmas and have very long solutions.

Most of these problems are from more obscure contests that will serve as good AIME and AMC practice. 
\section{Combinatorics}
\subsection{Casework}
\prob{2}{BmMT 2014}{Call a positive integer top-heavy if at least half of its digits are in the set \{7, 8, 9\}. How many
three digit top-heavy numbers exist? (No number can have a leading zero.)}

\begin{sol}
Consider $7,8,9$ to be the same and assign actual values at end, do casework on number of $7,8,9$.
\end{sol}

\prob{3}{PuMAC 2019}{Suppose Alan, Michael, Kevin, Igor, and Big Rahul are in a running race. It is given that
exactly one pair of people tie (for example, two people both get second place), so that no other
pair of people end in the same position. Each competitor has equal skill; this means that each
outcome of the race, given that exactly two people tie, is equally likely. The probability that
Big Rahul gets first place (either by himself or he ties for first) can be expressed in the form
$m/n$, where m, n are relatively prime, positive integers. Compute $m + n$.}

\begin{sol} 
First, find total number of outcomes. Then, casework on if Big Rahul wins by himself or if he ties for first
\end{sol}

\prob{3}{PuMAC 2019}{Prinstan Trollner and Dukejukem are competing at the game show WASS. Both players spin
a wheel which chooses an integer from 1 to 50 uniformly at random, and this number becomes
their score. Dukejukem then flips a weighted coin that lands heads with probability 3/5. If
he flips heads, he adds 1 to his score. A player wins the game if their score is higher than
the other player’s score. The probability Dukejukem defeats the Trollner to win WASS equals
$m/n$ where $m, n$ are co-prime positive integers. Compute $m + n$.}

\begin{sol} 
Casework on if Dukejukem flips heads or tails on coin.
\end{sol}

\prob{4}{Purple Comet 2015 HS}{
Seven people of seven different ages are attending a meeting. The seven people leave the meeting one at a
time in random order. Given that the youngest person leaves the meeting sometime before the oldest
person leaves the meeting, the probability that the third, fourth, and fifth people to leave the meeting do so in order of their ages (youngest to oldest) is $\frac{m}{n}$ , where m and n are relatively prime positive integers. Find $m + n$.}

\begin{sol}
 Casework on if youngest or oldest in third, fourth, fifth block.
\end{sol}

\prob{5}{HMMT November 2014}{Consider the set of 5-tuples of positive integers at most $5$. We say the tuple $(a_1, a_2, a_3, a_4, a_5)$ is perfect if for any distinct indices $i, j, k$, the three numbers $a_i
, a_j , a_k$ do not form an arithmetic progression (in any order). Find the number of perfect 5-tuples.}

\prob{5}{HMMT November 2013}{Find the number of positive integer divisors of $12!$ that leave a remainder of $1$ when divided by $3$.}

\begin{sol} 
Casework on parity of exponent of $2,5,11$. Note that there can't be any $3$ and $7$ can be included or excluded without consequence.
\end{sol}

\subsubsection{PIE}
\prob{1}{}{How many integers from $1$ to $100$ (inclusive) are multiples of $2$ or $3?$}

\begin{sol}
There are $50$ multiples of $2$ and $33$ multiples of $3$. If and only if a number is both a multiple of $2$ and $3$, then it is a multiple of $6$. There are $16$ multiples of $6$. Our answer is $50+33-16=\boxed{67}$.
\end{sol}

\prob{1}{AMC 10B 2017/13}{There are $20$ students participating in an after-school program offering classes in yoga, bridge, and painting. Each student must take at least one of these three classes, but may take two or all three. There are $10$ students taking yoga, $13$ taking bridge, and $9$ taking painting. There are $9$ students taking at least two classes. How many students are taking all three classes?}

\prob{8}{SLKK AIME 2020}{
Andy the Banana Thief is trying to hide from Sheriff Buffkin in a row of
6 distinct houses labeled 1 through 6. Andy and Sheriff Buffkin each pick a permutation
of the 6 houses, chosen uniformly at random. On the $n^{\text{th}}$ day, with $1 \leq n \leq 6$, Andy
and Sheriff Buffkin visit the $n^{\text{th}}$ house in their respective permutations, and Andy is
caught by the Sheriff on the first day they visit the same house. For example, if Andy’s
permutation is $1, 3, 4, 5, 6, 2$ and Sheriff Buffkin’s permutation is $3, 4, 1, 5, 6, 2$, Andy is
caught on day $4$. Given that Sheriff Buffkin catches Andy within 6 days and the expected
number of days it takes to catch Andy can be expressed as $\frac{a}{b}$
for relatively prime positive
integers $a$ and $b$, find the remainder when $a + b$ is divided by $1000$.}

\begin{sol} 
Notice that we can fix Sheriff Buffkin's permutation to just be $1,2,3,4,5,6$ because for each of Buffkin's permutation, the probability of being caught on the $n$th day clearly doesn't change. The number of Andy's permutations such that he does get caught is $6!-D_{6}=455$. Then, we do casework on day caught and then use complementary PIE. If he is caught on the $n$th day, the $n$th number inf Andy's permutation is clearly $n$. Then, we do PIE on the number of permutations  such that at least one of the previous numbers are the same. This turns out to be $\binom{n-1}{1}\cdot 4! - \binom{n-1}{2}\cdot 3! + \binom{n-1}{3}\cdot 2! \cdots$. Then, the probability that he is caught on the $n$th day is $\frac{5!-\binom{n-1}{1}\cdot 4! + \binom{n-1}{2}\cdot 3! - \binom{n-1}{3}\cdot 2! \cdots}{6!}$.


Doing the computation for all days, we get $\frac{1331}{455}\implies \boxed{786}$.
\end{sol}

\subsection{Perspectives}
\prob{1}{AIME I 2002/1}{Many states use a sequence of three letters followed by a sequence of three digits as their standard license-plate pattern. Given that each three-letter three-digit arrangement is equally likely, the probability that such a license plate will contain at least one palindrome (a three-letter arrangement or a three-digit arrangement that reads the same left-to-right as it does right-to-left) is $\dfrac{m}{n}$, where $m$ and $n$ are relatively prime positive integers. Find $m+n.$}

\begin{sol}
Calculate the complementary probability. The probability that the three numbers aren't a palindrome is $1\cdot 1\cdot \frac{9}{10}=\frac{9}{10}$. The probability the three lettters aren't a palindrome is $1\cdot 1\cdot \frac{25}{26}=\frac{25}{26}$. Multiplying together, the probability both aren't a palindrome is $\frac{9\cdot 25}{10\cdot 26}=\frac{9\cdot 5}{2\cdot 26}=\frac{45}{52}$. So, our wanted probability is $1-\frac{45}{52}=\frac{7}{52}\implies \boxed{59}$.
\end{sol}

\prob{1}{PuMAC 2019}{How many ways can you arrange 3 Alice’s, 1 Bob, 3 Chad’s, and 1 David in a line if the Alice’s
are all indistinguishable, the Chad’s are all indistinguishable, and Bob and David want to be
adjacent to each other? (In other words, how many ways can you arrange 3 A’s, 1 B, 3 C’s,
and 1 D in a row where the B and D are adjacent?)}

\begin{sol} 
Consider BD as one block of E and then multiply by $2!=2$ at the end to permute the B and D's. We want to arrange 3 A's, 1 E, and 3 C's. There are $\frac{7!}{3!3!}=\frac{5040}{6\cdot 6}=140$ such permutations. So, the total number of ways is $\boxed{280}$
\end{sol}

\prob{2}{MA$\theta$ 2016}{The product of any two of the elements of the set $\{30, 54, N\}$ is divisible by the third. Find
the number of possible values of N.}

\begin{sol}
 Consider the primes $2,3,5$ separately and get independent inequalities.
\end{sol}

\prob{2}{2017 AMC 10B/17}{Call a positive integer $monotonous$ if it is a one-digit number or its digits, when read from left to right, form either a strictly increasing or a strictly decreasing sequence. For example, $3$, $23578$, and $987620$ are monotonous, but $88$, $7434$, and $23557$ are not. How many monotonous positive integers are there?}


\prob{2}{AIME II 2002/9}{Let $\mathcal{S}$ be the set $\lbrace1,2,3,\ldots,10\rbrace$ Let $n$ be the number of sets of two non-empty disjoint subsets of $\mathcal{S}$. (Disjoint sets are defined as sets that have no common elements.) Find the remainder obtained when $n$ is divided by $1000$.}

\begin{sol} 
We count the number of ordered pairs for now and divide by two at the end. For each element, we have $3$ choices: in either of the sets or in neither. But, we also need to subtract off when at least one of the subsets is empty. If at least one is empty, there's $2$ choices for each element. If both are empty, there is $1$ choice for each element. We also have to divide by two in the end to make it unordered (and note that it's impossible for the two sets to be the same). 

$$\frac{3^{10}-2\cdot 2^{10}+1}{2} = \frac{59049-2048+1}{2}=\frac{57002}{2}=28\boxed{501}$$
\end{sol} 

\prob{2}{AMC 10B 2018/22}{Real numbers $x$ and $y$ are chosen independently and uniformly at random from the interval $[0,1]$. What is the probability that $x,y,$ and $1$ are the side lengths of an obtuse triangle?}

\begin{sol} 
Note that $1>x,y$ so $1$ is opposite the obtuse angle. This gives the inequalty $1>x^2+y^2$. Also, by the Triangle Inequalty, $1 < x+y$. Graphing this, we want the area under the quarter unit circle centered at $(0,0)$ but above the line from $(0,1)$ to $(1,0)$. This has area $\boxed{\frac{\pi-2}{4}}$
\end{sol}

\prob{2}{AMC 10B 2020/23}{Square $ABCD$ in the coordinate plane has vertices at the points $A(1,1), B(-1,1), C(-1,-1),$ and $D(1,-1).$ Consider the following four transformations:
    
\begin{itemize}
    \Item $L,$ a rotation of $90^{\circ}$ counterclockwise around the origin;

    \Item $R,$ a rotation of $90^{\circ}$ clockwise around the origin;
    
    \Item $H,$ a reflection across the $x$-axis; and

    \Item $V,$ a reflection across the $y$-axis.
\end{itemize}

Each of these transformations maps the squares onto itself, but the positions of the labeled vertices will change. For example, applying $R$ and then $V$ would send the vertex $A$ at $(1,1)$ to $(-1,-1)$ and would send the vertex $B$ at $(-1,1)$ to itself. How many sequences of $20$ transformations chosen from $\{L, R, H, V\}$ will send all of the labeled vertices back to their original positions? (For example, $R, R, V, H$ is one sequence of $4$ transformations that will send the vertices back to their original positions.)}

\begin{sol} 
Notice that after any sequence of $19$ moves, we have a unique move that takes the square back to its original square. So, the total number of sequences is $4^{19}=\boxed{2^{38}}$.
\end{sol}

\prob{2}{CNCM Online Round 1}{Pooki Sooki has 8 hoodies, and he may wear any of them throughout a 7 day week. He changes hishoodie exactly $2$ times during the week, and will only do so at one of the 6 midnights. Once he changes out of a
hoodie, he never wears it for the rest of the week. The number of ways he can wear his hoodies throughout the week
can be expressed as $\frac{8!}{2^k}$ . Find $k$.}

\begin{sol} 
There is $8\cdot 7\cdot 6$ ways to choose the hoodies Pooki wears and there are $\binom{6}{2}=15=3\cdot 5$ ways to choose the midnights he changes. The number of ways is then $8\cdot 7\cdot 6\cdot 5\cdot 3 = \frac{8!}{4\cdot 2}=\frac{8!}{2^{3}}$. Our answer is $\boxed{3}$.
\end{sol}

\prob{3}{BmMT 2014}{If you roll three regular six-sided dice, what is the probability that the three numbers showing
will form an arithmetic sequence? (The order of the dice does matter, but we count both (1,
3, 2) and (1, 2, 3) as arithmetic sequences.)}

\begin{sol} 
We count the number of ordered triplets $(a,b,c)$ such that two of the numbers average to the third number. WLOG $a\leq b\leq c$. Note that if $a,b,c$ are distinct, then this corresponds to $6$ ordered triplets. Otherwise, $a=b=c$ and this clearly corresponds to only $1$ ordered triplet.

For the first case, notice that this is identical to choosing two numbers $a,c$ (unordered) and seeing if their average is an integer. This happens if $a,c$ are the same parity. Then, this gives $\binom{3}{2} + \binom{3}{2}=6$ $(a,b,c)$ such that $a\leq b\leq c$. We multiply by $6$ to get the full number of ordered triplets which is $36$.

For the second case, this clearly has $6$ ordered $(a,b,c)$.

Then, our probability is $\frac{42}{216}=\boxed{\frac{7}{36}}$.
\end{sol}

\prob{4}{PuMAC 2019}{Keith has 10 coins labeled 1 through 10, where the ith coin has weight $2^i$. The coins are all
fair, so the probability of flipping heads on any of the coins is $\frac{1}{2}$. After flipping all of the
coins, Keith takes all of the coins which land heads and measures their total weight, $W$. If the
probability that $137 \leq W \leq 1061$ is $m/n$ for coprime positive integers $m, n$, determine $m + n$}

\begin{sol} 
This bijects to even binary numbers between $0$ and $2046$. There $\frac{1060-138}{2}+1=462$ even numbers in the range $[137,1061]$. Then, $\frac{462}{1024}=\frac{231}{512}\implies \boxed{743}$.
\end{sol}

\prob{4}{PHS HMMT TST 2016}{Compute the number of ordered triples of sets $(A_{1},A_{2},A_{3})$ that satisfy the following:
\begin{enumerate}
\item $A_{1}\cup A_{2}\cup A_{3} = \{1,2,3,4,5,6\}$
\item $A_{1}\cap A_{2}\cap A_{3} = \emptyset$
\end{enumerate}}

\begin{sol}
Consider each element separately and see where it can go in the Venn Diagram. It can go in $8-2=6$ sections as it can't be in all three or not in any. So, $\boxed{6^6}$.
\end{sol}

\prob{4}{Magic Math AMC 10 2020}{The county of Tropolis has five towns, with no roads built between any two of them. How
many ways are there for the mayor of Tropolis to build five roads between five different pairs
of towns such that it is possible to get from any town to any other town using the roads?}

\prob{5} {HMMT Feburary 2014}{We have a calculator with two buttons that displays an integer x. Pressing the first button replaces
$x$ by $\lfloor \frac{x}{2} \rfloor$ , and pressing the second button replaces $x$ by $4x + 1$. Initially, the calculator displays 0.
How many integers less than or equal to 2014 can be achieved through a sequence of arbitrary button
presses? (It is permitted for the number displayed to exceed 2014 during the sequence. Here, byc
denotes the greatest integer less than or equal to the real number y.)}

\begin{sol}
Any number with $1$ digits separated by one or more $0$'s  is valid. Notice that for $2015$ through $2047$, the first two digits are $11$ so they are not valid. Casework on number of digits now.
\end{sol}

\subsubsection{Stars and Bars}

\prob{1}{AMC 8 2019/25}{Alice has $24$ apples. In how many ways can she share them with Becky and Chris so that each person (include Alice) has at least $2$ apples?}

\begin{sol}
We have $a+b+c=24$ and $a\ge 2$ while $b,c\ge 0$. Let $a'=a-2$. Then, $a'\ge 0$. We have $a'+b+c=22$. By Stars and Bars, there are $\binom{24}{2}=23\cdot 12=\boxed{276}$ ways to distribute.
\end{sol}

\prob{1}{AMC 10A 2003/21}{Pat is to select six cookies from a tray containing only chocolate chip, oatmeal, and peanut butter cookies. There are at least six of each of these three kinds of cookies on the tray. How many different assortments of six cookies can be selected?}

\begin{sol} 
Let $a,b,c$ be the number of choclate chip, oatmetal, and peanut butter cookies repsecitvely. Then, $a+b+c=6$ has $\binom{8}{2}=\boxed{28}$ distributions.
\end{sol}

\prob{1}{AMC 10A 2018/11}{When 7 fair standard 6-sided dice are thrown, the probability that the sum of the numbers on the top faces is 10 can be written as \[\frac{n}{6^7},\]where $n$ is a positive integer. What is $n$?}

\begin{sol} 
Let the $i$th dice roll $x_{i}$. Notice that $x_{i}\ge 1$. Let $x_{i}'=x_{1}-1$. Then, $x_{1}+x_{2}\cdots + x_{7}=10\implies x_{1}+x_{2}\cdots + x_{7}=3$ has $\binom{9}{6}=\boxed{84}$ distributions. Note that it is impossible for any of the $x_{i}$'s to exceed $6$.
\end{sol}

\prob{4}{AMC 12A 2006/25}{How many non- empty subsets $S$ of $\{1,2,3,\ldots ,15\}$ have the following two properties?
\begin{enumerate}
    \item No two consecutive integers belong to $S$.
    
    \item If $S$ contains $k$ elements, then $S$ contains no number less than $k$.
\end{enumerate}}

\begin{sol} 
We can do casework on the number of elements $k$. Then, we can only have numbers in the range $\{k,k+1\cdots 15\}$. Now, let the elements in the subset be $a_{1},a_{2}\cdots a_{k}$. Let $d_{1}=a_{1}-k$, $d_{2}=a_{2}-a_{1}$, $d_{3}=a_{3}-a_{2}$ and so on until $d_{k}=a_{k}-a_{k-1}$ and $d_{k+1}=15-a_{k}$. We have that $d_{1}+d_{2}\cdots + d_{k}+d_{k+1}=15-k$. 

Notice that $d_{2},d_{3}\cdots d_{k}\ge 2$ to satisfy that no two consecutive integers are in $S$. So, let $d_{k}'=d_{k}-2$ for $k=2,3\cdots k$. So, $d_{1}+d_{2}'+d_{3}'\cdots +d_{k}'+d_{k+1}=15-k-2(k-1)=17-3k$. By Stars and Bars, there are $\binom{17-2k}{k}$ ordered $(k+1)$-tuplets $(d_{1},d_{2}\cdots d_{k+1})$. Notice that $(d_{1},d_{2}\cdots d_{k+1})$ determines $a_{1},a_{2}\cdots a_{k}$.
\end{sol}

\subsubsection{Expected Value}
\prob{3}{Math Prizes For Girls 2018}{Maryam has a fair tetrahedral die, with the four faces of the die labeled
1 through 4. At each step, she rolls the die and records which number
is on the bottom face. She stops when the current number is greater
than or equal to the previous number. (In particular, she takes at least
two steps.) What is the expected number (average number) of steps
that she takes? Express your answer as a fraction in simplest form.}

\prob{5}{SLKK AIME 2020}{
Woulard forms a 8 letter word by picking each letter from the set $\{w, o, u\}$
with equal probability. The score of a word is the nonnegative difference between the
number of distinct occurrences of the three-letter word “uwu” and the number of distinct
occurrences of the three-letter word “owo”. For example, the string “owowouwu” has a
score of 2 - 1 = 1. If the expected score of Woulard’s string can be expressed as $\frac{a}{b}$ for
relatively prime positive integers a and b, find the remainder when $a+b$ is divided by $1000$.}

\begin{sol} 
Let the $value$ of the string be the total number of uwu and owo's. Note that the value only differs from the score when there are both uwu and owo's. We can easily compute the value using Linearity of Expectation and do casework on when it differs.
\end{sol}

\prob{6}{PuMAC 2019}{. Marko lives on the origin of the Cartesian plane. Every second, Marko moves 1 unit up with
probability 2/9, 1 unit right with probability 2/9, 1 unit up and 1 unit right with probability
4/9, and he doesn’t move with probability 1/9. After 2019 seconds, Marko ends up on the
point (A, B). What is the expected value of A · B?}

\prob{6}{PuMAC 2019}{Kelvin and Quinn are collecting trading cards; there are 6 distinct cards that could appear
in a pack. Each pack contains exactly one card, and each card is equally likely. Kelvin buys
packs until he has at least one copy of every card, then he stops buying packs. If Quinn is
missing exactly one card, the probability that Kelvin has at least two copies of the card Quinn
is missing is expressible as $m/n$ for coprime positive integers $m, n$. Determine $m + n$.}

\begin{sol} 
Complementary counting, find probability that Kelvin has exactly one copy of the card Quinn is missing. 
\end{sol}

\subsubsection{Recursion}
\prob{4}{Math Prizes For Girls 2019}{A 1×5 rectangle is split into five unit squares (cells) numbered
1 through 5 from left to right. A frog starts at cell 1. Every second it jumps
from its current cell to one of the adjacent cells. The frog makes exactly 14
jumps. How many paths can the frog take to finish at cell 5?}

\prob{7}{CNCM Online Round 2}{On a chessboard with 6 rows and 9 columns, the Slow Rook is placed in the bottom-left
corner and the Blind King is placed on the top-left corner. Then, 8 Sleeping Pawns are placed such that
no two Sleeping Pawns are in the same column, no Sleeping Pawn shares a row with the Slow Rook or
the Blind King, and no Sleeping Pawn is in the rightmost column. The Slow Rook can move vertically
or horizontally 1 tile at a time, the Slow Rook cannot move into any tile containing a Sleeping Pawn,
and the Slow Rook takes the shortest path to reach the Blind King. How many ways are there to place
the Sleeping Pawns such that the Slow Rook moves exactly 15 tiles to get to the space containing the
Blind King?}

\subsection{Miscellaneous}

\prob{1}{Mandelbrot Nationals Sample Test}{Michael Jordan's probability of hitting any basketball shot is three times greater than mine, which never exceeds a third. To beat him in a game, I need to hit a shot myself and have Jordan miss the same shot. If I pick my shot optimally, what is the maximum probability of winning which I can attain?}

\begin{sol} 
What's the max of $p(1-3p)$? 
\end{sol}

\prob{2}{CNCM Online Round 2}{Adi the Baller is shooting hoops, and makes a shot with probability $p$. He keeps shooting
hoops until he misses. The value of $p$ that maximizes the chance that he makes between $35$ and $69$
(inclusive) buckets can be expressed as $\frac{1}{\sqrt[a]{b}}$
for a prime a and positive integer $b$. Find $a + b$.}

\prob{2}{PHS ARML TST 2017}{Consider a group of eleven high school students. To create a middle school math contest, they must
pick a four-person committee to write problems and a four-person committee to proofread. Every
student can be on neither committee, one committee, or both committees, except for one student who
does not want to be on both. How many combinations of committees are possible?}

\begin{sol} 
Complementary counting, find how many committees have that student on both and how many committees without that restriction
\end{sol}

\prob{2}{Mandelbrot Regionals 2009}{Mr. Strump has formed three person groups in his math class for working on projects. Every student is in exactly two groups, and any two groups have at most one person in common. In fact, if two groups are chosen at random then the probability that they have exactly one person in common is one-third. How many students are there in Mr. Strump’s class?}

\prob{6}{CRMT Team 2019}{A deck of the first 100 positive integers is randomly shuffled. Find the expected number of
draws it takes to get a prime number if there is no replacement.}

\prob{6}{CNCM PoTD}{Find the remainder when $\sum_{n=0}^{333}\sum_{k=3n}^{999} \binom{k}{3n}$ is divided by $70$.}

\setcounter{problem}{0}
\section{Number Theory}
\subsection{Divisors}
\prob{1}{MA$\theta$ 2018}{How many distinct prime numbers are in the first 50 rows of Pascal’s Triangle?}

\begin{sol} 
If $k\neq 1,n-1$ for $\binom{n}{k}$, then $\binom{n}{k}$ is composite by its explicit formula. So, $\binom{n}{1}=n$, how many of those are prime for $1\leq n\leq 50$? 
\end{sol}

\prob{2}{AHSME 1984}{ How many triples $(a,b,c)$ of positive integers satisfy the simultaneous equations:
$$ab+bc=44$$
$$ac+bc=23$$}

\prob{2}{PHS ARML TST 2017}{Compute the greatest prime factor of 
$$3^8+2\cdot 3^4\cdot 4^4+2^{16}$$}

\begin{sol} 
Let $3^4=x$ and $4^4=y$. Then, this is just $x^2+2xy+y^2=(x+y)^2$. So, it is $(81+256)^2=(337)^2$. Note that $337$ is prime as it is not divisible by $2,3,5,7,11,13,17$. So, the greatest prime factor is $\boxed{337}$.
\end{sol}

\prob{2}{HMMT November 2014}{Compute the greatest common divisor of $4^8 - 1$ and $8^{12} - 1$}

\prob{2}{BMT 2018}{. Suppose for some positive integers, that $\frac{p+\frac{1}{q}}{q+\frac{1}{p}}=17$. What is the greatest integer $n$ such that $\frac{p+q}{n}$
is always an integer?}

\begin{sol}
$\frac{p+\frac{1}{q}}{q+\frac{1}{p}}=\frac{p}{q}\implies p=17q$. Then $\frac{p+q}{n}=\frac{18p}{n}$. So, $n=\boxed{18}$.
\end{sol}

\prob{2}{BMT 2018}{How many multiples of 20 are also divisors of 17!?}

\prob{3}{HMMT Feburary 2018}{Distinct prime numbers $p,q,r$ satisfy the equation $$2pqr + 50pq = 7pqr + 55pr = 8pqr + 12qr  =A$$ for some positive integer $A$. What is $A$?}

\prob{3}{Math Prizes For Girls 2019}{How many positive integers less than 4000 are not divisible by 2, not
divisible by 3, not divisible by 5, and not divisible by 7?}

\prob{3}{MA$\theta$ 2018}{The number $1 \cdot 1! + 2 \cdot 2! + 3 \cdot 3! +\cdots + 100 \cdot 100!$ ends with a string of 9s. How many
consecutive 9s are at the end of the number?}

\begin{sol}
$n\cdot n! = (n+1)!-n!$, then telescope to get $101!-1!$. Then, we are finding how many $0$'s does $101!$ have at the end which is equivalent to the number of factors of $10$. The number of factors of $10$ is equivalent to the number of factors of $5$. By Legendre's, there are $20+4=\boxed{24}$ 5's.
\end{sol}

\prob{3}{BMT 2015}{There exists a unique pair of positive integers k, n such that k is divisible by 6, and $\sum_{i=1}^{k} i^2 = n^2$.
Find $(k, n)$.}

\begin{sol}
You get $k\cdot (6k+1)\cdot (12k+1)=n^2$. All of these are pairwise coprime so each are squares. Trying out the first couple of squares for $k$, we get $k=4$ is a solution. $k=4$ gives $2^2\cdot 5^2\cdot 7^2=n^2\implies n = 70$. Having both $b^2=6a^2+1$ and $c^2=12a^2+1$ be squares is not possible for larger $a$. So, our only solution is $\boxed{(4,70)}$.
\end{sol}
 
\prob{4}{Magic Math AMC 10 2020}{Find the number of ordered pairs of positive integers $(a, b)$ with $a < b < 2017$ such that $10a$ is divisible by $b$ and $10b$ is divisible by $a$.}

\prob{4}{HMMT Feburary 2016}{For which integers $n \in \{1, 2, \ldots , 15\}$ is $n^n + 1$ a prime number?}

\prob{5}{CNCM Online Round 1}{Consider all possible pairs of positive integers $(a, b)$ such that $a\ge b$ and both $\frac{a^2+b}{a-1}$ and $\frac{b^2+a}{b-1}$
are integers. Find the sum of all possible values of the product $ab$.}

\prob{6}{HMMT Feburary 2017}{Find all pairs $(a, b)$ of positive integers such that $a^{2017} + b$ is a multiple of $ab$.}

\prob{7}{HMMT Feburary 2017}{. Kelvin the Frog was bored in math class one day, so he wrote all ordered triples $(a, b, c)$ of positive integers such that $abc = 2310$ on a sheet of paper. Find the sum of all the integers he wrote down. In
other words, compute
$$\sum_{\substack{abc=2310 \\ a,b,c \in \mathbb{N}}} (a + b + c),$$
where $\mathbb{N}$ denotes the set of positive integers.}

\subsection{Modulo}

\prob{1}{MA$\theta$ 2018}{The number $4^{14}-1$ is divisible by $29$ but $2^{14}-1$ is not. What is the remainder when $2^{14}-1$ is divided by $29$?}

\begin{sol}
$4^{14}-1=(2^{14}-1)(2^{14}+1)=0\pmod{29}$. Since $2^{14}-1\neq 0 \pmod{29}$, $2^{14}+1=0\pmod{29}\implies 2^{14}-1=\boxed{27}\pmod{29}$
\end{sol}

\prob{1}{AMC 12A 2003/18}{Let $n$ be a $5$-digit number, and let $q$ and $r$ be the quotient and the remainder, respectively, when $n$ is divided by $100$. For how many values of $n$ is $q+r$ divisible by $11$?}

\begin{sol}
We have $n=100q+r\implies n\pmod{11}=q+r$. So, $q+r=0\pmod{11}\iff n=0\pmod{11}$. The smallest 5-digit multiple of $11$ is $10010$ and the largest is $99990$. So, there are $\frac{99990-10010}{11}+1=9090-910+1=\boxed{8181}$.
\end{sol}

\prob{1}{BMT 2018}{Find the minimal $N$ such that any $N$-element subset of \{1, 2, 3, 4, . . . 7\} has a subset $S$ such that
the sum of elements of $S$ is divisible by $7$.}

\begin{sol}
First, note that $N\leq 4$. Any $4$-element subset has all of at least one of the following: $\{1,6\},\{2,5\},\{3,4\},\{7\}$ by the Pigeonhole Principle. So, any $4$-element subset has a subset that has a sum divisible by $7$. Also, note $N>3$ because of the counterexample: $6,5,4$. So, $N=\boxed{4}$. 
\end{sol}

\prob{2}{AMC 10B 2019/14}{The base-ten representation for $19!$ is \[121,6T5,100,40M,832,H00,\] where $T$, $M$, and $H$ denote digits that are not given. What is $T+M+H$?}

\begin{sol}
Also, $19!=0\pmod{125}$ by Legendre's so $H=0$. Note that $19!=0\pmod{9}$ so $T+M+H+33=0\pmod{9}\implies T+M+H=3\pmod{9}\implies T+M=3\pmod{9}\implies T+M=3,12$.  We also have $19!=0\pmod{11}\implies H-2+3-8+M-0+4-0+0-1+5-T+6-1+2-1=0\pmod{11}\implies H+M-T+7=0\pmod{11}\implies H+M-T=4\pmod{11}\implies M-T=4\pmod{11}\implies M-T=4$ since $0\leq M,T\leq 9$. Notice that $T+M=3$ is impossible then. So, $T+M=12\implies H+T+M=\boxed{12}$
\end{sol}

\prob{2}{BMT 2018}{What is the remainder when $201820182018\ldots$ [2018 times] is divided by $15$?}

\begin{sol}
Let $N=201820182018\ldots$ [2018 times]. We will find $N\pmod{3}$ and $N\pmod{5}$ and then use CRT. Clearly, $N\pmod{5}=3$. Also, $N\pmod{3}=2018\cdot 2018=(-1)^2=1$. Then, $N\pmod{15}=\boxed{13}$.
\end{sol}

\prob{2}{CNCM PoTD}{Find the number of positive integer $x$ less than $100$ such that 
$$3^x+5^x+7^x+11^x+13^x+17^x+19^x$$ is prime.}

\begin{sol}
Considering $\pmod{3}$, we get $3(1)^x+3(-1)^x=0\pmod{3}$ which is impossible as the expression is clearly $>3$. So, $\boxed{0}$.
\end{sol}

\prob{3}{AMC 10B 2018/16}{Let $a_1,a_2,\dots,a_{2018}$ be a strictly increasing sequence of positive integers such that\[a_1+a_2+\cdots+a_{2018}=2018^{2018}.\]What is the remainder when $a_1^3+a_2^3+\cdots+a_{2018}^3$ is divided by $6$?}

\begin{sol}
Notice that $n^3=n\pmod{2}$ and that $n^3=n\pmod{3}$ by Euler's Totient Theorem. So, $n^3=n\pmod{6}$ by CRT. Then, $a_1^3+a_2^3\cdots + a_{2018}^3=a_{1}+a_{2}\cdots +a_{2018}\pmod{6}=2018^{2018}\pmod{6}$. Now, $2018^{2018}\pmod{2}=0$ and $2018^{2018}\pmod{3}=(-1)^{2018}\pmod{3}=1$. So, by CRT, $2018^{2018}=\boxed{4}\pmod{6}$.
\end{sol}
 
\prob{3}{AMC 10A 2020/18}{Let $(a,b,c,d)$ be an ordered quadruple of not necessarily distinct integers, each one of them in the set ${0,1,2,3}.$ For how many such quadruples is it true that $a\cdot d-b\cdot c$ is odd? (For example, $(0,3,1,1)$ is one such quadruple, because $0\cdot 1-3\cdot 1 = -3$ is odd.)}

\begin{sol}
We have two cases: $ad$ is even and $bc$ is odd or $ad$ is odd and $bc$ is even. If $ad$ is even, then we can do complementary counting and get $4^{2}-2^{2}=12$ ways for $(a,d)$. This is because if $ad$ was odd, this is equivalent to both $a,d$ being odd. If $bc$ is odd, we get $2^{2}=4$ ways for $(b,c)$.  The other case follows similarily. Altogether, are $2\cdot 12\cdot 4=\boxed{96}$ such $(a,b,c,d)$.
\end{sol}

\prob{3}{CNCM Online Round 2}{An ordered pair $(n, p)$ is juicy if $n^2 \equiv 1 \pmod{p^2}$ and $n \equiv -1 \pmod{p}$ for positive integer $n$ and odd prime $p$. How many juicy pairs exist such that $n, p \leq 200$?}

\prob{3}{HMMT Feburary 2018}{ There are two prime numbers $p$ so that $5p$ can be expressed in the form $\lfloor \frac{n^2}{5} \rfloor$ for some positive integer $n$. What is the sum of these two prime numbers?
}

\begin{sol}
The two prime numbers are $5$ and $29$, consider $\pmod{25}$. $\boxed{34}$
\end{sol}


\prob{3}{BMT 2019}{Compute the remainder when the product of all positive integers less than and relatively prime
to 2019 is divided by 2019.}

\prob{4}{BMT 2018}{If $r_{i}$ are integers such that $0 \leq r_{i} < 31$ and $r_{i}$ satisfies the polynomial $x^4 + x^3 + x^2 + x\equiv 30 \pmod{31}$, find 
$$\sum_{i=1}^{4}(r_{i}^2 + 1)^{-1} \pmod{31}.$$ 
where $x^{-1}$ is the modulo inverse of $x$, that is, it is the unique integer y such that $0 < y < 31$
and $xy - 1$ is divisible by $31$.}

\begin{sol}
Note that $x^4+x^3+x^2+x\equiv 30 \pmod{31}\iff x^4+x^3+x^2+x+1 \equiv 0\pmod{31}\iff \frac{x^5-1}{x-1}\equiv 0\pmod{31}$. So, $r_{i}$ are the roots of $x^{5}-1\equiv 0\pmod{31}$ except for $1$. Note that $31=2^{5}-1\implies 2^{5}\equiv 1\pmod{31}$. So, $2$ is a root. Then, $2^{2}=4,2^{3}=8,2^{4}=16$ are also roots. 

Now, we want to compute
$$\frac{1}{5}+\frac{1}{17}+\frac{1}{65}+\frac{1}{257}$$
$$=\frac{1}{5}+\frac{1}{17}+\frac{1}{3}+\frac{1}{9}$$
$$=\frac{1}{5}+\frac{1}{9}+\frac{1}{17}+\frac{1}{3}$$
$$=\frac{14}{45}+\frac{20}{51}$$
$$=\frac{14}{14}+\frac{20}{20}$$
$$=1+1$$
$$=\boxed{2}$$

\end{sol}
\prob{4}{SLLKK AIME 2020}{Smush is a huge Kobe Bryant fan. Smush randomly draws n jerseys
from his infinite collection of Kobe jerseys, each being either the recent \#24 jersey or the
throwback \#8 jersey with equal probability. Let $p(n)$ be the probability that Smush can
divide the $n$ jerseys into two piles such that the sum of all jersey numbers in each pile is
the same. If}

\prob{5}{BMT 2018}{Ankit wants to create a pseudo-random number generator using modular arithmetic. To do so
he starts with a seed $x_{0}$ and a function $f(x) = 2x + 25 \pmod{31}$. To compute the kth pseudo
random number, he calls $g(k)$ defined as follows:

$$g(k) = \begin{cases}
x_{0} & \text{if } k = 0 \\
f(g(k - 1)) & \text{if } k > 0 
\end{cases}$$

If $x_{0}$ is $2017$, compute $\sum_{
j=0}^{2017} g(j) \pmod{31}$.
}

\begin{sol}
$\boxed{21}$
\end{sol}

\prob{5}{PHS HMMT TST 2020}{Find the largest integer $0 < n < 100$ such that $n^2 + 2n$ divides $4(n-1)! + n + 4$.}

\begin{sol}
For $n$ is even, $n^2+2n=(n)(n+2)=4(\frac{n}{2})(\frac{n+2}{2})$. $4(n-1)!=0\pmod{4(\frac{n}{2})(\frac{n+2}}{2})$, then we get $n+4=0\pmod{n^2+2n}$ which is impossible. For $n$ is odd, $n^2+2n=(n)(n+2)$ and $\gcd(n,n+2)=1$. If either of $n, n+2$ are composite, WLOG $n=0\pmod{p}$ for some prime $p<n$, then $(n-1)!=0\pmod{p}\implies n+4=0\pmod{p}\implies 4 =0\pmod{p}$ contradiction. Similar for $n+2$. Then, we have that $n,n+2$ must be both primes. We show that this works. We have $4(n-1)!+n+4\pmod{n}=-4+4=0\pmod{n}$ by Wilsons'. Also, $4(n-1)!+n+4\pmod{n+2}=2+4\frac{-1}{(n+1)(n)}\pmod{n+2}=2+4\frac{-1}{2}\pmod{n+2}=2-2=0\pmod{n+2}$. Since $\gcd(n,n+2)=1$, we're done.

The largest twin primes in the range are $\boxed{71},73$.
\end{sol}

\prob{7}{HMMT November 2014}{Suppose that $m$ and $n$ are integers with $1 \leq m \leq 49$ and $n \ge 0$ such that $m$ divides $n^{n+1} + 1$. What is the number of possible values of $m$?}


\prob{7}{BMT 2018}{How many $1 < n \leq 2018$ such that the set $\{0, 1, 1+2, \ldots, 1+2+3+\cdots+i, \ldots , 1+2+\cdots+n-1\}$
is a permutation of $\{0, 1, 2, 3, 4, \cdots , n - 1\}$ when reduced modulo $n$?}

\begin{sol}
Only $2^{k}$ works so $\boxed{10}$.
\end{sol}

\prob{8}{BMT 2018}{Determine the number of ordered triples $(a, b, c)$, with $0 \leq a, b, c \leq 10$ for which there exists
$(x, y)$ such that $ax^2 + by^2 \equiv c (mod 11)$
}

\prob{8}{BMT 2018}{ Compute the following:
$$\sum_{i=0}^{99} (x^2+1)^{-1} \pmod{199}$$
where $x^{-1}$ is the value $0 \leq y \leq 199$ such that $xy - 1$ is divisible by 199}

\prob{10}{SLKK AIME 2020}{: Let p = 991 be a prime. Let S be the set of all lattice points $(x, y)$,
with $1 \leq x, y \leq p - 1$. On each point $(x, y)$ in S, Olivia writes the number $x^2 + y^2$. Let
$f(x, y)$ denote the product of the numbers written on all points in S that share at least
one coordinate with (x, y). Find the remainder when
$$\sum_{i=1}^{p-2} \sum_{j=1}^{p-2} f(i,j)$$
is divided by p.}

\subsection{Bases}
\prob{4}{HMMT November 2014}{Mark and William are playing a game with a stored value. On his turn, a player may either multiply
the stored value by 2 and add 1 or he may multiply the stored value by 4 and add 3. The first player
to make the stored value exceed 2100 wins. The stored value starts at 1 and Mark goes first. Assuming
both players play optimally, what is the maximum number of times that William can make a move?
(By optimal play, we mean that on any turn the player selects the move which leads to the best possible
outcome given that the opponent is also playing optimally. If both moves lead to the same outcome,
the player selects one of them arbitrarily.)
}

\prob{4}{HMMT November 2013}{ How many of the first 1000 positive integers can be written as the sum of finitely many distinct
numbers from the sequence $3^{0},3^{1},3^{2}\cdots$?}

\prob{6}{HMMT November 2014}{For any positive integers $a$ and $b$, define $a \oplus b$ to be the result when adding a to b in binary (base 2), neglecting any carry-overs. For example, $20 \oplus 14 = 101002 \oplus 11102 = 110102 = 26$. (The operation $\oplus$ is called the exclusive or.) Compute the sum
$$\sum_{k=0}^{2^{2014}-1} (k\oplus \lfloor \frac{k}{2} \rfloor)$$}

\prob{3}{CNCM Online Round 1}{Akshar is reading a 500 page book, with odd numbered pages on the left, and even
numbered pages on the right. Multiple times in the book, the sum of the digits of the two opened pages
are 18. Find the sum of the page numbers of the last time this occurs.}

\begin{sol}
We have three cases: $\overline{abc}, \overline{ab(c+1)}$ and $\overline{ab9}, \overline{a(b+1)0}$ and $\overline{a99}, \overline{(a+1)00}$. For the first case, the sum of digits is $2a+2b+2c+1=1\pmod{2}$ so it is impossible. For the second case, $2a+2b+10=18\implies a+b=4$. This gives $b=4$ and $a=0$ as the maximium solution for this case. For the third case, the sum of digits is $2a+19>18$. So, the maximium solutions are $409,410$ giving a sum of $\boxed{819}$.
\end{sol}

\prob{3}{CNCM Online Round 1}{Define $S(N)$ to be the sum of the digits of $N$ when it is written in base 10, and take
$S^{k}(N) = S(S(. . .(N). . .))$ with $k$ applications of S. The stability of a number $N$ is defined to be the
smallest positive integer $K$ where $S^{K}(N) = S^{K+1}(N) = S^{K+2}(N) = \ldots$. Let $T_{3}$ be the set of all
natural numbers with stability $3$. Compute the sum of the two least entries of $T_{3}$.}

\begin{sol} 
$S^{K}(N)=S^{K+1}(N)\implies S^{K}(N)=S(S^{K}(N))$. Note that for $n\ge 10$, $S(n) <10$. So, $S^{N}(N)$ is a one digit number. So, if a number has stability $3$, this implies that the first time $S^{k}(N)$ is a one-digit number is when $k=3$. Then, $S^{2}(N)\ge 10$ and $S(N)\ge 19$. So, the two smallest are $199,289$. Summing, we get $\boxed{488}$.
\end{sol}

\subsection{Binomial Theorem}
\prob{1}{BMT 2015}{Compute the sum of the digits of $1001^{10}$.}

\subsection{Miscellenous}
\prob{3}{BMT 2015}{Find all integer solutions to
$$x^2+2y^2+3z^2=36$$
$$3x^2+2y^2+z^2=84$$
$$xy+xz+yz=-7$$}

\begin{sol} 
Adding the first two and dividng by $4$ gives $x^2+y^2+z^2=30$. Then, $(x+y+z)^2=x^2+y^2+z^2+2(xy+xz+yz)=16\implies x+y+z=\pm 4$. Also, $(x+y)^2+(x+z)^2+(y+z)^2=2(x^2+y^2+z^2)+2(xy+xz+yz)=46$. 

Let $x+y=a, x+z=b, y+z=c$. We get $a+b+c=\pm 8$ and $a^2+b^2+c^2=46$. The only decomposition of $46$ into squares is $36,9,1$. So, $(6,3,-1)$ and $(-6,-3,1)$ are our only solutions for $(a,b,c)$ (including permutations since it is symmetric). Notice that $(a,b,c)$ being a permutation of $(6,3,-1)$ means $(x,y,z)$ is a permutation of $(5,1,-2)$. Similarily, $(a,b,c)$ being a permutation of $(-6,-3,1)$ means $(x,y,z)$ is a permutation of $(-5,-1,2)$.

We look at which permutations of $(5,1,-2)$ work. We can biject any valid permutation of $(5,1,-2)$ to get a corresponding solution $(-x,-y,-z)$ for permutations of $(-5,-1,2)$. It's clear only $x$ can be $5$ otherwise $2y^2,3z^2>36$. Then, $2y^2+z^2=9\implies y=2,z=-1$. So the only valid solution is $(5,2,-1)$ and the corresponding solution is $(-5,-2,1)$.

We get $\boxed{(5,2,-1),(-5,-2,1)}$.
\end{sol}

\prob{3}{Magic Math AMC 10 2020}{Let $s(n)$ denote the sum of the digits of a positive integer n. Find the number of three-digit positive integers $k \leq 500$ satisfying $s(k) = s(1000 - k)$.}

\prob{4}{BMT 2019}{For a positive integer $n$, define $\phi(n)$ as the number of positive integers less than or equal to $n$
that are relatively prime to $n$. Find the sum of all positive integers $n$ such that $\phi(n)=20$}

\prob{5}{CNCM Online Round 2}{Let $S$ be the set of all ordered pairs $(x, y)$ of integer solutions to the equation
$$6x^2 + y^2 + 6x = 3xy + 6y + x^2y.$$
$S$ contains a unique ordered pair $(a, b)$ with a maximal value of $b$. Compute $a + b$.}

\prob{5}{PHS HMMT TST 2020}{Find the unique triplet of integers $(a, b, c)$ with $a > b > c$ such that $a+b+c = 95$ and $a^2+b^2+c^3=3083$.}

\prob{5}{BMT 2019}{0. Let S(n) be the sum of the squares of the positive integers less than and coprime to n. For
example, $S(5) = 1^2 + 2^2 + 3^2 + 4^2$, but $S(4) = 1^2 + 3^2$.
Let $p = 2^7 - 1 = 127$ and $q = 2^5 - 1 = 31$ be primes. The quantity $S(pq)$ can be written in the
form
$$\frac{p^2q^2}{6}(a-\frac{b}{c})$$
where $a, b$, and $c$ are positive integers, with $b$ and $c$ coprime and $b < c$. Find $a$.
}

\prob{6}{CNCM PoTD}{How many positive integers $k$ are there such that $101\leq k\leq 10000$ and $\lfloor \sqrt{k-100}\rfloor$ is a divisor of $k$?}

\setcounter{problem}{0}
\section{Algebra}

\subsection{Polynomials}
Generally uses the following techniques: Vieta's, Binomial Theorem, Multinomial Theorem, Remainder Theorem, Newton's Sums, Reciprocal Roots Trick, Quadratic Formula (including using Determinant), Finite Differences, $x+\frac{k}{x}$ subsitution, Polynomial Interpolation

\prob{1}{CRMT Math Bowl 2019}{Find the sum of all real numbers such that 
$$\sqrt[4]{16x^4-32x^3+24x^2-8x+1}=5$$
}

\begin{sol}
Note that this is $\sqrt[4]{(2x-1)^4}=|2x-1|$ using the Binomial Theorem. Then, $|2x-1|=5\implies x=3,-2$. The sum is $\boxed{1}$.
\end{sol}

\prob{2}{TAMU 2019}{In the expansion of $(1+ax-x^2)^8$ where a is a positive constant, the coefficient of $x^2$ is $244$. Find the value of $a$}

\prob{3}{HMMT November 2014}{Let $f(x) = x^2 + 6x + 7$. Determine the smallest possible value of $f(f(f(f(x))))$ over all real numbers $x$.}

\prob{3}{HMMT February 2014}{Find the sum of all real numbers x such that $5x^4+10x^3 + 10x^2+5x+11 = 0$}

\begin{sol} 
Symmetric about $1$. Monotonic after $1$ so only two real roots that sum to $1$.
\end{sol}

\prob{3}{BmMT 2014}{Consider the graph of $f(x) = x^3 + x + 2014$. A line intersects this cubic at three points, two
of which have x-coordinates 20 and 14. Find the x-coordinate of the third intersection point}

\begin{sol}
Let the line have the equation $y=mx+b$. Then, $mx+b=x^3+x+2014\implies x^2+(1-m)x+2014-b=0$. Then, the sum of the roots are $0$. So, the third intersection point is $\boxed{-34}$.
\end{sol}

\prob{3}{ARML 2017}{ Compute the number of ordered pairs of integers $(a, b)$ such that the polynomials $x^2 - ax + 24$ and $x^2 - bx + 36$ have one root in common}

\begin{sol}
Let $r$ be the common root. Then, $r^2-ar+24=0\implies a = r + \frac{24}{r}$ and $r^2-br+36=0\implies b = r + \frac{36}{r}$. Since $a,b$ are integers, this implies $r|\gcd(24,36)=12$ ($r$ can be both positive and negative divisors). Then, there are $2\cdot 2\cdot 3 =12$ values for $r$ which then determine $(a,b)$. Also, clearly they don't have two roots in common as both are monic and the constant term is different. Our answer is $\boxed{12}$.
\end{sol}

\prob{3}{AMC 10A 2015/23}{The zeroes of the function $f(x)=x^2-ax+2a$ are integers. What is the sum of the possible values of $a$?}

\begin{sol}
If the roots are integers, then the discriminant must be a perfect square; otherwise, the roots are irrational. Then, $a^2-8a=n^2$ for some positive integer $n$. This gives $(a-4)^2-n^2=16\implies (a-n-4)(a+n-4)=16$. Note that the sum of the roots is $2a-8$. We have factor pairs $(-16,-1),(-8,-2),(-4,-4),(1,16),(2,8),(4,4)$. Note that if the roots are integers, $a$ must also be a integer because $a$ is the sum of the roots by Vieta's. From this, we get $a=-1, 0, 9, 8$. Trying, we get that all of these work. So, the sum is $\boxed{16}$.
\end{sol}

\prob{4}{HMMT Feburary 2017}{Let $Q(x) = a_{0} + a_{1}x + · · · + a_{n}x^n$ be a polynomial with integer coefficients, and $0 \leq ai < 3$ for all $0 \leq i \leq n$.
Given that $Q(\sqrt{3}) = 20 + 17\sqrt{3}$, compute $Q(2)$}

\prob{4}{TAMU 2018}{Suppose $f$ is a cubic polynomial with roots $a,b,c$ such that 
$$a = \frac{1}{3-bc}$$
$$b = \frac{1}{5-ac}$$
$$c = \frac{1}{7-ab}$$
If $f(0)=1$, find $f(abc+1)$.
}

\begin{sol} 
Let the leading coefficient be $k$. We have $f(0)=1\implies abc=\frac{-1}{k}$. Multiply out the expressions to get $3a-abc=1\implies 3a+\frac{1}{k}=1\implies a = \frac{1-\frac{1}{k}}{3}$. Similarily, $b=\frac{1-\frac{1}{k}}{5}, c= \frac{1-\frac{1}{k}}{7}$.  Also, $f(abc)+1=k(abc-a)(abc-b)(abc-c)$.
\end{sol}

\prob{4}{PuMAC 2019}{Let $f(x) = x^2 + 4x + 2$. Let $r$ be the difference between the largest and smallest real solutions
of the equation $f(f(f(f(x)))) = 0$. Then $r = a^{\frac{p}{q}}$ for some positive integers a, p, q so a is
square-free and $p, q$ are relatively prime positive integers. Compute $a + p + q$}

\begin{sol}
Some pattern finding gives $f^{n}(x)=0$ has solutions $x=-2\pm 2^{\frac{1}{2^{n}}}$.
\end{sol}

\prob{4}{HMMT February 2014}{Find all real numbers $k$ such that $r^4+kr^3+r^2+4kr+16=0$ is true for exactly one real number $r$.}

\begin{sol}
Divide by $r^2$ and subsitute $t=r+\frac{4}{r}$.
\end{sol}

\prob{4}{PHS HMMT TST 2020}{Let $a,b,c$ be the distinct real roots of $x^3+2x+5$. Find $(8-a^3)(8-b^3)(8-c^3)$.}

\prob{4}{PuMAC 2019}{Let Q be a quadratic polynomial. If the sum of the roots of $Q^{100}(x)$ (where $Q^{i}(x)$ is defined
by $Q^{1}(x) = Q(x), Q^{i}(x) = Q(Q^{i-1}(x))$ for integers $i \ge 2$) is $8$ and the sum of the roots of $Q$
is $S$, compute $|\log_{2}(S)|$.}

\prob{7}{HMMT Feburary 2017}{A polynomial P of degree $2015$ satisfies the equation $P(n) = \frac{1}{n^2}$ for $n = 1, 2, \ldots , 2016$. Find
$\lfloor 2017P(2017) \rfloor$}

\prob{8}{AIME I 2014/14}{Let $m$ be the largest real solution to the equation

$$ \dfrac{3}{x-3} + \dfrac{5}{x-5} + \dfrac{17}{x-17} + \dfrac{19}{x-19} = x^2 - 11x - 4$$

There are positive integers $a, b,$ and $c$ such that $m = a + \sqrt{b + \sqrt{c}}$. Find $a+b+c$.}

\subsubsection{Newton's Sums}
\prob{3}{BMT 2015}{Let $r, s$, and $t$ be the three roots of the equation $8x^3 + 1001x + 2008 = 0$. Find
$(r + s)^3 + (s + t)^3 + (t + r)^3$}

\prob{4}{BMT 2019}{
Let $r_1, r_2, r_3$ be the (possibly complex) roots of the polynomial $x^3 + ax^2 + bx + \frac{4}{3}$. How many
pairs of integers a, b exist such that $r_{1}^3 + r_{2}^3 + r_{3}^3 = 0$ ?}

\prob{6}{SLKK AIME 2020}{Let $a, b,$ and $c$ be the three distinct solutions to $x^{3} - 4x^{2} + 5x + 1 = 0$.
Find $$(a^3+b^3)(a^3+c^3)(b^3+c^3).$$
}

\subsubsection{Roots of Unity}
\prob{5}{BMT 2019}{ Let $a_n$ be the product of the complex roots of $x^{2n} = 1$ that are in the first quadrant of the
complex plane. That is, roots of the form $a + bi$ where $a, b > 0$. Let $r = a_1 \cdot a_2 \cdot ...\cdot a_{10}$. Find
the smallest integer $k$ such that r is a root of $x^k = 1$}

\prob{6}{BMT 2015}{Evaluate $\sum_{k=0}^{37} (-1)^{k} \binom{75}{2k}$.}

\begin{sol}
Roots of Unity Filter
\end{sol}
\subsection{Complex Numbers}

\prob{4}{HMMT Feburary 2016}{Let $z$ be a complex number such that $|z| = 1$ and $|z - 1.45| = 1.05$. Compute the real part of $z$.}

\prob{4}{2019 AMC 12B}{ How many nonzero complex numbers $z$ have the property that $0, z$, and $z^3$, when
represented by points in the complex plane, are the three distinct vertices of an equilateral
triangle?}

\prob{8}{2015 AIME I/13}{With all angles measured in degrees, the product 
$$\prod_{k=1}^{45}  \csc^2(2k - 1)^{\circ}=m^n,$$
where $m$ and $n$ are integers greater than $1$. Find $m + n$.}
\subsection{Manipulation}
\prob{1}{AMC 10A 2020/7}{The $25$ integers from $-10$ to $14,$ inclusive, can be arranged to form a $5$-by-$5$ square in which the sum of the numbers in each row, the sum of the numbers in each column, and the sum of the numbers along each of the main diagonals are all the same. What is the value of this common sum?}

\begin{sol}
Let the common sum be $s$. Note that if we sum each of the columns, we sum each integer in $-10$ to $14$ exactly once. So, $5s=-10-9-8\cdots +13+14=11+12+13+14=50\implies s=\boxed{10}$.
\end{sol}

\prob{1}{2018 AMC 10A/10}{Suppose that real number $x$ satisfies $$\sqrt{49-x^2}-\sqrt{25-x^2}=3.$$ What is the value of $\sqrt{49-x^2}+\sqrt{25-x^2}$?}

\begin{sol}
Multiplying $(\sqrt{49-x^2}+\sqrt{25-x^2})(\sqrt{49-x^2}-\sqrt{25-x^2}) $ gives $$(49-x^2) - (25-x^2) = 24.$$ We know that $\sqrt{49-x^2}-\sqrt{25-x^2}=3$, so $\sqrt{49-x^2}+\sqrt{25-x^2}$ must be $\frac{24}{3} = \boxed{8}$.
\end{sol}

\prob{2}{2018 BmMT}{Let $x$ be a positive real number so that $x - \frac{1}{x} = 1$.  Compute  $x^8 - \frac{1}{x^8}$.}

\begin{sol} 
We square twice to get $x^2+\frac{1}{x^2}=3$ and $x^4+\frac{1}{x^4}=7$. Then, note that $x^8-\frac{1}{x^8}=(x^4-\frac{1}{x^4})(x^4+\frac{1}{x^4})=7(x^2-\frac{1}{x^2})(x^2+\frac{1}{x^2})=21(x-\frac{1}{x})(x+\frac{1}{x})=21(x+\frac{1}{x})=21(x+\frac{1}{x})$. To find $x+\frac{1}{x}$, we set it to a value $n$. $n^2 = x^2 + \frac{1}{x^2} + 2 \implies n^2 = 5$, and we know $x+\frac{1}{x}$ is positive, so $x+\frac{1}{x} = \sqrt{5}$. So, our final answer is $\boxed{21\sqrt{5}}$.
\end{sol}

\prob{2}{HMMT November 2013}{Evaluate 
$$\frac{1}{2-\frac{1}{2-\frac{1}{2- \cdots\frac{1}{2-\frac{1}{2}}}}}$$
where the digit $2$ appears $2013$ times}

\begin{sol}
We can do some pattern finding when $2$ appears $n$ times. When $n=1$, this is $\frac{1}{2}$. When $n=2$, this is $\frac{2}{3}$. We conjecture that when $2$ appears $n$ times, the expresion is $\frac{n}{n+1}$. We can prove this by induction. With $2$ appearing $n+1$ times, we have $\frac{1}{2-\frac{1}{2-\frac{1}{2\cdots}}} = \frac{1}{2-\frac{n}{n+1}}=\frac{n+1}{n+2}$.
\end{sol}

\prob{2}{Mandelbrot}{If $\frac{x^2}{y^2}=\frac{8y}{x}=z$, find the sum of all possible $z$.
}

\begin{sol} 
Let $\frac{x}{y}=k$. Then, $k^2=\frac{8}{k}\implies k^3 = 8\implies k =2$. Then, $k^2=4$ is only possible $z$. The sum is $\boxed{4}$ 
\end{sol}

\prob{2}{AMC 10A 2018/14}{What is the greatest integer less than or equal to\[\frac{3^{100}+2^{100}}{3^{96}+2^{96}}?\]}

\begin{sol}
Conider $\frac{3^{100}+2^{100}}{3^{96}+2^{96}}=3^4-\frac{3^{4}2^{96}-2^{100}}{3^{96}+2^{96}}$. Clearly, $3^{4}2^{96}>2^{100}$ but $3^{4}2^{96}-2^{100}=(81-16)2^{96}=65\cdot 2^{96}$ is much less than $3^{96}$. So, the expression is less than $81$ but greater than $80$. Our answer is $\boxed{80}$.
\end{sol}

\prob{3}{BMT 2016}{Simplify $\frac{1}{\sqrt[3]{81}+\sqrt[3]{72}+\sqrt[3]{64}}$.}

\begin{sol}
Let $\sqrt[3]{8}=a$ and $\sqrt[3]{9}=b$. Then, it is equivalent to $\frac{1}{a^2+ab+b^2}=\frac{a-b}{a^3-b^3}=\boxed{\sqrt[3]{9}-\sqrt[3]{8}}$.
\end{sol}

\prob{3}{MA$\theta$ 2018}{The solutions to $3\sqrt{2x^2-5x-3}+2x^2-5x=7$ can be written in the form $x=\frac{a \pm \sqrt{b}}{c}$ where $a,b,c$ are positive integers and $x$ is in simplest form. Find $a+b+c$.}

\begin{sol}
Subsitute $\sqrt{2x^2-5x-3}=y$. We get $3y+y^2+3=7\implies y^2+3y-4=\implies (y+4)(y-1)$. Since $y$ is positive, $y=1$. Then, $2x^2-5x-3=1\implies 2x^2-5x-4=0$. By the Quadratic Formula, we get $x=\frac{5 \pm \sqrt{57}}{4}$. We get $5+57+4=\boxed{66}$ as our answer.
\end{sol}

\prob{3}{Mandelbrot Nationals 2008}{
Find the positive real number $x$ for which $5\sqrt{1-x}+5\sqrt{1+x}=7\sqrt{2}$.}

\begin{sol}
Let $\sqrt{1-x}=a$ and $\sqrt{1+x}=b$. We get $5a+5b=7\sqrt{2}\implies a+b=\frac{7\sqrt{2}}{5}\implies a^2+2ab+b^2 = \frac{98}{25}$. We also have $a^2+b^2=2$. So, $ab=\frac{24}{25}\implies \sqrt{1-x^2}=\frac{24}{25}\implies 1-x^2=\frac{576}{625}$. This gives $x=\boxed{\frac{7}{25}}$.
\end{sol}

\prob{4}{NEMO 2019}{Suppose $x$ and $y$ are positive real numbers satisfying
$$\sqrt{xy}=x-y=\frac{1}{x+y}=k$$
Determine $k$.}

\begin{sol}
$$x-y=\frac{1}{x+y}\implies x^2-y^2=1$$
$$x-y=k\implies x^2-2xy+y^2=k^2\implies x^2+y^2=3k^2$$
Combining, we get $x^2=\frac{3k^2+1}{2}$ and $y^2=\frac{3k^2-1}{2}$. Then, $\sqrt{xy}=k\implies x^2y^2=k^4\implies \frac{9k^4-1}{4}=k^4\implies k=5^{-\frac{1}{4}}$.
\end{sol}

\prob{4}{2014 November HMMT}{ Let $a, b, c, x$ be reals with $(a + b)(b + c)(c + a) \neq 0$ that satisfy
$$\frac{a^2}{a+b}=\frac{a^2}{a+c}+20, \frac{b^2}{b+c}=\frac{b^2}{b+a} + 14 \text{, and } \frac{c^2}{c+a}=\frac{c^2}{c+b}+x$$
Compute $x$.}

\begin{sol}
Notice that $\frac{a^2}{a+b}=\frac{a^2}{a+c}+20\implies \frac{a^2(a+c)-a^2(a+b)}{(a+b)(a+c)}=20\implies \frac{a^2(c-b)}{(a+b)(a+c)}=20$. Similarily, $\frac{b^2(a-c)}{(b+c)(b+a)}=14$ and $\frac{c^2(b-a)}{(c+a)(c+b)}=x$. Summing, we get $\frac{a^{2}(c^2-b^2)+b^{2}(a^2-c^2)+c^2(b^2-a^2)}{(a+b)(b+c)(c+a)}=0=34+x\implies x = \boxed{-34}$.
\end{sol}
\prob{5}{Magic Math AMC 10 2020}{Determine the remainder when
$\sum (-1)^{m+n}mn$
is divided by $1009$, where the sum is taken over all pairs of integers $(m, n) $satisfying $1 \leq m <
n \leq 2020$}

\begin{sol}
Notice that by the Distributive Property, $(\sum_{m=1}^{2020} (-1)^{m} m)(\sum_{n=1}^{2020} (-1)^{n} n) = 2\sum_{1\leq m < n\leq 2020} (-1)^{m+n}mn + \sum_{i=1}^{2020} i^2$.  Note that $(\sum_{m=1}^{2020} (-1)^{m} m)(\sum_{n=1}^{2020} (-1)^{n} n)=(\sum_{m=1}^{2020} (-1)^{m} m)^2=(-1+2-3+4\cdots -2019+2020)=1010^2=1\pmod{1009}$. Also, $\sum_{i=1}^{2020} i^2=\frac{2020\cdot 2021\cdot 4041}{6} =\frac{2\cdot 3\cdot 5}{6} \pmod{1009}=5$. 

Then, letting $x=\sum_{1\leq m < n\leq 2020} (-1)^{m+n}mn$,we have 
$$1=2x+5\pmod{1009}$$
$$-2=x\pmod{1009}$$
$$x=\boxed{1007} \pmod{1009}$
\end{sol}

\prob{5}{Math Prizes For Girls 2015}{
Let $S$ be the sum of all distinct real solutions of the equation 
$$\sqrt{x + 2015} = x^2 - 2015.$$
Compute $\lfloor 1/S \rfloor$.  Recall that if $r$ is a real number, then $\lfloor r \rfloor$ (the floor of $r$) is the greatest integer that is less than or equal to $r$}

\begin{sol}
Let $2015=y$. Then, we have $\sqrt{x+y}=x^2-y\implies x+y=x^4-2x^2y+y^2\implies y^2+(-2x^2-1)y+x^4-x=0$. Then, $y=\frac{2x^2+1 \pm (2x+1)}{2}$. Now, we have $2015=x^2+x+1$ or $2015=x^2-x$. These give $x=\frac{-1 \pm \sqrt{8057}}{2}$ and $x=\frac{1 \pm \sqrt{8061}}{2}$. Now, note that we have $x+y\ge 0 \implies x \ge -2015$ and $x^2-y\ge 0\implies |x|\ge \sqrt{2015}$. \\
\\
We can see that $\frac{-1 + \sqrt{8057}}{2}>\sqrt{2015}$ and that $\frac{-1-\sqrt{8057}}{2} < -\sqrt{2015}$. Also, $\frac{1-\sqrt{8061}}{2} > - \sqrt{2015}$ and $\frac{1+\sqrt{8061}}{2} > \sqrt{2015}$. So, we have that our two solutions are $\frac{-1-\sqrt{8057}}{2}$ and $\frac{1+\sqrt{8061}}{2}$. \\
\\
Then, $\frac{1}{S}=\frac{2}{\sqrt{8061}-\sqrt{8057}}=\frac{\sqrt{8061}+\sqrt{8057}}{2}$. So, $89 < \frac{1}{S} < 90$ so our answer is $\boxed{89}$
\end{sol}

\subsection{Series and Sequences}
\prob{1}{djmathman Mock AMC 2013/9}{ Let $p$ and $q$ be numbers with  $|p| <1$ and $|q| <1$ such that \[p+pq+pq^2+pq^3 + \dots = 2  ~~\text{and} ~~ q + qp + qp^2 + \dots = 3.\] What is $100pq$?}

\begin{sol}
Using the formula for an infinite geometric series, we get $\frac{p}{1-q}=2\implies p=2-2q$ and $\frac{q}{1-p}=3\implies q=3-3p\implies p=\frac{3-q}{3}$. So, $\frac{3-q}{3}=2-2q\implies 3-q=6-6q\implies 5q=3\implies q =\frac{3}{5}\implies p =\frac{4}{5}$. Then $100pq=100\cdot \frac{12}{25}=\boxed{48}$.
\end{sol}

\prob{1}{PHS HMMT TST 2020}{What is the value of $\frac{\frac{1}{1^2}+\frac{1}{2^2}+\frac{1}{3^2} \cdots}{\frac{1}{1^2}+\frac{1}{3^2}+\frac{1}{5^2}\cdots}$? Remember that $\frac{1}{1^2}+\frac{1}{2^2}\cdots = \frac{\pi^2}{6}$}

\begin{sol}
Notice that every positive integer can be uniquely represented as $2^{k}\cdot o$ for some non-negative $k$ and odd integer $o$. Then, $\frac{1}{1^2}+\frac{1}{2^2}\cdots = (\frac{1}{1^2}+\frac{1}{2^2}+\frac{1}{4^2}\cdots)(\frac{1}{1^2}+\frac{1}{3^2}\cdots) = \frac{4}{3}(\frac{1}{1^2} + \frac{1}{3^2}\cdots)$. So, our answer is $\boxed{\frac{4}{3}}$.
\end{sol}

\prob{4}{Math Prizes For Girls 2019}{For each integer from 1 through 2019, Tala calculated the product of
its digits. Compute the sum of all 2019 of Tala’s products.}

\prob{4}{HMMT Feburary 2017}{Find the value of
$$\sum_{1\leq a<b<c} \frac{1}{2^{a}3^{b}5^{c}}$$
(i.e the sum of $\frac{1}{2^{a}3^{b}5^{c}}$ over all triples of positive integers $(a, b, c)$ satisfying $a < b < c$)}

\prob{5}{HMMT Feburary 2016}{Let $A$ denote the set of all integers $n$ such that $1 \leq n \leq 10000$, and moreover the sum of the decimal digits of $n$ is $2$. Find the sum of the squares of the elements of $A$.}

\prob{5}{HMMT Feburary 2016}{Determine the remainder when
$$\sum_{i=0}^{2015} \lfloor \frac{2^{i}}{25} \rfloor $$
is divided by $100$, where $\lfloor x \rfloor$ denotes the largest integer not greater than $x$.}
\subsubsection{Telescoping}

\prob{3}{Purple Comet 2015 HS}{
$$\left(1 + \frac{1}{1+2^1}\right)\left(1+\frac{1}{1+2^2}\right)\left(1 + \frac{1}{1+2^3}\right)\cdots\left(1 + \frac{1}{1+2^{10}}\right)= \frac{m}{n},$$ where $m$ and $n$ are relatively prime positive integers. Find $m + n$.}

\subsection{Trigonometry}

\prob{4}{MA$\theta$ 1992}{If $A$ and $B$ are both in $[0,2\pi)$ and $A$ and $B$ satisfy the equations
$$\sin A + \sin B = \frac{1}{3}$$
$$\cos A + \cos B  = \frac{4}{3}$$
find $\cos(A-B)$}

\prob{4}{TAMU 2019}{Simplify $\arctan \frac{1}{1+1+1^2}+ \arctan \frac{1}{1+2+2^2} + \arctan \frac{1}{1+3+3^2} \cdots + \arctan \frac{1}{1+n+n^2}$}

\prob{6}{Purple Comet 2015 HS}{
Let $x$ be a real number between 0 and $\tfrac{\pi}{2}$ for which the function  $3\sin^2 x + 8\sin x \cos x + 9\cos^2 x$ obtains its maximum value, $M$. Find the value of $M + 100\cos^2x$.}

\subsection{Logarithms}
\prob{2}{MA$\theta$ 1991}{Given that $\log_{10}2 \approx 0.3010$, how many digits are in $5^{44}$?}

\prob{3}{PuMAC 2019}{ If $x$ is a real number so $3^x = 27x$, compute $\log_{3}(\frac{3^{3^x}}{x^{3^{3}}})$.}

\prob{4}{MA$\theta$ 1991}{Suppose $a$ and $b$ are positive numbers for which $\log_{4n} 40\sqrt{3} = \log_{3n} 45$, find $n^3$}

\prob{4}{MA$\theta$ 1992}{Suppose $a$ and $b$ are positive numbers for which
 $$\log_{9} a  =\ log_{15} b =\log_{25}(a+2b)$$
What is the value of $\frac{b}{a}$?}

\prob{4}{PHS ARML TST 2017}{Positive real numbers $x, y$, and $z$ satisfy the following system of equations:
$$x^{\log(yz)} = 100$$
$$y^{\log(xz)} = 10$$
$$z^{\log(xy)} = 10\sqrt{10}$$
Compute the value of the expression $(\log(xyz))^2$}

\prob{4}{If $60^{a}=3$ and $60^{b} = 5$, then find $12^{\frac{1-a-b}{2-2b}}$.}

\prob{5}{SLKK AIME 2020}{Let $x$ be a real number in the interval $(0,\frac{\pi}{2})$ such that
$\log_{\sin^2(x)} \cos(x) + \log_{\cos^2(x)} \sin(x) = \frac{5}{4}$.
If $\sin^2(2x)$ can be expressed as $m\sqrt{n} -p$, where $m, n$, and $p$ are positive integers such
that $n$ is not divisible by the square of a prime, find $m + n + p$
}


\subsection{Sequences}
\prob{3}{BMT 2015}{Let $\{a_n\}$ be a sequence of real numbers with $a_1 = -1, a_2 = 2$ and for all $n \ge 3$,
$a_{n+1} - a_{n} - a_{n+2} = 0$.
Find $a_{1} + a_{2} + a_{3} + . . . + a_{2015}$.}

\begin{sol} 
Note that the condition is $a_{n+2}=a_{n+1}-a_{n}$ and that $a_{3}=3,a_{4}=1,a_{5}=-2,a_{6}=-3,a_{7}=-1,a_{8}=2$. So, it repeats with period $6$. 
\end{sol}

\subsection{Functions}
\prob{3}{Mandelbrot Nationals 2009}{Let $f(x)$ be a function defined for all positive real numbers satisfying the conditions $f(x) > 0$ for all $x > 0$ and $f(x-y) = \sqrt{f(xy) + 1}$ for all $x > y > 0$. Determine $f(2009)$.}

\prob{5}{HMMT Feburary 2017}{Let $f : \mathbb{R} \to \mathbb{R}$ be a function satisfying $f(x)f(y) = f(x-y)$. Find all possible values of $f(2017)$.}

\subsection{Inequalities}
\prob{1}{PuMAC 2019}{ Let a, b be positive integers such that $a+b = 10$. Let $\frac{p}{q}$
be the difference between the maximum and minimum possible values of $\frac{1}{a}+\frac{1}{b}$, where $p$ and $q$ are relatively prime positive integers.
Compute $p + q$.}

\prob{2}{Math Prizes For Girls 2019}{The degree measures of the six interior angles of a convex hexagon form
an arithmetic sequence (not necessarily in cyclic order). The common
difference of this arithmetic sequence can be any real number in the
open interval $(-D, D)$. Compute the greatest possible value of $D$.}

\prob{3}{Math Prizes For Girls 2019}{Find the least real number $K$ such that for all real numbers x
and y, we have $(1+20x^2)(1+19y^2)\ge Kxy$. Express your answer in simplified
radical form.}

\prob{4}{HMMT February 2014}{Suppose that $x$ and $y$ are positive real numbers such that $x^2-xy+2y^2=8$.  Find the maximum possible value of $x^2+xy+2y^2$.}

\prob{4}{HMMT November 2013}{ Find the largest real number $\lambda$ such that $a^2 + b^2 + c^2 + d^2 \geq ab + \lambda bc + cd$ for all real numbers $a, b, c, d$.}

\prob{5}{BMT 2019}{Find the number of ordered integer triplets $x, y, z$ with absolute value less than or equal to $100$
such that $2x^2 + 3y^2 + 3z^2 + 2xy + 2xz - 4yz < 5$}

\begin{sol}
 $(x+y)^2+(x+z)^2+2(y-z)^2<5$
\end{sol}

\subsection{Fake Algebra}
\prob{3}{BMT 2019}{Find the maximum value of $\frac{x}{y}$ if x and y are real numbers such that $x^2 +y^2-8x-6y + 20 = 0$.}

\prob{3}{BMT 2015}{Let x and y be real numbers satisfying the equation $x^2 - 4x + y^2 + 3 = 0$. If the maximum
and minimum values of $x^2 + y^2$
are $M$ and $m$ respectively, compute the numerical value of
$M - m$.}

\prob{3}{Magic Math AMC 10 2020}{Find the number of ordered pairs of real numbers $(x, y)$ satisfying $x =\frac{2y}{y^2-1}$ and $y =\frac{2x}{x^2-1}$.
}

\prob{4}{PuMAC 2019}{ Let x and y be positive real numbers that satisfy $(\log x)^2+(\log y)^2 = \log x^2 + \log y^2$.
Compute the maximum possible value of $(\log xy)^2$.
}

\begin{sol} 
Subsitute $\log x =a$, $\log y = b$. You get the equation of a circle $(a-1)^2+(b-1)^2=2$. You want to find the y-intersect of tangent line with slope $-1$ on "top" of the circle. Draw a perpendicular to the line from the center to find the tangency point. This has slope $1$ and it is $\sqrt{2}$ long. So, the coordinates of this tangency is $(2,2)$ and $a+b=4$.
\end{sol}

\prob{7}{HMMT February 2014}{Given that $a, b,$ and $c$ are complex numbers satisfying
$$a^2 + ab + b^2 = 1+ i$$
$$b^2 + bc + c^2 = 2$$
$$c^2 + ca + a^2 = 1,$$
compute $(ab + bc + ca)^2$}

\setcounter{problem}{0}
\section{Geometry}

\subsection{Coordinate Geometry}

\prob{1}{BmMT 2014}{Find the area of the convex quadrilateral with vertices at the points $(-1, 5),(3, 8),(3, -1)$,
and $(-1, -2)$.}

\begin{sol} 
Direct application of Shoelace.
\end{sol}

\prob{2}{CRMT Individuals 2019}{
Let $S$ be the set of all distinct points in the coordinate plane that form an acute isosceles triangle
with the points $(32, 33)$ and $(63, 63)$. Given that a line $L$ crosses $S$ a finite number of times, find
the maximum number of times $L$ can cross $S$.
}

\begin{sol}
Replace $(32,33)$ and $(63,63)$ by $A$ and $B$. Then, we do casework on $AC=BC$ or $CB=AB$ or $CA=BA$. We get a line and two semicircles. A line can intersect a semicircle two times and a line one time.
\end{sol}

\prob{2}{HMMT November 2013}{Plot points A, B, C at coordinates (0, 0), (0, 1), and (1, 1) in the plane, respectively. Let S denote
the union of the two line segments AB and BC. Let X1 be the area swept out when Bobby rotates
S counterclockwise 45 degrees about point A. Let X2 be the area swept out when Calvin rotates S
clockwise 45 degrees about point A. Find $\frac{X_{1}+X_{2}}{2}$}

\prob{3}{Math Prizes For Girls 2019}{Two ants sit at the vertex of the parabola $y = x^2$. One starts walking
northeast (i.e., upward along the line $y = x$) and the other starts
walking northwest (i.e., upward along the line $y = -x$). Each time they
reach the parabola again, they swap directions and continue walking.
Both ants walk at the same speed. When the ants meet for the eleventh
time (including the time at the origin), their paths will enclose $10$
squares. What is the total area of these squares?}

\prob{5}{BMT 2019}{
A regular hexagon has positive integer side length. A laser is emitted from one of the hexagon’s
corners, and is reflected off the edges of the hexagon until it hits another corner. Let $a$ be the
distance that the laser travels. What is the smallest possible value of $a^2$ such that $a > 2019$?
You need not simplify/compute exponents.}

\prob{5}{SLKK AIME 2020}{Mr. Duck draws points $ A = (a, 0), B = (0, b), C = (3, 5)$ and $O = (0, 0)$
such that $a, b > 0$ and $\angle ACB = 45^{\circ}$. If the maximum possible area of $\triangle AOB$ can be
expressed as $m-n\sqrt{p}$ where $m, n,$ and $p$ are positive integers such that p is not divisible
by the square of a prime, find $m + n + p$
}

\subsection{3D Geometry}
\prob{1}{AMC 12A 2009/5}{One dimension of a cube is increased by $1$, another is decreased by $1$, and the third is left unchanged. The volume of the new rectangular solid is $5$ less than that of the cube. What was the volume of the cube?}

\prob{1}{AMC 12A 2008/8}{What is the volume of a cube whose surface area is twice that of a cube with volume 1?}

\prob{1}{BMT 2018}{A cube has side length $5$. Let $S$ be its surface area and $V$ its volume. Find $\frac{S^3}{V^2}$.}

\prob{2}{AMC 12B 2008/18}{On a sphere with a radius of $2$ units, the points $A$ and $B$ are $2$ units away from each other. Compute the distance from the center of the sphere to the line segment $AB.$}

\prob{2}{AMC 12B 2005/16}{Eight spheres of radius 1, one per octant, are each tangent to the coordinate planes. What is the radius of the smallest sphere, centered at the origin, that contains these eight spheres?}

\prob{3}{AIME 1984/9}{In tetrahedron $ABCD$, edge $AB$ has length 3 cm. The area of face $ABC$ is $15\mbox{cm}^2$ and the area of face $ABD$ is $12 \mbox { cm}^2$. These two faces meet each other at a $30^\circ$ angle. Find the volume of the tetrahedron in $\mbox{cm}^3$.}

\prob{3}{HMMT February 2014}{Let $C$ be a circle in the $xy$ plane with radius 1 and center $(0, 0, 0)$, and let P be a point in space
with coordinates $(3, 4, 8)$. Find the largest possible radius of a sphere that is contained entirely in the
slanted cone with base C and vertex P.}

\subsection{Inequalties}
\prob{2}{Math Prizes For Girls}{A paper equilateral triangle with area 2019 is folded over a line parallel
to one of its sides. What is the greatest possible area of the overlap of
folded and unfolded parts of the triangle?}

\prob{3}{BMT 2018}{If $A$ is the area of a triangle with perimeter $1$, what is the largest possible value of $A^2$?}

\begin{sol}
Using Heron's Formula and a smooth argument gives that an equilaterial triangle gives the largest area. 
\end{sol}

\subsection{General}
\prob{1}{MA$\theta$ 2018}{A parallelograms has diagonals of length $10$ and $20$. Find the area inclosed by the circle inscribed in the parallelogram.
}

\prob{1}{TAMU 2019}{An acute isosceles triangle ABC is inscribed in a circle. Through B and C, tangents to
the circle are drawn, meeting at D. If $\angle ABC = 2\angle CDB$, then find the radian measure of
$\angle BAC$.}

\prob{1}{PHS PuMAC TST 2017}{In triangle $ABC$, let $D$ and $E$ be the midpoints of $BC$ and $AC$. Suppose $AD$ and $BE$ meet at $F$. If
the area of $\triangle DEF$ is $50$, then what is the area of $\triangle CDE$?}

\prob{1}{AMC 10A 2019/13}{Let $\triangle ABC$ be an isosceles triangle with $BC = AC$ and $\angle ACB = 40^{\circ}$. Construct the circle with diameter $\overline{BC}$, and let $D$ and $E$ be the other intersection points of the circle with the sides $\overline{AC}$ and $\overline{AB}$, respectively. Let $F$ be the intersection of the diagonals of the quadrilateral $BCDE$. What is the degree measure of $\angle BFC?$}

\prob{2}{AMC 10B 2011/17}{In the given circle, the diameter $\overline{EB}$ is parallel to $\overline{DC}$, and $\overline{AB}$ is parallel to $\overline{ED}$. The angles $AEB$ and $ABE$ are in the ratio $4 : 5$. What is the degree measure of angle $BCD?$}

\prob{2}{Brazil 2007}{Let $ABC$ be a triangle with circumcenter $O$. Let $P$ be the intersection of straight lines $BO$ and $AC$ and $\omega$ be the circumcircle of triangle $AOP$. Suppose that $BO = AP$ and that the measure of the arc $OP$ in $\omega$, that does not contain $A$, is $40^{\circ}$. Determine the measure of the angle $\angle OBC$.}

\prob{2}{BMT 2018}{A $1$ by $1$ square ABCD is inscribed in the circle $m$. Circle $n$ has radius $1$ and is centered around
$A$. Let $S$ be the set of points inside of $m$ but outside of $n$. What is the area of $S$?}

\prob{2}{AMC 10B 2011/18}{Rectangle $ABCD$ has $AB = 6$ and $BC = 3$. Point $M$ is chosen on side $AB$ so that $\angle AMD = \angle CMD$. What is the degree measure of $\angle AMD?$}

\prob{2}{TAMU 2019}{Let $AA_{1}$ be an altitude of triangle $\triangle ABC$, and let $A_{2}$ be the midpoint of the side $BC$.
Suppose that $AA_{1}$ and $AA_{2}$ divide angle $\angle BAC$ into three equal angles. Find the product of the angles of 
$\triangle ABC$ when the angles are expressed in degrees.}

\begin{sol}
Let $\angle BAA_{1}=\angle A_{1}AA_{2}=\angle A_{2}AC = \alpha$. We have that $\triangle ABA_{2}$ is an isosceles triangle as $\angle ABA_{1}=\angle AA_{2}A_{1}=90-\alpha$. Then, as $AA_{1}$ is an altitude, $BA_{1}=A_{1}A_{2}$. Let $BA_{1}=A_{1}A_{2}=x$, then $A_{2}C=BA_{2}=2x$. Consider triangles $BAA_{1}$ and $\triangle A_{1}AC$. We have that $\tan \alpha = \frac{x}{AA_{1}}$ and $\tan 2\alpha = \frac{3x}{AA_{1}}$. So, $\frac{\tan 2\alpha}{\tan \alpha}= 3 \implies \tan \alpha = \frac{1}{\sqrt{3}}\implies \alpha = 30$. So, $\angle BAC = 90$, $\angle ABC = 60$, and $\angle ACB=30$. The product is $\boxed{162000}$.
\end{sol}

\prob{2}{PHS PuMAC TST 2017}{A trapezoid has area 32, and the sum of the lengths of its two bases and altitude is 16. If one of the
diagonals is perpendicular to both bases, then what is the length of the other diagonal?}

\prob{2}{Autumn Mock AMC 10}{Equilateral triangle ABC has side length $6$. Points $D, E, F$ lie within the lines $AB, BC$ and $AC$ such that $BD=2AD$, $BE=2CE$, and $AF=2CF$. Let $N$ be the numerical value of the area of triangle $DEF$. Find $N^2$.}

\prob{3}{AHSME 1984/28}{Triangle ABC has area 10. Points D, E, and F, all distinct from A, B, and C, are on sides AB, BC, and CA, respectively, and AD = 2, DB = 3. Triangle ABE and quadrilateral
DBEF have equal areas s. Find s.}

\prob{3}{HMMT February 2014}{
In quadrilateral ABCD, $\angle DAC = 98, \angle DBC = 82, \angle BCD = 70$, and $BC = AD$. Find $\angle ACD$.}

\begin{sol}
Reflect.
\end{sol}

\prob{3}{FARML 2012/6}{In triangle $ABC,$ $AB=7,$ $AC=8,$ and $BC=10.$ $D$ is on $AC$ and $E$ is on $BC$ such that $\angle AEC=\angle BED=\angle B+\angle C.$ Compute the length $AD.$}

\prob{3}{HMMT November 2013}{Let ABC be an isosceles triangle with $AB = AC$. Let D and E be the midpoints of segments $AB$ and $AC$, respectively. Suppose that there exists a point F on ray $\overrightarrow{DE}$ outside of $ABC$ such that triangle $BFA$ is similar to triangle $ABC$. Compute $\frac{AB}{BC}$ .}

\prob{3}{Magic Math AMC 10 2020}{Two concentric circles have radii $\sqrt{2}$ and $2$. Three points $A, B, C$ are chosen on the larger circle such that $\triangle ABC$ is equilateral. Find the area of the region in the smaller circle that is also inside the triangle.}

\prob{3}{CNCM Online Round 2}{There is a rectangle $ABCD$ such that $AB = 12$ and $BC = 7$. $E$ and $F$ lie on sides $AB$
and CD respectively such that $\frac{AE}{EB} = 1$ and $\frac{CF}{FD} =\frac{1}{2}$. Call $X$ the intersection of $AF$ and $DE$. What is
the area of pentagon $BCFXE$?
}
\prob{4}{Mandelbrot Nationals 2009}{Triangle ABC has sides of length $AB =\sqrt{41}, AC = 5$,and $BC = 8$. Let O be the center of the circumcircle of $\triangle ABC$, and let $A'$ be the point diametrically opposite $A$, as shown. Determine the area of $\triangle A'BC$.
}

\prob{4}{HMMT February 2014}{Triangle ABC has sides $AB = 14,BC = 13,$ and $CA = 15$. It is inscribed in circle, which has center $O$. Let $M$ be the midpoint of $AB$, let $B'$ be the point on  diametrically opposite $B$, and let $X$ be the intersection of $AO$ and $MB'$. Find the length of $AX$.}

\begin{sol}
AX is the centroid of $ABB'$.
\end{sol}

\prob{4}{AIME 1989}{Triangle $ABC$ has an right angle at $B$ and contains a point $P$ such that $AP=10$, $BP=6$, and $\angle APC = \angle CPB =\angle BPA$. Find $CP$.
}
\begin{sol}
Law of Cosines and Pythagorean Theorem gives $CP=33$.
\end{sol}

\prob{4}{106 Geometry Problems}{In triangle $ABC$, medians $BB_{1}$ and $CC_{1}$ are perepndicular. Given that $AC=19$ and $AB=22$, find $BC$.}

\begin{sol}
Let $BG=2x, GB_{1}=x$ and $CG=2y, GC_{1}=y$. Set systems of equations and solve.
\end{sol}

\prob{4}{PHS ARML TST 2017}{An algorithm starts with an equilateral triangle $A_{0}B_{0}C_{0}$ of side length 1. At step k, points $A_k, B_k,
$and $C_k$ are chosen on line segments $B_{k-1}C_{k-1}, C_{k-1}A_{k-1}$ and $A_{k-1}B_{k-1}$ respectively, such that

$$B_{k-1}A_{k}:A_{k}C_{k-1}=1:1$$
$$C_{k-1}B_{k}:B_{k}A_{k-1}=1:2$$ 
$$A_{k-1}C_{k}:C_{k}B_{k-1}=1:3$$

What is the value of the infinite series:
$$\sum_{i=0}^{\infty} \text{Area}[\triangle A_{k}B_{k}C_{k}]$$}

\prob{4}{AIME 2005}{In quadrilateral $ABCD$, let $BC=8, CD=12, AD=10$ and $\angle A = \angle B = 60^{\circ}$.}

\begin{sol}
Extend $AD$ and $BC$ to make an equilateral triangle and then Law of Cosines.
\end{sol}

\prob{4}{HMMT November 2013}{ Let ABC be a triangle and D a point on BC such that $AB=\sqrt{2}, BC=\sqrt{3}, \angle BAD = 30^{\circ}$, and$\angle CAD = 45^{\circ}$. Find AD.}

\prob{4}{CMIMC 2020}{
Let $ABC$ be a triangle with centroid $G$ and $BC = 3$. If $ABC$ is similar to $GAB$, compute the area of $ABC$.
}

\prob{4}{Magic Math AMC 10 2020}{Isosceles triangle $\triangle ABC$ has $AB = AC$ and circumcenter O. The circle with diameter $AO$ intersects segment $BC$ at $P$ and $Q$, with $P$ closer to $B$, such that $BP = QC = 4$ and $PQ = 6$.
Compute the area of $\triangle A$}

\prob{5}{PHS HMMT TST 2020}{$\triangle$ ABC has side lengths $AB = 11, BC = 13, CA = 20$. A circle is drawn with diameter $AC$. Line $AB$
intersects the circle at $D \neq A$, and line BC intersects the circle at $E \neq B$. Find the length of $DE$.}

\prob{5}{CMIMC 2020}{
Let $ABC$ be a triangle with centroid $G$ and $BC = 3$. If $ABC$ is similar to $GAB$, compute the area of $ABC$.
}

\prob{5}{Magic Math AMC 10 2020}{Six points $A, B, C, D, E, F$ are selected on a circle with center $O$ such that $ABCDEF$ is an equiangular hexagon with $AB = CD = EF < BC = DE = F A$. Diagonals $AD, BE, F C$ are
drawn, intersecting at points $X, Y, Z$. The three circles passing through $O$ and two of $X, Y, Z$
are each internally tangent to $O$. Find $BC/AB$.}

\prob{6}{SLKK AIME 2020}{Cyclic quadrilateral $AXBY$ is inscribed in circle $\omega$ such that $AB$ is a
diameter of $\omega$. $M$ is the midpoint of $XY$ and $AM = 13, BM = 5$, and $AB = 16$. If the
area of AXBY can be expressed as $m\sqrt{p}+n$, where $m, n$, and $p$ are positive integers such
that $m$ and $n$ are relatively prime and $p$ is not divisible by the square of a prime, find
the remainder when $m + n + p$ is divided by $1000$.}

\prob{8}{SLKK AIME 2020}{Squares $ABCD$ and $DEFG$ are drawn in the plane with both sets of
vertices $A, B, C, D$ and $D, E, F, G$ labeled counterclockwise. Let P be the intersection of
lines $AE$ and $CG$. If $DA = 35, DG = 20$, and $BF = 25\sqrt{2}$, find $DP^2$.}

\begin{sol}
Spiral Similarity from ABCD to DEFG.
\end{sol}

\setcounter{problem}{0}
\section{Misc}
These are problems that don't really fall into any other category at all.
\subsection{General}
\prob{2}{Math Prizes For Girls 2019}{Let $a_{1}, a_{2}, \ldots , a_{2019}$ be a sequence of real numbers. For every five
indices $i, j, k$, and $m$ from $1$ through $2019$, at least two of the
numbers $a_{i},a_{j},a_{k}$ and $a_{m}$ have the same absolute value. What
is the greatest possible number of distinct real numbers in the given
sequence?}

\begin{sol}
There can't be five or more distinct absolute values because then we can just take those five. It's clear that only having four distinct absolute values work. By the Pigeonhole Principle, at least two are the same. Then, from those four, we can have at most $4\cdot 2=\boxed{8}$ distinct real numbers.
\end{sol}

\subsection{Games}

\prob{2}{BmMT 2016}{
Suppose you have a $20 \times 16$ bar of chocolate squares. You want to break the bar into smaller
chunks, so that after some sequence of breaks, no piece has an area of more than 5. What is the
minimum possible number of times that you must break the bar?}

\begin{sol} 
By the Pigeon Hole Principle, at the end, there should be at least $\frac{320}{5}=64$ pieces. It takes at least $63$ moves to break it into $64$ pieces as each move creates one more piece. We can also construct a sequence of moves by noticing that this is equalty case of the Pigeon Hole Principle so each piece must have exactly $5$ squares. Break it into $16$ of $20\times 1$ pieces. This uses $15$ moves. Then, break each of those pieces into $4$ of $5\times 1$ pieces. This uses $3\cdot 16=48$ moves. So, our answer is $\boxed{63}$
\end{sol}

\prob{3}{BMT 2015}{Two players play a game with a pile with N coins is on a table. On a player’s turn, if there
are n coins, the player can take at most $\frac{n}{2} + 1$ coins, and must take at least one coin. The
player who grabs the last coin wins. For how many values of N between 1 and 100 (inclusive)
does the first player have a winning strategy?}

\subsection{Logic}
\prob{2}{BmMT 2014}{Alice, Bob, Carl, and Dave are either lying or telling the truth. If the four of them make the
following statements, who has the coin?
\begin{center}
Alice: I have the coin.

Bob: Carl has the coin.

Carl: Exactly one of us is telling the truth.

Dave: The person who has the coin is male.
\end{center}}

\begin{sol}
Analyzing, Carl must be lying since if he was telling the truth, everyone else are lying which means Alice can't have the coin but the person who has the coin isn't male which is a contradiction. Also note that all of them can't be lying by similar reasoning. So, either two, three, or four are telling the truth. Four telling the truth is impossible as Carl's statement is false. Three telling the truth means everyone but Carl is telling the truth which is impossible as Alice and Dave's statements conflict. So, two must be telling the truth. Dave's and Alice's statements are true if and only if the other is false. If Dave is false and Alice is true, then Bob must also be false which is a contradiction to the fact two are telling the truth. If Dave is true and Alice is false, then Bob must be telling the truth and \boxed{Carl} has the coin. This is the only possible case.
\end{sol}

\prob{2}{Berkeley Math Circle 2013}{
Ten people sit side by side at a long table, all facing the same direction. Each of them is either a knight (and always
tells the truth) or a knave (and always lies). Each of the people announces: “There are more knaves on my left than
knights on my right.” How many knaves are in the line?
}

\begin{sol}
$5$ knaves on the left, $5$ knights on the right gives a valid solution. So, $\boxed{5}$.
\end{sol}
\end{document}