\documentclass[11pt]{article}
\usepackage{william}
\usepackage{comment}
\usepackage[inline]{asymptote} 
\title{Collected Problems: Computational}
\author{William Dai}
\date{Started June 30, 2020}
\DeclareMathOperator{\lcm}{lcm}
\DeclareMathOperator{\cis}{cis}

\excludecomment{sol}
\begin{document}
\maketitle

%%-----------------------------------------------------------------------------------------------------------------------------------%%
%%%%%%%%%%%%%%%%%%%%%%%%%%%%%%%%%%%%%%%%%%%%%%%%%%%%%
%%-----------------------------------------------------------------------------------------------------------------------------------%%
\section{Introduction}
I use the following scheme: 1 point is roughly AMC 10 8-14 level. 2 points is roughly AMC 10 \# 15-17 level, 3 points are AMC 10 \# 18-21 level, 4 points are AMC 10 \# 22-23 level and 5 points are \# 24-25 level. Furthermore, 6 points are \#6-8 AIME level, 7 points are \# 9-11 AIME level,   8 points are \#12-13 AIME level, 9 points are \#14-15 AIME level, and 10 points are a hypothetical \#16-18 AIME level. 10 pointers are usually olympiad-style problems that require several major lemmas and have very long solutions.

Most of these problems are from more obscure contests that will serve as good AIME and AMC practice. 

%%-----------------------------------------------------------------------------------------------------------------------------------%%
%%%%%%%%%%%%%%%%%%%%%%%%%%%%%%%%%%%%%%%%%%%%%%%%%%%%%
%%-----------------------------------------------------------------------------------------------------------------------------------%%
\section{Combinatorics}
%%-----------------------------------------------------------------------------------------------------------------------------------%%
\subsection{Casework}
\prob{1}{ARML 2014}{Let $A, B$, and $C$ be randomly chosen (not necessarily distinct) integers between 0 and 4
inclusive. Pat and Chris compute the value of $A+B \cdot C$ by two different methods. Pat follows
the proper order of operations, computing $A+(B \cdot C)$. Chris ignores order of operations,
choosing instead to compute $(A + B) \cdot C$. Compute the probability that Pat and Chris get
the same answer.}

\begin{sol}
$\boxed{\frac{9}{25}}$
\end{sol}

\prob{2}{BmMT 2014}{Call a positive integer top-heavy if at least half of its digits are in the set \{7, 8, 9\}. How many
three digit top-heavy numbers exist? (No number can have a leading zero.)}

\begin{sol}
Consider $7,8,9$ to be the same and assign actual values at end, do casework on number of $7,8,9$.

$\boxed{207}$
\end{sol}

\prob{3}{PuMAC 2019}{Suppose Alan, Michael, Kevin, Igor, and Big Rahul are in a running race. It is given that
exactly one pair of people tie (for example, two people both get second place), so that no other
pair of people end in the same position. Each competitor has equal skill; this means that each
outcome of the race, given that exactly two people tie, is equally likely. The probability that
Big Rahul gets first place (either by himself or he ties for first) can be expressed in the form
$m/n$, where m, n are relatively prime, positive integers. Compute $m + n$.}

\begin{sol} 
First, find total number of outcomes. Then, casework on if Big Rahul wins by himself or if he ties for first
\end{sol}

\prob{3}{2018 Memorial Day Mock AMC 10}{In the $xy$-coordinate plane, three distinct points with integer $x$- and $y$- coordinates are randomly chosen within the area bounded by $-1 \leq x \leq 1$ and $-1 \leq y \leq 1$. What is
the probability that the three points, when connected, form a triangle with an area of
$1$?}

\begin{sol}
$\boxed{\frac{8}{21}}$
\end{sol}

\prob{3}{PuMAC 2019}{Prinstan Trollner and Dukejukem are competing at the game show WASS. Both players spin
a wheel which chooses an integer from 1 to 50 uniformly at random, and this number becomes
their score. Dukejukem then flips a weighted coin that lands heads with probability 3/5. If
he flips heads, he adds 1 to his score. A player wins the game if their score is higher than
the other player’s score. The probability Dukejukem defeats the Trollner to win WASS equals
$m/n$ where $m, n$ are co-prime positive integers. Compute $m + n$.}

\begin{sol} 
Casework on if Dukejukem flips heads or tails on coin.
\end{sol}

\prob{3}{Scrabber94 Mock AMC 10 2020}{Robert writes all positive divisors of the number $216$ on separate slips of
paper, then places the slips into a hat. He randomly selects three slips from
the hat, with replacement. What is the probability that the product of the
numbers on the three slips Robert selects is a divisor of $216$?}

\begin{sol}
$\boxed{\frac{25}{256}}$
\end{sol}

\prob{4}{Scrabber94 Mock AMC 10 2020}{How many ways can six people of different heights stand in line such that for
all $1 \leq k \leq 6$, the kth tallest person must stand next to either the (k + 1)th
or (k - 1)th tallest person (or both)? In particular, the tallest person must
stand next to the second tallest person, and the shortest person must stand
next to the second shortest person.}

\begin{sol}
$\boxed{54}$
\end{sol}

\prob{4}{Purple Comet 2015 HS}{
Seven people of seven different ages are attending a meeting. The seven people leave the meeting one at a
time in random order. Given that the youngest person leaves the meeting sometime before the oldest
person leaves the meeting, the probability that the third, fourth, and fifth people to leave the meeting do so in order of their ages (youngest to oldest) is $\frac{m}{n}$ , where m and n are relatively prime positive integers. Find $m + n$.}

\begin{sol}
 Casework on if youngest or oldest in third, fourth, fifth block. $\boxed{25}$
\end{sol}

\prob{4}{Payback Mock AMC 10}{Let $T$ be the set of positive integer divisors of $27000$. Find the number of positive
integers that can be expressed as a product of $3$ pairwise distinct elements of $T$.}

\begin{sol}
$\boxed{944}$
\end{sol}

\prob{4}{PersonPsychopath\rq{}s 2nd 2015-2016 Mock AMC 10}{A number is called elegant if, for all $0 \leq d \leq 9$, the digit $d$ does not appear more than $d$ times. For example, the number $3$ cannot appear more than $3$ times in an elegant number (and the digit $0$ cannot appear at all). How many positive $4$-digit elegant numbers exist? Some numbers to include
are $1345, 5999, 6554$ and $7828$.}

\begin{sol}
Complementary count and casework on if $1,2,3$ is overused. Note that there are no overlaps.
$\boxed{6110}$
\end{sol}

\prob{4}{AIME I 2020/5}{Six cards numbered $1$ through $6$ are to be lined up in a row. Find the number of arrangements of these six cards where one of the cards can be removed leaving the remaining five cards in either ascending or descending order.}

\begin{sol}
$\boxed{52}$
\end{sol}

\prob{5}{AMC 10A 2020}{
Jason rolls three fair standard six-sided dice. Then he looks at the rolls and chooses a subset of the dice (possibly empty, possibly all three dice) to reroll. After rerolling, he wins if and only if the sum of the numbers face up on the three dice is exactly $7.$ Jason always plays to optimize his chances of winning. What is the probability that he chooses to reroll exactly two of the dice?}

\begin{sol}
$\boxed{\frac{7}{36}}$
\end{sol}

\prob{5}{AMC 12B 2017/22}{Abby, Bernardo, Carl, and Debra play a game in which each of them starts with four coins. The game consists of four rounds. In each round, four balls are placed in an urn---one green, one red, and two white. The players each draw a ball at random without replacement. Whoever gets the green ball gives one coin to whoever gets the red ball. What is the probability that, at the end of the fourth round, each of the players has four coins?}

\prob{5}{HMMT November 2013}{Find the number of positive integer divisors of $12!$ that leave a remainder of $1$ when divided by $3$.}

\prob{6}{HMMT November 2014}{Consider the set of 5-tuples of positive integers at most $5$. We say the tuple $(a_1, a_2, a_3, a_4, a_5)$ is perfect if for any distinct indices $i, j, k$, the three numbers $a_i
, a_j , a_k$ do not form an arithmetic progression (in any order). Find the number of perfect 5-tuples.}
\begin{sol} 
Casework on parity of exponent of $2,5,11$. Note that there can't be any $3$ and $7$ can be included or excluded without consequence.
\end{sol}

\prob{5}{HMMT February 2014}{. Find the number of triples of sets $(A,B,C)$ such that:
\begin{enumerate}
\item $A,B,C \subseteq \{1, 2, 3,... , 8\}$.
\item $|A \cap B| = |B \cap C| = |C \cap A| = 2$.
\item $|A| = |B| = |C| = 4$.
\end{enumerate}
}

\begin{sol}
$\boxed{45360}$
\end{sol}

\prob{7}{HMMT February 2014}{Six distinguishable players are participating in a tennis tournament. Each player plays one match of
tennis against every other player. There are no ties in this tournament—each tennis match results
in a win for one player and a loss for the other. Suppose that whenever A and B are players in the
tournament such that A wins strictly more matches than B over the course of the tournament, it is also
true that A wins the match against B in the tournament. In how many ways could the tournament
have gone?}

\begin{sol}
$\boxed{2048}$
\end{sol}

%%-----------------------------------------------------------------------------------------------------------------------------------%%
\subsubsection{PIE}
\prob{1}{}{How many integers from $1$ to $100$ (inclusive) are multiples of $2$ or $3?$}

\begin{sol}
There are $50$ multiples of $2$ and $33$ multiples of $3$. If and only if a number is both a multiple of $2$ and $3$, then it is a multiple of $6$. There are $16$ multiples of $6$. Our answer is $50+33-16=\boxed{67}$.
\end{sol}

\prob{1}{AMC 10B 2017/13}{There are $20$ students participating in an after-school program offering classes in yoga, bridge, and painting. Each student must take at least one of these three classes, but may take two or all three. There are $10$ students taking yoga, $13$ taking bridge, and $9$ taking painting. There are $9$ students taking at least two classes. How many students are taking all three classes?}

\prob{8}{SLKK AIME 2020}{
Andy the Banana Thief is trying to hide from Sheriff Buffkin in a row of
6 distinct houses labeled 1 through 6. Andy and Sheriff Buffkin each pick a permutation
of the 6 houses, chosen uniformly at random. On the $n^{\text{th}}$ day, with $1 \leq n \leq 6$, Andy
and Sheriff Buffkin visit the $n^{\text{th}}$ house in their respective permutations, and Andy is
caught by the Sheriff on the first day they visit the same house. For example, if Andy’s
permutation is $1, 3, 4, 5, 6, 2$ and Sheriff Buffkin’s permutation is $3, 4, 1, 5, 6, 2$, Andy is
caught on day $4$. Given that Sheriff Buffkin catches Andy within 6 days and the expected
number of days it takes to catch Andy can be expressed as $\frac{a}{b}$
for relatively prime positive
integers $a$ and $b$, find the remainder when $a + b$ is divided by $1000$.}

\begin{sol} 
Notice that we can fix Sheriff Buffkin's permutation to just be $1,2,3,4,5,6$ because for each of Buffkin's permutation, the probability of being caught on the $n$th day clearly doesn't change. The number of Andy's permutations such that he does get caught is $6!-D_{6}=455$. Then, we do casework on day caught and then use complementary PIE. If he is caught on the $n$th day, the $n$th number inf Andy's permutation is clearly $n$. Then, we do PIE on the number of permutations  such that at least one of the previous numbers are the same. This turns out to be $\binom{n-1}{1}\cdot 4! - \binom{n-1}{2}\cdot 3! + \binom{n-1}{3}\cdot 2! \cdots$. Then, the probability that he is caught on the $n$th day is $\frac{5!-\binom{n-1}{1}\cdot 4! + \binom{n-1}{2}\cdot 3! - \binom{n-1}{3}\cdot 2! \cdots}{6!}$.


Doing the computation for all days, we get $\frac{1331}{455}\implies \boxed{786}$.
\end{sol}
%%-----------------------------------------------------------------------------------------------------------------------------------%%
\subsection{Perspectives}
\prob{1}{AIME I 2002/1}{Many states use a sequence of three letters followed by a sequence of three digits as their standard license-plate pattern. Given that each three-letter three-digit arrangement is equally likely, the probability that such a license plate will contain at least one palindrome (a three-letter arrangement or a three-digit arrangement that reads the same left-to-right as it does right-to-left) is $\dfrac{m}{n}$, where $m$ and $n$ are relatively prime positive integers. Find $m+n.$}

\begin{sol}
Calculate the complementary probability. The probability that the three numbers aren't a palindrome is $1\cdot 1\cdot \frac{9}{10}=\frac{9}{10}$. The probability the three lettters aren't a palindrome is $1\cdot 1\cdot \frac{25}{26}=\frac{25}{26}$. Multiplying together, the probability both aren't a palindrome is $\frac{9\cdot 25}{10\cdot 26}=\frac{9\cdot 5}{2\cdot 26}=\frac{45}{52}$. So, our wanted probability is $1-\frac{45}{52}=\frac{7}{52}\implies \boxed{59}$.
\end{sol}

\prob{1}{PuMAC 2019}{How many ways can you arrange 3 Alice’s, 1 Bob, 3 Chad’s, and 1 David in a line if the Alice’s
are all indistinguishable, the Chad’s are all indistinguishable, and Bob and David want to be
adjacent to each other? (In other words, how many ways can you arrange 3 A’s, 1 B, 3 C’s,
and 1 D in a row where the B and D are adjacent?)}

\begin{sol} 
Consider BD as one block of E and then multiply by $2!=2$ at the end to permute the B and D's. We want to arrange 3 A's, 1 E, and 3 C's. There are $\frac{7!}{3!3!}=\frac{5040}{6\cdot 6}=140$ such permutations. So, the total number of ways is $\boxed{280}$
\end{sol}

\prob{2}{MA$\theta$ 2016}{The product of any two of the elements of the set $\{30, 54, N\}$ is divisible by the third. Find
the number of possible values of N.}

\begin{sol}
 Consider the primes $2,3,5$ separately and get independent inequalities.
\end{sol}

\prob{2}{AMC 10B 2017/17}{Call a positive integer $monotonous$ if it is a one-digit number or its digits, when read from left to right, form either a strictly increasing or a strictly decreasing sequence. For example, $3$, $23578$, and $987620$ are monotonous, but $88$, $7434$, and $23557$ are not. How many monotonous positive integers are there?}


\prob{2}{AIME II 2002/9}{Let $\mathcal{S}$ be the set $\lbrace1,2,3,\ldots,10\rbrace$ Let $n$ be the number of sets of two non-empty disjoint subsets of $\mathcal{S}$. (Disjoint sets are defined as sets that have no common elements.) Find the remainder obtained when $n$ is divided by $1000$.}

\begin{sol} 
We count the number of ordered pairs for now and divide by two at the end. For each element, we have $3$ choices: in either of the sets or in neither. But, we also need to subtract off when at least one of the subsets is empty. If at least one is empty, there's $2$ choices for each element. If both are empty, there is $1$ choice for each element. We also have to divide by two in the end to make it unordered (and note that it's impossible for the two sets to be the same). 

$$\frac{3^{10}-2\cdot 2^{10}+1}{2} = \frac{59049-2048+1}{2}=\frac{57002}{2}=28\boxed{501}$$
\end{sol} 

\prob{2}{AMC 10B 2018/22}{Real numbers $x$ and $y$ are chosen independently and uniformly at random from the interval $[0,1]$. What is the probability that $x,y,$ and $1$ are the side lengths of an obtuse triangle?}

\begin{sol} 
Note that $1>x,y$ so $1$ is opposite the obtuse angle. This gives the inequalty $1>x^2+y^2$. Also, by the Triangle Inequalty, $1 < x+y$. Graphing this, we want the area under the quarter unit circle centered at $(0,0)$ but above the line from $(0,1)$ to $(1,0)$. This has area $\boxed{\frac{\pi-2}{4}}$
\end{sol}

\prob{2}{AMC 10B 2020/23}{Square $ABCD$ in the coordinate plane has vertices at the points $A(1,1), B(-1,1), C(-1,-1),$ and $D(1,-1).$ Consider the following four transformations:
    
\begin{itemize}
    \Item $L,$ a rotation of $90^{\circ}$ counterclockwise around the origin;

    \Item $R,$ a rotation of $90^{\circ}$ clockwise around the origin;
    
    \Item $H,$ a reflection across the $x$-axis; and

    \Item $V,$ a reflection across the $y$-axis.
\end{itemize}

Each of these transformations maps the squares onto itself, but the positions of the labeled vertices will change. For example, applying $R$ and then $V$ would send the vertex $A$ at $(1,1)$ to $(-1,-1)$ and would send the vertex $B$ at $(-1,1)$ to itself. How many sequences of $20$ transformations chosen from $\{L, R, H, V\}$ will send all of the labeled vertices back to their original positions? (For example, $R, R, V, H$ is one sequence of $4$ transformations that will send the vertices back to their original positions.)}

\begin{sol} 
Notice that after any sequence of $19$ moves, we have a unique move that takes the square back to its original square. So, the total number of sequences is $4^{19}=\boxed{2^{38}}$.
\end{sol}

\prob{2}{CNCM Online Round 1}{Pooki Sooki has 8 hoodies, and he may wear any of them throughout a 7 day week. He changes hishoodie exactly $2$ times during the week, and will only do so at one of the 6 midnights. Once he changes out of a
hoodie, he never wears it for the rest of the week. The number of ways he can wear his hoodies throughout the week
can be expressed as $\frac{8!}{2^k}$ . Find $k$.}

\begin{sol} 
There is $8\cdot 7\cdot 6$ ways to choose the hoodies Pooki wears and there are $\binom{6}{2}=15=3\cdot 5$ ways to choose the midnights he changes. The number of ways is then $8\cdot 7\cdot 6\cdot 5\cdot 3 = \frac{8!}{4\cdot 2}=\frac{8!}{2^{3}}$. Our answer is $\boxed{3}$.
\end{sol}

\prob{3}{BmMT 2014}{If you roll three regular six-sided dice, what is the probability that the three numbers showing
will form an arithmetic sequence? (The order of the dice does matter, but we count both (1,
3, 2) and (1, 2, 3) as arithmetic sequences.)}

\begin{sol} 
We count the number of ordered triplets $(a,b,c)$ such that two of the numbers average to the third number. WLOG $a\leq b\leq c$. Note that if $a,b,c$ are distinct, then this corresponds to $6$ ordered triplets. Otherwise, $a=b=c$ and this clearly corresponds to only $1$ ordered triplet.

For the first case, notice that this is identical to choosing two numbers $a,c$ (unordered) and seeing if their average is an integer. This happens if $a,c$ are the same parity. Then, this gives $\binom{3}{2} + \binom{3}{2}=6$ $(a,b,c)$ such that $a\leq b\leq c$. We multiply by $6$ to get the full number of ordered triplets which is $36$.

For the second case, this clearly has $6$ ordered $(a,b,c)$.

Then, our probability is $\frac{42}{216}=\boxed{\frac{7}{36}}$.
\end{sol}

\prob{4}{2018 Memorial Day Mock AMC 10}{How many integer values of n between 1 and 100, inclusive, cause the value of the
following expression to be an integer?
$$\frac{\sqrt[2]{n^{3n}}}{\sqrt[3]{n^{2n}}}$$
}

\begin{sol}
$\boxed{21}$
\end{sol}

\prob{4}{PuMAC 2019}{Keith has 10 coins labeled 1 through 10, where the ith coin has weight $2^i$. The coins are all
fair, so the probability of flipping heads on any of the coins is $\frac{1}{2}$. After flipping all of the
coins, Keith takes all of the coins which land heads and measures their total weight, $W$. If the
probability that $137 \leq W \leq 1061$ is $m/n$ for coprime positive integers $m, n$, determine $m + n$}

\begin{sol} 
This bijects to even binary numbers between $0$ and $2046$. There $\frac{1060-138}{2}+1=462$ even numbers in the range $[137,1061]$. Then, $\frac{462}{1024}=\frac{231}{512}\implies \boxed{743}$.
\end{sol}

\prob{4}{PHS HMMT TST 2016}{Compute the number of ordered triples of sets $(A_{1},A_{2},A_{3})$ that satisfy the following:
\begin{enumerate}
\item $A_{1}\cup A_{2}\cup A_{3} = \{1,2,3,4,5,6\}$
\item $A_{1}\cap A_{2}\cap A_{3} = \emptyset$
\end{enumerate}}

\begin{sol}
Consider each element separately and see where it can go in the Venn Diagram. It can go in $8-2=6$ sections as it can't be in all three or not in any. So, $\boxed{6^6}$.
\end{sol}

\prob{4}{Magic Math AMC 10 2020}{The county of Tropolis has five towns, with no roads built between any two of them. How
many ways are there for the mayor of Tropolis to build five roads between five different pairs
of towns such that it is possible to get from any town to any other town using the roads?}

\begin{sol}
We do complementary counting and find the number of ways so there are two components in the graph. Note that it is impossible for the largest component to be a $3$-component (or smaller) because there is at most 3 roads in the $3$-component and at most 1 road between the two other cities. This means there is at most $4$ roads which is a contradiction as there are $5$ roads.

So, the largest component must be a $4$-component (since we cannot have a $5$-component). This means the graph looks like a connected $4$-component and an isolated city. Note that no matter how we place roads in the $4$-component, it is connected (by similar reasoning by before as there cannot be a 3-component). There are $\binom{4}{2}=6$ possible roads. Then, there are $\binom{6}{5}=6$ ways to create the roads. 

Now, the total number of ways to addin roads without the condition the whole graph is connected is $\binom{\binom{5}{2}}{2}=45$. So, the number of ways to have a connected graph after adding in the roads is $45-6=\boxed{39}$.
\end{sol}

\prob{5}{HMMT February 2013}{Thaddeus is given a $2013 \times 2013$ array of integers each between $1$ and $2013$, inclusive. He is
allowed two operations:
1. Choose a row, and subtract $1$ from each entry.
2. Chooses a column, and add $1$ to each entry.
He would like to get an array where all integers are divisible by $2013$. On how many arrays is this
possible?}

\begin{sol}
Bijection between first row, first column and a valid grid.
$\boxed{2013^{4025}}$
\end{sol}
\prob{5} {HMMT Feburary 2014}{We have a calculator with two buttons that displays an integer x. Pressing the first button replaces
$x$ by $\lfloor \frac{x}{2} \rfloor$ , and pressing the second button replaces $x$ by $4x + 1$. Initially, the calculator displays 0.
How many integers less than or equal to 2014 can be achieved through a sequence of arbitrary button
presses? (It is permitted for the number displayed to exceed 2014 during the sequence. Here, byc
denotes the greatest integer less than or equal to the real number y.)}

\begin{sol}
Any number with $1$ digits separated by one or more $0$'s  is valid. Notice that for $2015$ through $2047$, the first two digits are $11$ so they are not valid. Casework on number of digits now.
\end{sol}

\prob{7}{HMMT February 2014}{An up-right path from $(a,b) \in \mathbf{R}^2$ to $(c,d) \in \mathbf{R}^2$
is a finite sequence $(x_1,y_1),\ldots  ,(x_k,y_k)$ of points
in $\mathbf{R}^2$
such that $(a,b) = (x_1,y_1), (c,d) = (x_k,y_k)$, and for each $1 \leq i < k$ we have that either
$(x_{i+1},y_{i+1}) = (x_i + 1,y_i)$ or $(x_{i+1},y_{i+1}) = (x_i
,y_i + 1)$. Two up-right paths are said to intersect if they
share any point.

Find the number of pairs $(A,B)$ where $A$ is an up-right path from $(0, 0)$ to $(4, 4)$, B is an up-right path
from $(2, 0)$ to $(6, 4)$, and A and B do not intersect.}

\begin{sol}
Bijection
$\boxed{1750}$
\end{sol}

\prob{7}{MMATHS 2019}{Alice wishes to walk from the point (0, 0) to the point (6, 4) in increments of (1, 0) and (0, 1), and Bob wishes to walk from the point
(0, 1) to the point (6, 5) in increments of (1, 0) and (0, 1). How many ways are there for Alice and Bob to get to their destinations if their
paths never pass through the same point (even at different times)?}

\begin{sol}
Key observation is that the number of pairs of intersecting paths from $(0,0)$ to $(m,n)$ and $(0,1)$ to $(m,n+1)$ is equal to the number of pairs of paths from $(0,0) to $(m,n+1)$ and $(0,1)$ and $(m,n)$ by a bijection.

$\boxed{5292}$
\end{sol}
%%-----------------------------------------------------------------------------------------------------------------------------------%%
\subsubsection{Stars and Bars}

\prob{1}{AMC 8 2019/25}{Alice has $24$ apples. In how many ways can she share them with Becky and Chris so that each person (include Alice) has at least $2$ apples?}

\begin{sol}
We have $a+b+c=24$ and $a\ge 2$ while $b,c\ge 0$. Let $a'=a-2$. Then, $a'\ge 0$. We have $a'+b+c=22$. By Stars and Bars, there are $\binom{24}{2}=23\cdot 12=\boxed{276}$ ways to distribute.
\end{sol}

\prob{1}{AMC 10A 2003/21}{Pat is to select six cookies from a tray containing only chocolate chip, oatmeal, and peanut butter cookies. There are at least six of each of these three kinds of cookies on the tray. How many different assortments of six cookies can be selected?}

\begin{sol} 
Let $a,b,c$ be the number of choclate chip, oatmetal, and peanut butter cookies repsecitvely. Then, $a+b+c=6$ has $\binom{8}{2}=\boxed{28}$ distributions.
\end{sol}

\prob{1}{AMC 10A 2018/11}{When 7 fair standard 6-sided dice are thrown, the probability that the sum of the numbers on the top faces is 10 can be written as \[\frac{n}{6^7},\]where $n$ is a positive integer. What is $n$?}

\begin{sol} 
Let the $i$th dice roll $x_{i}$. Notice that $x_{i}\ge 1$. Let $x_{i}'=x_{1}-1$. Then, $x_{1}+x_{2}\cdots + x_{7}=10\implies x_{1}+x_{2}\cdots + x_{7}=3$ has $\binom{9}{6}=\boxed{84}$ distributions. Note that it is impossible for any of the $x_{i}$'s to exceed $6$.
\end{sol}

\prob{4}{AMC 12A 2006/25}{How many non- empty subsets $S$ of $\{1,2,3,\ldots ,15\}$ have the following two properties?
\begin{enumerate}
    \item No two consecutive integers belong to $S$.
    
    \item If $S$ contains $k$ elements, then $S$ contains no number less than $k$.
\end{enumerate}}

\begin{sol} 
We can do casework on the number of elements $k$. Then, we can only have numbers in the range $\{k,k+1\cdots 15\}$. Now, let the elements in the subset be $a_{1},a_{2}\cdots a_{k}$. Let $d_{1}=a_{1}-k$, $d_{2}=a_{2}-a_{1}$, $d_{3}=a_{3}-a_{2}$ and so on until $d_{k}=a_{k}-a_{k-1}$ and $d_{k+1}=15-a_{k}$. We have that $d_{1}+d_{2}\cdots + d_{k}+d_{k+1}=15-k$. 

Notice that $d_{2},d_{3}\cdots d_{k}\ge 2$ to satisfy that no two consecutive integers are in $S$. So, let $d_{k}'=d_{k}-2$ for $k=2,3\cdots k$. So, $d_{1}+d_{2}'+d_{3}'\cdots +d_{k}'+d_{k+1}=15-k-2(k-1)=17-3k$. By Stars and Bars, there are $\binom{17-2k}{k}$ ordered $(k+1)$-tuplets $(d_{1},d_{2}\cdots d_{k+1})$. Notice that $(d_{1},d_{2}\cdots d_{k+1})$ determines $a_{1},a_{2}\cdots a_{k}$.
\end{sol}

\prob{4}{PersonPsychopath\rq{}s 2nd 2015-2016 Mock AMC 10}{Define an increasing sequence $a_{1},a_{2}\ldots a_{k}$ of integers to be $n$-true if it satises the following conditions for positive integers $n$ and $k$:
\begin{itemize}
\item The difference between any two consecutive terms is less than $n$.
\item The sequence must start with $0$ and end with $10$.
\end{itemize}
How many $5$-true sequences exist?}

\begin{sol}
$\boxed{401}$
\end{sol}

\prob{4}{HMMT February 2014}{Eli, Joy, Paul, and Sam want to form a company; the company will have 16 shares to split among the
4 people. The following constraints are imposed:
\begin{enumerate}
\item Every person must get a positive integer number of shares, and all 16 shares must be given out.
\item No one person can have more shares than the other three people combined.
\end{enumerate}
Assuming that shares are indistinguishable, but people are distinguishable, in how many ways can the
shares be given out?}

\begin{sol}
$\boxed{315}$
\end{sol}

%%-----------------------------------------------------------------------------------------------------------------------------------%%
\subsubsection{Expected Value}

\prob{3}{Math Prizes For Girls 2018}{Maryam has a fair tetrahedral die, with the four faces of the die labeled
1 through 4. At each step, she rolls the die and records which number
is on the bottom face. She stops when the current number is greater
than or equal to the previous number. (In particular, she takes at least
two steps.) What is the expected number (average number) of steps
that she takes? Express your answer as a fraction in simplest form.}

\prob{4}{HMMT February 2013}{
Values $a_{1},\ldots ,a_{2013}$ are chosen independently and at random from the set $\{1,\ldots , 2013\}$. What is
expected number of distinct values in the set $\{a_{1},... ,a_{2013}\}$?}

\begin{sol}
Let $X_{n}$ be indicator function for if $n$ is in the set and compute sum of all $E(X_{n})$ using Linearity of Expectastion.
$\boxed{\frac{2013^{2013}-2012^{2013}}{2013^{2012}}}$
\end{sol}

\prob{4}{BMT 2020}{Three lights are placed horizontally on a line on the ceiling. All the lights are initially on. Every second, Neil picks one of the three lights uniformly at random to switch: if it is off, he switches it on; if it is on, he switches it off. When a light is switched, any lights directly to the left or
right of that light also get turned on (if they were off) or off (if they were on). The expected
number of lights that are on after Neil has flipped switches three times can be expressed in the
form $\frac{m}{n}$ , where $m$ and $n$ are relatively prime positive integers. Compute $m + n$}

\begin{sol}
$\frac{55}{27}\implies \boxed{82}$
\end{sol}
\prob{5}{SLKK AIME 2020}{
Woulard forms a 8 letter word by picking each letter from the set $\{w, o, u\}$
with equal probability. The score of a word is the nonnegative difference between the
number of distinct occurrences of the three-letter word “uwu” and the number of distinct
occurrences of the three-letter word “owo”. For example, the string “owowouwu” has a
score of 2 - 1 = 1. If the expected score of Woulard’s string can be expressed as $\frac{a}{b}$ for
relatively prime positive integers a and b, find the remainder when $a+b$ is divided by $1000$.}

\begin{sol} 
Let the $value$ of the string be the total number of uwu and owo's. Note that the value only differs from the score when there are both uwu and owo's. We can easily compute the value using Linearity of Expectation and do casework on when it differs.
$\boxed{269}$
\end{sol}

\prob{6}{PuMAC 2019}{. Marko lives on the origin of the Cartesian plane. Every second, Marko moves 1 unit up with
probability 2/9, 1 unit right with probability 2/9, 1 unit up and 1 unit right with probability
4/9, and he doesn’t move with probability 1/9. After 2019 seconds, Marko ends up on the
point (A, B). What is the expected value of A · B?}

\prob{6}{PuMAC 2019}{Kelvin and Quinn are collecting trading cards; there are 6 distinct cards that could appear
in a pack. Each pack contains exactly one card, and each card is equally likely. Kelvin buys
packs until he has at least one copy of every card, then he stops buying packs. If Quinn is
missing exactly one card, the probability that Kelvin has at least two copies of the card Quinn
is missing is expressible as $m/n$ for coprime positive integers $m, n$. Determine $m + n$.}

\begin{sol} 
Complementary counting, find probability that Kelvin has exactly one copy of the card Quinn is missing. 
\end{sol}

\prob{7}{OMO Spring 2019}{The sum
$$\sum_{i=0}^{1000} \frac{\binom{1000}{i}}{\binom{2019}{i}}$$
can be expressed in the form $\frac{p}{q}$, where $p$ and $q$ are relatively prime positive integers. Compute $p + q$}

\begin{sol}
$\boxed{152}$
\end{sol}

\prob{7}{2017 AMC 12B/25}{A set of $n$ people participate in an online video basketball tournament. Each person may be a member of any number of $5$-player teams, but no two teams may have exactly the same $5$ members. The site statistics show a curious fact: The average, over all subsets of size $9$ of the set of $n$ participants, of the number of complete teams whose members are among those $9$ people is equal to the reciprocal of the average, over all subsets of size $8$ of the set of $n$ participants, of the number of complete teams whose members are among those $8$ people. How many values $n$, $9\leq n\leq 2017$, can be the number of participants?}

\begin{sol}
Let $t$ be the number of teams. We get that 
$$\frac{t\binom{n-5}{4}}{\binom{n}{9}} = \frac{1}{\frac{t\binom{n-5}{3}}{\binom{n}{8}}}$$
by counting how many times each individual team gets counted. For the LHS, we can choose $4$ other people out of the $n-5$ non team members. The RHS follows similarily.

Then, after some bashing, we get 
$$t=\frac{\frac{n!}{(n-5)!}}{2^5\cdot 3^2\cdot 5\cdot 7}$$

We find that $\frac{n!}{(n-5)!}$ is a multiple of $9,5$ always. Also, $\frac{n}{(n-5)!}$ is a multiple of $7$ for $5$ resdiues and (by some inelegant bashing) is a multiple of $32$ for $16$ residues. So, in each period $[224n, 224n+223]$, we have $80$ successful $n$. So, in $[2,2017]$ there are $80\cdot 7=560$ successful $n$ (we subtract out $1$ which works and add in $2017$ which works). We also have to subtract out $[2,8]$ out of which $3$ work. We get $\boxed{557}$.
\end{sol}

%%-----------------------------------------------------------------------------------------------------------------------------------%%
\subsubsection{Recursion}
\prob{3}{MMATHS 2019}{
Noah has an old-style M\&M machine. Each time he puts a coin into the machine, he is equally likely to get 1 M\&M or 2 M\&M’s. He
continues putting coins into the machine and collecting M\&M’s until he has at least 6 M\&M’s. What is the probability that he actually ends up with 7 M\&M’s?}

\begin{sol}
$\boxed{\frac{21}{64}}$
\end{sol}
\prob{3}{OMO Spring 2019}{Susan is presented with six boxes $B_{1}, \cdots , B_{6}$, each of which is initially empty, and two identical coins of denomination $2^{k}$
for each $k = 0, \cdots , 5$. Compute the number of ways for Susan to place the coins in
the boxes such that each box Bk contains coins of total value $2^k$.}

\begin{sol}
$\boxed{32}$
\end{sol}

\prob{4}{Math Prizes For Girls 2019}{A 1×5 rectangle is split into five unit squares (cells) numbered
1 through 5 from left to right. A frog starts at cell 1. Every second it jumps
from its current cell to one of the adjacent cells. The frog makes exactly 14
jumps. How many paths can the frog take to finish at cell 5?}

\prob{4}{JMC 10 Year 1}{What is the base-10 sum of all positive integers such that, when expressed in binary,
have 7 digits and have no two consecutive 1's?}

\begin{sol}
$\boxed{963}
\end{sol}

\prob{4}{OMO Spring 2019}{When two distinct digits are randomly chosen in $N = 123456789$ and their places are swapped, one gets a new number $N_{0}$ (for example, if 2 and 4 are swapped, then $N_{0} = 143256789$). The expected value of
$N_{0}$ is equal to $\frac{m}{n}$ , where m and n are relatively prime positive integers. Compute the remainder when
$m + n$ is divided by $10^6$.}

\begin{sol}
$\boxed{962963}$
\end{sol}

\prob{6}{HMMT February 2014}{We have a calculator with two buttons that displays an integer $x$. Pressing the first button replaces
$x$ by $\lfloor\frac{x}{2}\rfloor$, and pressing the second button replaces $x$ by $4x + 1$. Initially, the calculator displays $0$.
How many integers less than or equal to $2014$ can be achieved through a sequence of arbitrary button
presses? (It is permitted for the number displayed to exceed $2014$ during the sequence. Here, $\lfloor y \rfloor$
denotes the greatest integer less than or equal to the real number y.)}

\begin{sol}
$\boxed{233}$
\end{sol}

\prob{7}{CNCM Online Round 2}{On a chessboard with 6 rows and 9 columns, the Slow Rook is placed in the bottom-left
corner and the Blind King is placed on the top-left corner. Then, 8 Sleeping Pawns are placed such that
no two Sleeping Pawns are in the same column, no Sleeping Pawn shares a row with the Slow Rook or
the Blind King, and no Sleeping Pawn is in the rightmost column. The Slow Rook can move vertically
or horizontally 1 tile at a time, the Slow Rook cannot move into any tile containing a Sleeping Pawn,
and the Slow Rook takes the shortest path to reach the Blind King. How many ways are there to place
the Sleeping Pawns such that the Slow Rook moves exactly 15 tiles to get to the space containing the
Blind King?}
%%-----------------------------------------------------------------------------------------------------------------------------------%%
\subsubsection{Burnside\rq{}s}
\prob{2}{AlcumusGuy Mock AMC 10 2016-2017}{George wants to construct a bracelet with 2 identical red beads, 2
identical white beads, and 2 identical blue beads, spaced equally around a circle. How many different bracelets can George make if rotations and reflections of a bracelet are not distinct from each other?}

\begin{sol}
$\boxed{11}$
\end{sol}
%%-----------------------------------------------------------------------------------------------------------------------------------%%
\subsection{Sequences}
\prob{4}{ARML 2009}{For $k \ge 3$, we define an ordered k-tuple of real numbers $(x_{1}, x_{2}, . . . , x_{k})$ to be special if, for every i such that $1 \leq i \leq k$, the product $x_{1} \cdot x_{2} \cdot \ldots \cdot x_{k} = x_{i}^2$. Compute the smallest value of $k$ such that there are at least $2009$ distinct special $k$-tuples.}

\begin{sol}
$\boxed{12}$
\end{sol}
%%-----------------------------------------------------------------------------------------------------------------------------------%%
\subsection{Miscellaneous}

\prob{1}{CountofMC's Last-Minute Mock AMC 12}{Let $S = \{P_1, P_2, P_3, \ldots , P_m\}$ be a set of lattice points in $n$-dimensional
space; that is, each point $P_i$ can be expressed as the $n$-tuple $(x_1, x_2, x_3, \ldots , x_n)$,
where each of the $x_j$ ’s are integers. In terms of $n$, what is the minimum number
of points in $S$ required to be able to find two points in $S$ whose midpoint is also
a lattice point?}

\begin{sol}
$\boxed{2^{n}+1}$
\end{sol}

\prob{1}{MMATHS 2019}{An ant starts at the top vertex of a triangular pyramid (tetrahedron). Each day, the ant randomly chooses an adjacent vertex to move
to. What is the probability that it is back at the top vertex after three days?}

\begin{sol}
$\boxed{\frac{2}{9}}$
\end{sol}

\prob{1}{PersonPsychopath\rq{}s 2nd 2015-2016 Mock AMC 10}{Ten different people are attending a party. Three of them have black shirts, four of them have red shirts, two people are wearing blue shirts, and only one person wears a white shirt. Each person then shakes hands with everyone wearing a shirt of different color than himself, and two people with the same shirt color do not shake hands. How many distinct handshakes take place?}

\begin{sol}
$\boxed{35}$
\end{sol}

\prob{1}{Mandelbrot Nationals Sample Test}{Michael Jordan's probability of hitting any basketball shot is three times greater than mine, which never exceeds a third. To beat him in a game, I need to hit a shot myself and have Jordan miss the same shot. If I pick my shot optimally, what is the maximum probability of winning which I can attain?}

\begin{sol} 
The probability is $p(1-3p)=p-3p^2=3(-p^2+\frac{p}{3})=3(-(p-\frac{1}{6})^2+\frac{1}{36})$. So, the maximium probability is $\frac{1}{36}\cdot 3 = \boxed{\frac{1}{12}}$. 
\end{sol}

\prob{2}{CNCM Online Round 2}{Adi the Baller is shooting hoops, and makes a shot with probability $p$. He keeps shooting
hoops until he misses. The value of $p$ that maximizes the chance that he makes between $35$ and $69$
(inclusive) buckets can be expressed as $\frac{1}{\sqrt[a]{b}}$
for a prime a and positive integer $b$. Find $a + b$.}

\begin{sol}
$p=\frac{1}{\sqrt[35]{2}}\implies \boxed{37}$
\end{sol}

\prob{2}{PHS ARML TST 2017}{Consider a group of eleven high school students. To create a middle school math contest, they must
pick a four-person committee to write problems and a four-person committee to proofread. Every
student can be on neither committee, one committee, or both committees, except for one student who
does not want to be on both. How many combinations of committees are possible?}

\begin{sol} 
Complementary counting, find how many committees have that student on both and how many committees without that restriction
\end{sol}

\prob{2}{Mandelbrot Regionals 2009}{Mr. Strump has formed three person groups in his math class for working on projects. Every student is in exactly two groups, and any two groups have at most one person in common. In fact, if two groups are chosen at random then the probability that they have exactly one person in common is one-third. How many students are there in Mr. Strump’s class?}

\prob{3}{ARML 2014}{Bobby, Peter, Greg, Cindy, Jan, and Marcia line up for ice cream. In an acceptable lineup,
Greg is ahead of Peter, Peter is ahead of Bobby, Marcia is ahead of Jan, and Jan is ahead
of Cindy. For example, the lineup with Greg in front, followed by Peter, Marcia, Jan, Cindy,
and Bobby, in that order, is an acceptable lineup. Compute the number of acceptable lineups.}

\begin{sol}
The probability Greg, Peter, and Bobby line up in their specific order is $\frac{1}{6}$. Similarily, the probability that Marica, Jan, and Cindy line up in their specific order is $\frac{1}{6}$. So, the number of acceptable lineups is $720\cdot \frac{1}{36}=\boxed{20}$
\end{sol}

\prob{3}{AMC 12B 2017/17}{A coin is biased in such a way that on each toss the probability of heads is $\frac{2}{3}$ and the probability of tails is $\frac{1}{3}$. The outcomes of the tosses are independent. A player has the choice of playing Game A or Game B. In Game A she tosses the coin three times and wins if all three outcomes are the same. In Game B she tosses the coin four times and wins if both the outcomes of the first and second tosses are the same and the outcomes of the third and fourth tosses are the same. Find the probability of winning game A minus the probability of winning game B.}

\prob{3}{ARML 2009}{Some students in a gym class are wearing blue jerseys, and the rest are wearing red jerseys.
There are exactly 25 ways to pick a team of three players that includes at least one player
wearing each color. Compute the number of students in the class.}

\begin{sol}
$\boxed{7}$
\end{sol}

\prob{3}{MMATHS 2019}{Erik wants to divide the integers 1 through 6 into nonempty sets A and B such that no (nonempty) sum of elements in A is a multiple
of 7 and no (nonempty) sum of elements in B is a multiple of 7. How many ways can he do this? (Interchanging A and B counts as a
different solution.)}

\begin{sol}
$\boxed{6}$
\end{sol}

\prob{4}{HMMT February 2013}{
A cafe has $3$ tables and $5$ individual counter seats. People enter in groups of size between $1$ and
$4$, inclusive, and groups never share a table. A group of more than $1$ will always try to sit at a table,
but will sit in counter seats if no tables are available. Conversely, a group of $1$ will always try to sit at
the counter first. One morning, $M$ groups consisting of a total of $N$ people enter and sit down. Then,
a single person walks in, and realizes that all the tables and counter seats are occupied by some person
or group. What is the minimum possible value of $M + N$?}

\begin{sol}
$\boxed{13}$
\end{sol}

\prob{4}{MMATHS 2019}{ How many ways are there to tile a 4 × 6 grid with L-shaped triominoes? (A triomino consists of three connected $1 \times 1$ squares not all in a line.}
\begin{sol}
$16+2=\boxed{18}$
\end{sol}
\prob{4}{MMATHS 2019}{A subset of \{1, 2, 3, 4, 5, 6, 7, 8\} of size 3 is called special if whenever a and b are in the set, the remainder when $a + b$ is divided by $8$
is not in the set. (a and b can be the same.) How many special subsets exist?} 

\begin{sol}
$\boxed{6}$
\end{sol}

\prob{5}{BMT 2018}{Suppose there are 2017 spies, each with $\frac{1}{2017}$ th of a secret code. They communicate by telephone;
when two of them talk, they share all information they know with each other. What is the
minimum number of telephone calls that are needed for all 2017 people to know all parts of the
code?}

\begin{sol}
Use induction, then claim the minimium is $2n-4$.
$\boxed{4030}$
\end{sol}

\prob{6}{CRMT Team 2019}{A deck of the first 100 positive integers is randomly shuffled. Find the expected number of draws it takes to get a prime number if there is no replacement.}

\prob{6}{CNCM PoTD}{Find the remainder when $\sum_{n=0}^{333}\sum_{k=3n}^{999} \binom{k}{3n}$ is divided by $70$.}
%%-----------------------------------------------------------------------------------------------------------------------------------%%
%%%%%%%%%%%%%%%%%%%%%%%%%%%%%%%%%%%%%%%%%%%%%%%%%%%%%
%%-----------------------------------------------------------------------------------------------------------------------------------%%
\setcounter{problem}{0}
\section{Number Theory}
%%-----------------------------------------------------------------------------------------------------------------------------------%%
\subsection{Divisors}
\prob{1}{ARML 2013}{Compute the smallest positive integer n such that $n^{2}+n^{0}+n^{1}+n^{3}$ is a multiple of $13$.}

\begin{sol}
$\boxed{5}$
\end{sol}
\prob{1}{LMT Fall 2019}{$a$ and $b$ are positive integers and $8^{a} 9^{b}$ has $578$ factors. Find $ab$.}

\begin{sol}
$\boxed{88}$
\end{sol}

\prob{1}{MA$\theta$ 2018}{How many distinct prime numbers are in the first 50 rows of Pascal’s Triangle?}

\begin{sol} 
If $k\neq 1,n-1$ for $\binom{n}{k}$, then $\binom{n}{k}$ is composite by its explicit formula. So, it is clear only $\binom{p}{1}=p$ works. $\boxed{15}$
\end{sol}

\prob{2}{AHSME 1984}{ How many triples $(a,b,c)$ of positive integers satisfy the simultaneous equations:
$$ab+bc=44$$
$$ac+bc=23$$}

\prob{2}{PHS ARML TST 2017}{Compute the greatest prime factor of 
$$3^8+2\cdot 3^4\cdot 4^4+2^{16}$$}

\begin{sol} 
Let $3^4=x$ and $4^4=y$. Then, this is just $x^2+2xy+y^2=(x+y)^2$. So, it is $(81+256)^2=(337)^2$. Note that $337$ is prime as it is not divisible by $2,3,5,7,11,13,17$. So, the greatest prime factor is $\boxed{337}$.
\end{sol}

\prob{2}{HMMT November 2014}{Compute the greatest common divisor of $4^8 - 1$ and $8^{12} - 1$}

\prob{2}{DMC Mock AMC 10}{Let the sum of $n \ge 2$ consecutive integers be a positive prime number, where the
smallest of the integers is $a$. If $a+n = 28$, what is the sum of all possible values of $a$?}

\begin{sol}
$\boxed{1}$
\end{sol}

\prob{2}{BMT 2018}{. Suppose for some positive integers, that $\frac{p+\frac{1}{q}}{q+\frac{1}{p}}=17$. What is the greatest integer $n$ such that $\frac{p+q}{n}$
is always an integer?}

\begin{sol}
$\frac{p+\frac{1}{q}}{q+\frac{1}{p}}=\frac{p}{q}\implies p=17q$. Then $\frac{p+q}{n}=\frac{18p}{n}$. So, $n=\boxed{18}$.
\end{sol}

\prob{2}{LMT Spring 2020}{Let LMT represent a 3-digit positive integer where L and M are nonzero digits. Suppose that the 2-digit number MT divides LMT. Compute the difference between the maximum and minimum possible values of LMT .}

\begin{sol}
$100L + MT = 0\pmod{MT}\implies 100L =0\pmod{MT}$. Max $LMT$ occurs when $L=9$ and $MT=45$. Min $LMT$ occurs when $L=1$ and $MT=25$. So, $945-125=\boxed{820}$.
\end{sol}

\prob{2}{BMT 2018}{How many multiples of 20 are also divisors of 17!?}

\prob{2}{Scrabbler94 Mock AMC 10 2020}{Let $n$ be the smallest positive integer with the property that $\lcm(n, 2020!) =
2021!$, where $\lcm(a, b)$ denotes the least common multiple of $a$ and $b$. How
many positive factors does $n$ have?}

\begin{sol}
By Legendre\rq{}s, $2021!$ has $43^{48}\cdot 47^{43}$.
So, $n=43^{48}\cdot 47^{43}\implies 49\cdot 44 =\boxed{2156}$
\end{sol}


\prob{2}{NanoMath Fall Meet 2020}{ If $\phi(n)$ is the number of integers $1\leq p \leq n$ such that $p$ is relatively prime to $n$, then find the number of even values $n$ between $1$ and $100$ inclusive for which $\phi(n)=\phi(\frac{n}{2})$}

\begin{sol}
$n=0\pmod{4}\implies \boxed{25}$
\end{sol}
\prob{2}{LMT Spring 2020}{Suppose there are $n$ ordered pairs of positive integers ($a_{i},b_{i}$) such that $a_{i} +b_{i} = 2020$ and $a_{i}, b_{i}$ is a multiple of $2020$,
where $1 \leq i \leq n$. Compute the sum
$$\sum_{i=1}^{n} a_{i} +b_{i}.$$}

\prob{3}{A superdomino is a domino that can have a positive integer amount of up to 12 dots in each of its two squares.
A superdomino is called unique if the number of dots on its top square is relatively prime to the number of
dots on its bottom square. One such unique superdomino is shown below. What is the sum of the numbers (top and bottom) on all such unique superdominoes?}
	
\begin{sol}
$\boxed{1124}$
\end{sol}

\prob{3}{HMMT Feburary 2018}{Distinct prime numbers $p,q,r$ satisfy the equation $$2pqr + 50pq = 7pqr + 55pr = 8pqr + 12qr  =A$$ for some positive integer $A$. What is $A$?}

\begin{sol}
$\boxed{110}$
\end{sol}

\prob{3}{Math Prizes For Girls 2019}{How many positive integers less than 4000 are not divisible by 2, not
divisible by 3, not divisible by 5, and not divisible by 7?}

\begin{sol}
The amount of numbers in $[210n+1, 210n+210]$ that are not divisible by any of $2,3,5,7$ is $\phi(210)=\frac{1}{2}\cdot \frac{2}{3}\cdot \frac{4}{5}\cdot \frac{6}{7}\cdot 210= 48$. Then, the amount of such numbers in $[1,3990]$ is $19\cdot 48= 912$.  In, $[3991, 4000]$, only $3991$ is relatively prime to both $2,3,5,7$. So, our answer is $\boxed{913}$.
\end{sol}

\prob{3}{MA$\theta$ 2018}{The number $1 \cdot 1! + 2 \cdot 2! + 3 \cdot 3! +\cdots + 100 \cdot 100!$ ends with a string of 9s. How many consecutive 9s are at the end of the number?}

\begin{sol}
$n\cdot n! = (n+1)!-n!$, then telescope to get $101!-1!$. Then, we are finding how many $0$'s does $101!$ have at the end which is equivalent to the number of factors of $10$. The number of factors of $10$ is equivalent to the number of factors of $5$. By Legendre's, there are $20+4=\boxed{24}$ 5's.
\end{sol}

\prob{3}{AMC 12B 2017/16}{The number $21!=51,090,942,171,709,440,000$ has over $60,000$ positive integer divisors. One of them is chosen at random. What is the probability that it is odd?}

\prob{3}{MMATHS 2019}{If n is an integer between $4$ and $1000$, what is the largest possible power of $2$ that $n^4 - 13n^2 + 36$ could be divisible by?}

\begin{sol}
$\boxed{2048}$
\end{sol}
\prob{3}{BMT 2015}{There exists a unique pair of positive integers k, n such that k is divisible by 6, and $\sum_{i=1}^{k} i^2 = n^2$.
Find $(k, n)$.}

\begin{sol}
You get $k\cdot (6k+1)\cdot (12k+1)=n^2$. All of these are pairwise coprime so each are squares. Trying out the first couple of squares for $k$, we get $k=4$ is a solution. $k=4$ gives $2^2\cdot 5^2\cdot 7^2=n^2\implies n = 70$. Having both $b^2=6a^2+1$ and $c^2=12a^2+1$ be squares is not possible for larger $a$. So, our only solution is $\boxed{(4,70)}$.
\end{sol}
 
 \prob{4}{LMT Spring 2020}{ Let $\phi(k)$ denote the number of positive integers less than or equal to $k$ that are relatively prime to $k$. For example, $\phi(2) = 1$ and $\phi(10) = 4$. Compute the number of positive integers $n \leq 2020$ such that $\phi(n^2)=2\phi(n)^2$.}

\begin{sol}
Note that $\phi(n^2)=n\cdot \phi(n)$. This is clear as each of the intervals $[kn,kn+n-1]$ has $\phi(n)$ numbers and also from the explicit interval. Then, we have $n\cdot phi(n)=2\phi(n)^2\implies \phi(n)=\frac{n}{2}$. Using the explicit formula, the only prime factor of $n$ is $2$ clearly. Otherwise, consider the largest prime factor of $n$ which is $p>2$. Then, the denominator has factor $p$ as $p$ cannot be canceled out (it would have to be canceled out by a multiple of $p$ in the numerator, implying a larger prime factor of $n$). There are $\boxed{10}$ powers of $2$.
\end{sol}

\prob{4}{Magic Math AMC 10 2020}{Find the number of ordered pairs of positive integers $(a, b)$ with $a < b < 2017$ such that $10a$ is divisible by $b$ and $10b$ is divisible by $a$.}

\begin{sol}
Let $\gcd(a,b=k$. Then, let $x=\frac{a}{k}$ and $y=\frac{b}{k}$. It is clear that $\gcd(x,y)=1$. Now, $b|10a\iff yk|10xk\iff y|10x$. Similarily, $b|10a\iff x|10y$. Also, note that $n|m$ if and only if $\nu_{p}(n)\leq \nu_{p}(m)$ for all primes $p$. Now, consider $p=2$. We get $\nu_{2}(x)\leq \nu_{2}(y)+1$ and $\nu_{2}(y)\leq \nu_{2}(x)+1$. Combining, we get $\nu_{2}(x)-1\leq \nu_{2}(y)\leq \nu_{2}(x)+1$. Also, note that as $\gcd(x,y)=1$, both $\nu_{2}(x)$ and $\nu_{2}(y)$ cannot be positive. So, at least one is zero. WLOG $\nu_{2}(x)=0$, then $\nu_{2}(y)=0,1$. This gives the solutions $(\nu_{2}(x),\nu_{2}(y))=(0,0),(0,1),(1,0)$. Similarily, we get the solutions $(\nu_{5}(x),\nu_{5}(y))=(0,0),(0,1),(1,0)$. Note that $\nu_{5}$ and $\nu_{2}$ are essentially independent. Then, we get $(x,y)$ can possibly be $(1,1),(1,2),(1,5),(1,10),(2,1),(2,5),(5,1),(5,2),(10,1)$. But, also note $a<b$ so $x<y$ so only $(1,2),(1,5),(1,10),(2,5)$ are valid. We have $b<2017\implies yk<2017\implies y < \frac{2017}{k}$. This gives $1008+403+201+403=\boxed{2015}$ solutions.
\end{sol}

\prob{4}{CMC 10A 2021}{For a certain positive integer $n$, there are exactly $2021$ ordered pairs of positive divisors
$(d_1, d_2)$ of $n$ for which $d_1$ and $d_2$ are relatively prime. What is the sum of all possible
values of the number of divisors of n?}

\begin{sol}
$\boxed{1539}$
\end{sol}

\prob{4}{HMMT Feburary 2016}{For which integers $n \in \{1, 2, \ldots , 15\}$ is $n^n + 1$ a prime number?}
\begin{sol}
Clearly, $n$ must be even otherwise $n^{n}+1$ is even. Furthermore, $n$ cannot have any odd factors. Otherwise, expressing $n=mk$ for odd $k$ gives $n^{n}+1=n^{mk}+1=(m^{m})^{k}+1^{k}=(m^{m}+1)(m^{mk-m}\ldots)$ which means it cannot be prime. Then, $n$ must be $2^{x}$ for some $x$. Then, $n=1,2,4,8$ are only possible choices now. Now, note $1^{1}+1, 2^{2}+1, 4^{4}+1$ are clearly prime, Now, $8^{8}+1=2^{24}+1=(2^{8})^3+1^3$ so it is divisible by $2^{8}+1$ and clearly not prime.

Our answers are $\boxed{1,2,4}$.
\end{sol}
\prob{5}{CNCM Online Round 1}{Consider all possible pairs of positive integers $(a, b)$ such that $a\ge b$ and both $\frac{a^2+b}{a-1}$ and $\frac{b^2+a}{b-1}$
are integers. Find the sum of all possible values of the product $ab$.}

\prob{6}{HMMT Feburary 2017}{Find all pairs $(a, b)$ of positive integers such that $a^{2017} + b$ is a multiple of $ab$.}

\prob{7}{HMMT Feburary 2017}{. Kelvin the Frog was bored in math class one day, so he wrote all ordered triples $(a, b, c)$ of positive integers such that $abc = 2310$ on a sheet of paper. Find the sum of all the integers he wrote down. In
other words, compute
$$\sum_{\substack{abc=2310 \\ a,b,c \in \mathbb{N}}} (a + b + c),$$
where $\mathbb{N}$ denotes the set of positive integers.}
%%-----------------------------------------------------------------------------------------------------------------------------------%%
\subsection{Modulo}

\prob{1}{MA$\theta$ 2018}{The number $4^{14}-1$ is divisible by $29$ but $2^{14}-1$ is not. What is the remainder when $2^{14}-1$ is divided by $29$?}

\begin{sol}
$4^{14}-1=(2^{14}-1)(2^{14}+1)=0\pmod{29}$. Since $2^{14}-1\neq 0 \pmod{29}$, $2^{14}+1=0\pmod{29}\implies 2^{14}-1=\boxed{27}\pmod{29}$
\end{sol}

\prob{1}{AMC 12A 2003/18}{Let $n$ be a $5$-digit number, and let $q$ and $r$ be the quotient and the remainder, respectively, when $n$ is divided by $100$. For how many values of $n$ is $q+r$ divisible by $11$?}

\begin{sol}
We have $n=100q+r\implies n\pmod{11}=q+r$. So, $q+r=0\pmod{11}\iff n=0\pmod{11}$. The smallest 5-digit multiple of $11$ is $10010$ and the largest is $99990$. So, there are $\frac{99990-10010}{11}+1=9090-910+1=\boxed{8181}$.
\end{sol}

\prob{1}{BMT 2018}{Find the minimal $N$ such that any $N$-element subset of \{1, 2, 3, 4, . . . 7\} has a subset $S$ such that
the sum of elements of $S$ is divisible by $7$.}

\begin{sol}
First, note that $N\leq 4$. Any $4$-element subset has all of at least one of the following: $\{1,6\},\{2,5\},\{3,4\},\{7\}$ by the Pigeonhole Principle. So, any $4$-element subset has a subset that has a sum divisible by $7$. Also, note $N>3$ because of the counterexample: $6,5,4$. So, $N=\boxed{4}$. 
\end{sol}

\prob{2}{HMMT February 2013}{Find the rightmost non-zero digit of the expansion of $(20)(13!)$}

\begin{sol}
$\boxed{6}$
\end{sol}
\prob{2}{AMC 10B 2019/14}{The base-ten representation for $19!$ is \[121,6T5,100,40M,832,H00,\] where $T$, $M$, and $H$ denote digits that are not given. What is $T+M+H$?}

\begin{sol}
Also, $19!=0\pmod{125}$ by Legendre's so $H=0$. Note that $19!=0\pmod{9}$ so $T+M+H+33=0\pmod{9}\implies T+M+H=3\pmod{9}\implies T+M=3\pmod{9}\implies T+M=3,12$.  We also have $19!=0\pmod{11}\implies H-2+3-8+M-0+4-0+0-1+5-T+6-1+2-1=0\pmod{11}\implies H+M-T+7=0\pmod{11}\implies H+M-T=4\pmod{11}\implies M-T=4\pmod{11}\implies M-T=4$ since $0\leq M,T\leq 9$. Notice that $T+M=3$ is impossible then. So, $T+M=12\implies H+T+M=\boxed{12}$
\end{sol}

\prob{2}{BMT 2018}{What is the remainder when $201820182018\ldots$ [2018 times] is divided by $15$?}

\begin{sol}
Let $N=201820182018\ldots$ [2018 times]. We will find $N\pmod{3}$ and $N\pmod{5}$ and then use CRT. Clearly, $N\pmod{5}=3$. Also, $N\pmod{3}=2018\cdot 2018=(-1)^2=1$. Then, $N\pmod{15}=\boxed{13}$.
\end{sol}


\prob{2}{NEMO 2017}{Let $a, b, c$ be distinct integers from the set \{3, 5, 7, 8\}. What is the greatest possible value of the units digit of $a^{(b^c)}$?}

\begin{sol}
We try $9$. This either comes from $a=3,7$. For, $a=3$, $b^c=2\pmod{4}$ which is impossible. For $a=7$, $b^c=2\pmod{4}$.

We try $8$. This comes from $a=8$. Then $b^c=1\pmod{4}$ which comes from $b=5$.
$\boxed{8}$
\end{sol}
\prob{2}{BMT 2020}{Compute the remainder when $98!$ is divided by $101$}

\begin{sol}
Wilson's. $\boxed{50}$
\end{sol}

\prob{2}{CNCM PoTD}{Find the number of positive integer $x$ less than $100$ such that 
$$3^x+5^x+7^x+11^x+13^x+17^x+19^x$$ is prime.}

\begin{sol}
Considering $\pmod{3}$, we get $3(1)^x+3(-1)^x=0\pmod{3}$ which is impossible as the expression is clearly $>3$. So, $\boxed{0}$.
\end{sol}

\prob{3}{HMMT February 2013}{ For how many integers $1 \leq k \leq 2013$ does the decimal representation of $k^k$ end with a $1$?}

\begin{sol}
$\boxed{202}$
\end{sol}

\prob{3}{AMC 10B 2018/16}{Let $a_1,a_2,\dots,a_{2018}$ be a strictly increasing sequence of positive integers such that\[a_1+a_2+\cdots+a_{2018}=2018^{2018}.\]What is the remainder when $a_1^3+a_2^3+\cdots+a_{2018}^3$ is divided by $6$?}

\begin{sol}
Notice that $n^3=n\pmod{2}$ and that $n^3=n\pmod{3}$ by Euler's Totient Theorem. So, $n^3=n\pmod{6}$ by CRT. Then, $a_1^3+a_2^3\cdots + a_{2018}^3=a_{1}+a_{2}\cdots +a_{2018}\pmod{6}=2018^{2018}\pmod{6}$. Now, $2018^{2018}\pmod{2}=0$ and $2018^{2018}\pmod{3}=(-1)^{2018}\pmod{3}=1$. So, by CRT, $2018^{2018}=\boxed{4}\pmod{6}$.
\end{sol}
 
\prob{3}{AMC 10A 2020/18}{Let $(a,b,c,d)$ be an ordered quadruple of not necessarily distinct integers, each one of them in the set ${0,1,2,3}.$ For how many such quadruples is it true that $a\cdot d-b\cdot c$ is odd? (For example, $(0,3,1,1)$ is one such quadruple, because $0\cdot 1-3\cdot 1 = -3$ is odd.)}

\begin{sol}
We have two cases: $ad$ is even and $bc$ is odd or $ad$ is odd and $bc$ is even. If $ad$ is even, then we can do complementary counting and get $4^{2}-2^{2}=12$ ways for $(a,d)$. This is because if $ad$ was odd, this is equivalent to both $a,d$ being odd. If $bc$ is odd, we get $2^{2}=4$ ways for $(b,c)$.  The other case follows similarily. Altogether, are $2\cdot 12\cdot 4=\boxed{96}$ such $(a,b,c,d)$.
\end{sol}

\prob{3}{CNCM Online Round 2}{An ordered pair $(n, p)$ is juicy if $n^2 \equiv 1 \pmod{p^2}$ and $n \equiv -1 \pmod{p}$ for positive integer $n$ and odd prime $p$. How many juicy pairs exist such that $n, p \leq 200$?}

\prob{3}{MMATHS 2018}{ For any prime number $p$, let $S_p$ be the sum of all the positive divisors of $37^{p}p^{37}$ (including $1$ and $37^p p^37$). Find the sum of all primes
$p$ such that $S_p$ is divisible by $p$}

\begin{sol}
$\boxed{19}$
\end{sol}

\prob{3}{HMMT Feburary 2018}{ There are two prime numbers $p$ so that $5p$ can be expressed in the form $\lfloor \frac{n^2}{5} \rfloor$ for some positive integer $n$. What is the sum of these two prime numbers?
}

\begin{sol}
The two prime numbers are $5$ and $29$, consider $\pmod{25}$. Our answer is $\boxed{34}$
\end{sol}

\prob{3}{BMT 2019}{Compute the remainder when the product of all positive integers less than and relatively prime
to 2019 is divided by 2019.}

\prob{4}{AMC 12B 2017/21}{Last year Isabella took $7$ math tests and received $7$ different scores, each an integer between $91$ and $100$, inclusive. After each test she noticed that the average of her test scores was an integer. Her score on the seventh test was $95$. What was her score on the sixth test?}

\prob{4}{BMT 2018}{If $r_{i}$ are integers such that $0 \leq r_{i} < 31$ and $r_{i}$ satisfies the polynomial $x^4 + x^3 + x^2 + x\equiv 30 \pmod{31}$, find 
$$\sum_{i=1}^{4}(r_{i}^2 + 1)^{-1} \pmod{31}.$$ 
where $x^{-1}$ is the modulo inverse of $x$, that is, it is the unique integer y such that $0 < y < 31$
and $xy - 1$ is divisible by $31$.}

\begin{sol}
Note that $x^4+x^3+x^2+x\equiv 30 \pmod{31}\iff x^4+x^3+x^2+x+1 \equiv 0\pmod{31}\iff \frac{x^5-1}{x-1}\equiv 0\pmod{31}$. So, $r_{i}$ are the roots of $x^{5}-1\equiv 0\pmod{31}$ except for $1$. Note that $31=2^{5}-1\implies 2^{5}\equiv 1\pmod{31}$. So, $2$ is a root. Then, $2^{2}=4,2^{3}=8,2^{4}=16$ are also roots. 

Now, we want to compute
$$\frac{1}{5}+\frac{1}{17}+\frac{1}{65}+\frac{1}{257}$$
$$=\frac{1}{5}+\frac{1}{17}+\frac{1}{3}+\frac{1}{9}$$
$$=\frac{1}{5}+\frac{1}{9}+\frac{1}{17}+\frac{1}{3}$$
$$=\frac{14}{45}+\frac{20}{51}$$
$$=\frac{14}{14}+\frac{20}{20}$$
$$=1+1$$
$$=\boxed{2}$$

\end{sol}

\prob{4}{NEMO 2017}{Find all positive integers $1 \leq k \leq 289$ such that $k^2-32$ is divisible by $289$.}

\begin{sol}
$k=\pm 7 \pmod{17}$.
Let $k=17a\pm 7$. Then, $k^2-32=17(\pm 14a+1)=0\pmod{289}$ so $14a+1=0\pmod{17}\implies -3a=33\pmod{17}\implies a=6\pmod{17}$ and $-14a+1=0\pmod{17}\implies 3a=33\implies a =11$. So, $k=\boxed{109, 180}$.
\end{sol}

\prob{4}{SLLKK AIME 2020}{Smush is a huge Kobe Bryant fan. Smush randomly draws n jerseys
from his infinite collection of Kobe jerseys, each being either the recent \#24 jersey or the
throwback \#8 jersey with equal probability. Let $p(n)$ be the probability that Smush can
divide the $n$ jerseys into two piles such that the sum of all jersey numbers in each pile is
the same. If}

\prob{5}{BMT 2018}{Ankit wants to create a pseudo-random number generator using modular arithmetic. To do so
he starts with a seed $x_{0}$ and a function $f(x) = 2x + 25 \pmod{31}$. To compute the kth pseudo
random number, he calls $g(k)$ defined as follows:

$$g(k) = \begin{cases}
x_{0} & \text{if } k = 0 \\
f(g(k - 1)) & \text{if } k > 0 
\end{cases}$$

If $x_{0}$ is $2017$, compute $\sum_{
j=0}^{2017} g(j) \pmod{31}$.
}

\begin{sol}
$\boxed{21}$
\end{sol}

\prob{5}{LMT Spring 2020}{Compute the maximum integer value of $k$ such that $2^k$ divides $3^{2n+3} +40n -27$ for any positive integer $n$.}

\begin{sol}
Factorize it into $8(27(3^{2n-2}+3^{2n-4}\cdots + 1)+5n)$ and analyze the inner term $\pmod{8}$ and $\pmod{16}$. Then, get $\boxed{6}$.
\end{sol}

\prob{5}{PMC Mock AMC 10 2020}{
For all positive intgers $n$, let $f(n)$ denote the remainder when ${{(n+1)^{(n+2)}}^{\cdots}}^{2n}$ is divided by $n^2$. Compute $f(101)$.
}

\begin{sol}
Repeated Euler\rq{}s Totient, $\boxed{506}$
\end{sol}

\prob{6}{PHS HMMT TST 2020}{Find the largest integer $0 < n < 100$ such that $n^2 + 2n$ divides $4(n-1)! + n + 4$.}

\begin{sol}
For $n$ is even, $n^2+2n=(n)(n+2)=4(\frac{n}{2})(\frac{n+2}{2})$. $4(n-1)!=0\pmod{4(\frac{n}{2})(\frac{n+2}}{2})$, then we get $n+4=0\pmod{n^2+2n}$ which is impossible. For $n$ is odd, $n^2+2n=(n)(n+2)$ and $\gcd(n,n+2)=1$. If either of $n, n+2$ are composite, WLOG $n=0\pmod{p}$ for some prime $p<n$, then $(n-1)!=0\pmod{p}\implies n+4=0\pmod{p}\implies 4 =0\pmod{p}$ contradiction. Similar for $n+2$. Then, we have that $n,n+2$ must be both primes. We show that this works. We have $4(n-1)!+n+4\pmod{n}=-4+4=0\pmod{n}$ by Wilsons'. Also, $4(n-1)!+n+4\pmod{n+2}=2+4\frac{-1}{(n+1)(n)}\pmod{n+2}=2+4\frac{-1}{2}\pmod{n+2}=2-2=0\pmod{n+2}$. Since $\gcd(n,n+2)=1$, we're done.

The largest twin primes in the range are $\boxed{71},73$.
\end{sol}

\prob{7}{NEMO 2017}{Compute the remainder when $\binom{3^4}{3^2}$ is divided by $3^8$.}

\begin{sol}
Consider $f(x)=(x-1)(x-2)\cdots (x-8)-8!$.
$\boxed{2925}$
\end{sol}

\prob{7}{HMMT November 2014}{Suppose that $m$ and $n$ are integers with $1 \leq m \leq 49$ and $n \ge 0$ such that $m$ divides $n^{n+1} + 1$. What is the number of possible values of $m$?}

\begin{sol}
$\boxed{29}$
\end{sol}

\prob{7}{BMT 2018}{How many $1 < n \leq 2018$ such that the set $\{0, 1, 1+2, \ldots, 1+2+3+\cdots+i, \ldots , 1+2+\cdots+n-1\}$
is a permutation of $\{0, 1, 2, 3, 4, \cdots , n - 1\}$ when reduced modulo $n$?}

\begin{sol}
Only $2^{k}$ works so $\boxed{10}$.
\end{sol}

\prob{8}{BMT 2018}{Determine the number of ordered triples $(a, b, c)$, with $0 \leq a, b, c \leq 10$ for which there exists
$(x, y)$ such that $ax^2 + by^2 \equiv c\pmod{11}$
}

\prob{8}{BMT 2018}{ Compute the following:
$$\sum_{i=0}^{99} (x^2+1)^{-1} \pmod{199}$$
where $x^{-1}$ is the value $0 \leq y \leq 199$ such that $xy - 1$ is divisible by 199}

\prob{10}{SLKK AIME 2020}{: Let p = 991 be a prime. Let S be the set of all lattice points $(x, y)$,
with $1 \leq x, y \leq p - 1$. On each point $(x, y)$ in S, Olivia writes the number $x^2 + y^2$. Let
$f(x, y)$ denote the product of the numbers written on all points in S that share at least
one coordinate with (x, y). Find the remainder when
$$\sum_{i=1}^{p-2} \sum_{j=1}^{p-2} f(i,j)$$
is divided by p.}

\begin{sol}
$\boxed{24}$
\end{sol}
%%-----------------------------------------------------------------------------------------------------------------------------------%%
\subsection{Bases}
\prob{4}{HMMT November 2014}{Mark and William are playing a game with a stored value. On his turn, a player may either multiply
the stored value by 2 and add 1 or he may multiply the stored value by 4 and add 3. The first player
to make the stored value exceed 2100 wins. The stored value starts at 1 and Mark goes first. Assuming
both players play optimally, what is the maximum number of times that William can make a move?
(By optimal play, we mean that on any turn the player selects the move which leads to the best possible
outcome given that the opponent is also playing optimally. If both moves lead to the same outcome,
the player selects one of them arbitrarily.)
}

\prob{3}{HMMT February 2013}{Let $S$ be the set of integers of the form $2^x + 2^y + 2^z$, where $x, y, z$ are pairwise distinct non-negative
integers. Determine the $100$th smallest element of $S$.}

\begin{sol}
$\boxed{577}$
\end{sol}


\prob{3}{CNCM Online Round 1}{Akshar is reading a 500 page book, with odd numbered pages on the left, and even
numbered pages on the right. Multiple times in the book, the sum of the digits of the two opened pages
are 18. Find the sum of the page numbers of the last time this occurs.}

\begin{sol}
We have three cases: $\overline{abc}, \overline{ab(c+1)}$ and $\overline{ab9}, \overline{a(b+1)0}$ and $\overline{a99}, \overline{(a+1)00}$. For the first case, the sum of digits is $2a+2b+2c+1=1\pmod{2}$ so it is impossible. For the second case, $2a+2b+10=18\implies a+b=4$. This gives $b=4$ and $a=0$ as the maximum solution for this case. For the third case, the sum of digits is $2a+19>18$. So, the maximum solutions are $409,410$ giving a sum of $\boxed{819}$.
\end{sol}

\prob{3}{CNCM Online Round 1}{Define $S(N)$ to be the sum of the digits of $N$ when it is written in base 10, and take
$S^{k}(N) = S(S(. . .(N). . .))$ with $k$ applications of S. The stability of a number $N$ is defined to be the
smallest positive integer $K$ where $S^{K}(N) = S^{K+1}(N) = S^{K+2}(N) = \ldots$. Let $T_{3}$ be the set of all
natural numbers with stability $3$. Compute the sum of the two least entries of $T_{3}$.}

\begin{sol} 
$S^{K}(N)=S^{K+1}(N)\implies S^{K}(N)=S(S^{K}(N))$. Note that for $n\ge 10$, $S(n) <10$. So, $S^{N}(N)$ is a one digit number. So, if a number has stability $3$, this implies that the first time $S^{k}(N)$ is a one-digit number is when $k=3$. Then, $S^{2}(N)\ge 10$ and $S(N)\ge 19$. So, the two smallest are $199,289$. Summing, we get $\boxed{488}$.
\end{sol}

\prob{4}{HMMT November 2013}{ How many of the first 1000 positive integers can be written as the sum of finitely many distinct
numbers from the sequence $3^{0},3^{1},3^{2}\cdots$?}

\prob{6}{HMMT November 2014}{For any positive integers $a$ and $b$, define $a \oplus b$ to be the result when adding a to b in binary (base 2), neglecting any carry-overs. For example, $20 \oplus 14 = 101002 \oplus 11102 = 110102 = 26$. (The operation $\oplus$ is called the exclusive or.) Compute the sum
$$\sum_{k=0}^{2^{2014}-1} (k\oplus \lfloor \frac{k}{2} \rfloor)$$}

\prob{6}{Scrabbler94 Mock AMC 10 2020}{How many ordered 11-tuples $(a_0, a_1, a_2, \ldots , a_{10})$ of integers satisfy the equation
$a_0 + 2a_1 + 2^{2}a_2 + \ldots + 2^{10}a_{10} = 2020$
where $0 \leq a_i \leq 2$ for all $0 \leq i \leq 10$?}

\begin{sol}
$\boxed{51}$
\end{sol}
%%-----------------------------------------------------------------------------------------------------------------------------------%%
\subsection{Diophantine Equations}

\prob{4}{2018 Memorial Day Mock AMC 10}{$q$ and $r$ are positive integers in the following equation. 
$$\frac{q}{r} + \frac{r}{q} = \frac{qr}{144}$$
There is only one possible value of $q+r$. Let $S(n)$ represent the sum of the digits of positive integer $n$. What is $S(q+r)$?}

\begin{sol}
$\boxed{8}$
\end{sol}
\subsection{Binomial Theorem}
\prob{2}{BMT 2015}{Compute the sum of the digits of $1001^{10}$.}

\begin{sol}
$(1000+1)^{10}$, no overlap.
$2(1+1+4+5+1+2+2+1)+2+5+2=\boxed{43}$
\end{sol}

\prob{3}{LMT Fall 2019}{Determine the remainder when $13^{2020} +11^{2020}$ is divided by $144$.}

\begin{sol}
$13^{2020} + 11^{2020} = (12+1)^{2020}+(12-1)^{2020} = 12\cdot \binom{2020}{1}+1-12\cdot \binom{2020}+1\pmod{144}=\boxed{2}$
\end{sol}

\prob{3}{NEMO 2017}{Find the last two digits of $17^{(20^{17})}$
in base 9. (Here, the numbers given are in base 10.)}

\begin{sol}
$(18-1)^{20^{17}}=-18\cdot 20^{17}+1$. Then, $20^{17}\pmod{9} = 2^{17}=\frac{1}{2} = 5$. Then, $-18\cdot 5+1=-89=74\pmod{81}$.
So, we have $82_{9}$.

$\boxed{82}$
\end{sol}

\prob{7}{AIME I 2020/12}{Let $n$ be the least positive integer for which $149^n-2^n$ is divisible by $3^3\cdot5^5\cdot7^7.$ Find the number of positive integer divisors of $n.$}

\begin{sol}
$n=3^2\cdot 4\cdot 5^4\cdot 7^6=2^2\cdot 3^2\cdot 5^4\cdot 7^5$. So, $3\cdot 3\cdot 5\cdot 6=\boxed{270}$
\end{sol}
%%-----------------------------------------------------------------------------------------------------------------------------------%%
\subsection{Series and Sequences}
\prob{1}{ARML 2009}{Let $p$ be a prime number. If $p$ years ago, the ages of three children formed a geometric
sequence with a sum of $p$ and a common ratio of $2$, compute the sum of the children’s current
ages.}

\begin{sol}
$\boxed{30}$
\end{sol}

\prob{3}{ARML 2014}{For each positive integer $k$, let $S_{k}$ denote the infinite arithmetic sequence of integers with first term $k$ and common difference $k^2$. For example, $S_{3}$ is the sequence $3, 12, 21,\ldots$ Compute
the sum of all $k$ such that $306$ is an element of $S_{k}$}

\prob{4}{ARML 2014}{The arithmetic sequences $a_{1}, a_{2}, a_{3}, . . . , a_{20}$ and $b_{1}, b_{2}, b_{3}, . . . , b_{20}$ consist of $40$ distinct positive integers, and $a_{20} + b_{14} = 1000$. Compute the least possible value for $b_{20} + a_{14}$.}

\begin{sol}
$\boxed{10}$
\end{sol}
%%-----------------------------------------------------------------------------------------------------------------------------------%%
\subsection{Miscellenous}
\prob{1}{LMT Spring 2020}{Compute the smallest nonnegative integer that can be written as the sum of 2020 distinct integers.}

\begin{sol}
Note that $0=(-1010+1010)+(-1009+1009)\cdots + (-1+1)$. So, our answer is $\boxed{0}$.
\end{sol}

\prob{2}{PersonPsychopath\rq{}s 2nd 2015-2016 Mock AMC 10}{A positive integer has the property that the product of its digits is $10!$. What is the minimum
possible sum of its digits?}

\begin{sol}
Use as many $2,3,5,7$ as possible. $\boxed{45}$
\end{sol}

\prob{2}{Reun\rq{}s Thanksgiving 2017 Mock AMC 10}{Denote the operation $a\circ b$ for positive integers as $ab^{a}$ For some constant x,  the pair $(m, n)$ satises $m \circ n = n^{m+2}$ and $m + n = x$. Similarly, the pair
$(p, q)$ satises $p \circ q = q^{p+2}$ and $p + q = 3x$. If $m, n, p$ and $q$ are distinct,
what is the smallest possible value of $mq + np$?}

\begin{sol}
$\boxed{630}$
\end{sol}

\prob{2}{OMO Spring 2019}{Daniel chooses some distinct subsets of $\{1, . . . , 2019\}$ such that any two distinct subsets chosen are disjoint. Compute the maximum possible number of subsets he can choose}

\begin{sol}
Each element can only be used once and you can also choose the empty subset. So, $2019+1=\boxed{2020}$
\end{sol}

\prob{2}{BMT 2018}{How many integers can be expressed in the form:
$\pm 1 \pm 2 \pm 3\cdots \pm 2018$?}

\begin{sol}
All odd integers in between $-1-2\cdots - 2018$ and $1+2\cdots + 2018$ work. So, $\boxed{2018\cdot 1009+1}$
\end{sol}
\prob{3}{BMT 2015}{Find all integer solutions to
$$x^2+2y^2+3z^2=36$$
$$3x^2+2y^2+z^2=84$$
$$xy+xz+yz=-7$$}

\begin{sol} 
Adding the first two and dividng by $4$ gives $x^2+y^2+z^2=30$. Then, $(x+y+z)^2=x^2+y^2+z^2+2(xy+xz+yz)=16\implies x+y+z=\pm 4$. Also, $(x+y)^2+(x+z)^2+(y+z)^2=2(x^2+y^2+z^2)+2(xy+xz+yz)=46$. 

Let $x+y=a, x+z=b, y+z=c$. We get $a+b+c=\pm 8$ and $a^2+b^2+c^2=46$. The only decomposition of $46$ into squares is $36,9,1$. So, $(6,3,-1)$ and $(-6,-3,1)$ are our only solutions for $(a,b,c)$ (including permutations since it is symmetric). Notice that $(a,b,c)$ being a permutation of $(6,3,-1)$ means $(x,y,z)$ is a permutation of $(5,1,-2)$. Similarily, $(a,b,c)$ being a permutation of $(-6,-3,1)$ means $(x,y,z)$ is a permutation of $(-5,-1,2)$.

We look at which permutations of $(5,1,-2)$ work. We can biject any valid permutation of $(5,1,-2)$ to get a corresponding solution $(-x,-y,-z)$ for permutations of $(-5,-1,2)$. It's clear only $x$ can be $5$ otherwise $2y^2,3z^2>36$. Then, $2y^2+z^2=9\implies y=2,z=-1$. So the only valid solution is $(5,2,-1)$ and the corresponding solution is $(-5,-2,1)$.

We get $\boxed{(5,2,-1),(-5,-2,1)}$.
\end{sol}

\prob{3}{Magic Math AMC 10 2020}{Let $s(n)$ denote the sum of the digits of a positive integer n. Find the number of three-digit positive integers $k \leq 500$ satisfying $s(k) = s(1000 - k)$.}

\begin{sol}
We do casework on $k=\overline{abc}$, $k=\overline{ab0}$ or $k=\overline{a00}$ for non-zero $a,b,c$. We find that $k=\overline{a00}$ works for $a=5$, $k=\overline{ab0}$ never works as it implies $2(a+b)=19$ and $k=\overline{abc}$ works when $a+b+c=14$. Then, $a=1,2\cdots 4$ gives $6+7+8+9= 30$. Our answer is $\boxed{31}$.
\end{sol}

\prob{3}{OMO Spring 2019}{Jay is given $99$ stacks of blocks, such that the ith stack has $i^2$ blocks. Jay must choose a positive integer $N$ such that from each stack, he may take either $0$ blocks or exactly $N$ blocks. Compute the
value Jay should choose for $N$ in order to maximize the number of blocks he may take from the 99
stacks.}

\begin{sol}
The number of blocks taken is $N(100-\lceil \sqrt{N} \rceil)$. Note that the maximium occurs when $N=k^2$ for some integer $k$. Otherwise, $N+1$ would give a larger expression. So, we want to maximize $k^2(100-k)=(k)(k)(100-k)=4(\frac{k}{2})(\frac{k}{2})(100-k)$. By AM-GM, this is maximized when $\frac{k}{2}=100-k\implies k=\frac{200}{3}$. But, $k$ is an integer, so we have to take either $66$ or $67$. We can check $k=67$ gives a larger expression than $k=66$.
$\boxed{67^2}$
\end{sol}

\prob{4}{Reun\rq{}s Thanksgiving 2017 Mock AMC 10}{Denote by $S(n)$ the sum of the digits of $n$: Find the sum of all positive
integers $n < 100$ such that $S(n^2) = (S(n))^2$}

\begin{sol}
Casework, note that $S(n) \leq 6$, $\boxed{176}$
\end{sol}

\prob{4}{BMT 2019}{For a positive integer $n$, define $\phi(n)$ as the number of positive integers less than or equal to $n$
that are relatively prime to $n$. Find the sum of all positive integers $n$ such that $\phi(n)=20$}

\prob{4}{DMC Mock AMC 10}{Joy picks an integer $n$ from the interval $[1, 40]$. She tells Amy the remainder when
$n$ is divided by $7$ and Sid the number of divisors of $n$. Amy and Sid both know $n$ is
in the interval $[1, 40]$, but they get confused and believe Amy was told the number of
divisors and Sid was told the remainder. Amy says, “I know what $n$ is.” Sid replies,
“If so, then I also know what $n$ is.” As it turns out, they thought of the same value
but were wrong due to their confusion. If Amy and Sid tell the truth based on their
beliefs and can reason perfectly, what is the sum of all possible actual values of $n$?
}

\begin{sol}
$\boxed{24}$
\end{sol}

\prob{5}{CNCM Online Round 2}{Let $S$ be the set of all ordered pairs $(x, y)$ of integer solutions to the equation
$$6x^2 + y^2 + 6x = 3xy + 6y + x^2y.$$
$S$ contains a unique ordered pair $(a, b)$ with a maximal value of $b$. Compute $a + b$.}

\prob{5}{PHS HMMT TST 2020}{Find the unique triplet of integers $(a, b, c)$ with $a > b > c$ such that $a+b+c = 65$ and $a^2+b^2+c^2=3083$.}

\prob{5}{BMT 2019}{0. Let S(n) be the sum of the squares of the positive integers less than and coprime to n. For
example, $S(5) = 1^2 + 2^2 + 3^2 + 4^2$, but $S(4) = 1^2 + 3^2$.
Let $p = 2^7 - 1 = 127$ and $q = 2^5 - 1 = 31$ be primes. The quantity $S(pq)$ can be written in the
form
$$\frac{p^2q^2}{6}(a-\frac{b}{c})$$
where $a, b$, and $c$ are positive integers, with $b$ and $c$ coprime and $b < c$. Find $a$.
}


\prob{6}{CNCM PoTD}{How many positive integers $k$ are there such that $101\leq k\leq 10000$ and $\lfloor \sqrt{k-100}\rfloor$ is a divisor of $k$?}

\prob{6}{Compute the number of ordered triples of positive integers $(a, b, c)$ such that $a + b + c + ab + bc + ac = abc + 1$}

\begin{sol}
$\boxed{15}$
\end{sol}

%%-----------------------------------------------------------------------------------------------------------------------------------%%
%%%%%%%%%%%%%%%%%%%%%%%%%%%%%%%%%%%%%%%%%%%%%%%%%%%%%
%%-----------------------------------------------------------------------------------------------------------------------------------%%
\setcounter{problem}{0}
\section{Algebra}

%%-----------------------------------------------------------------------------------------------------------------------------------%%
\subsection{Polynomials}
Generally uses the following techniques: Vieta's, Binomial Theorem, Multinomial Theorem, Remainder Theorem, Newton's Sums, Reciprocal Roots Trick, Quadratic Formula (including using Determinant), Finite Differences, $x+\frac{k}{x}$ subsitution, Polynomial Interpolation

\prob{1}{Scrabber94 Mock AMC 10 2020}{Let $f(x) = |x+4|$ and $g(x) = x^2$ for all real numbers $x$. How many real
numbers $x$ satisfy $f(g(x)) = g(f(x))$?}

\begin{sol}
$\boxed{1}$
\end{sol}

\prob{1}{AMC 12B 2019/8}{Let $f(x) = x^{2}(1-x)^{2}$. What is the value of the sum

$$f \left(\frac{1}{2019} \right)-f  \left(\frac{2}{2019} \right)+f \left(\frac{3}{2019} \right)-f \left(\frac{4}{2019} \right)+\cdots + f \left(\frac{2017}{2019} \right) - f \left(\frac{2018}{2019} \right)?$$}

\begin{sol}
$\boxed{0}$
\end{sol}

\prob{1}{CRMT Math Bowl 2019}{Find the sum of all real numbers such that 
$$\sqrt[4]{16x^4-32x^3+24x^2-8x+1}=5$$
}

\begin{sol}
Note that this is $\sqrt[4]{(2x-1)^4}=|2x-1|$ using the Binomial Theorem. Then, $|2x-1|=5\implies x=3,-2$. The sum is $\boxed{1}$.
\end{sol}

\prob{2}{TAMU 2019}{In the expansion of $(1+ax-x^2)^8$ where a is a positive constant, the coefficient of $x^2$ is $244$. Find the value of $a$}

\begin{sol}
By the Multinomial Theorem, the coefficient of $x^2$ is $\binom{8}{2}\cdot a^2-\binom{8}{1}=28a^2-8$. So, $28a^2-8=244\implies a^2=9\implies a = \boxed{3}$.
\end{sol}

\prob{2}{BMT 2020}{Let $a$ and $b$ be the roots of the polynomial $x^2 + 2020x+c$. Given that $\frac{a}{b}+\frac{b}{a}= 98$, compute $\sqrt{c}$.}

\begin{sol}
$\boxed{202}$
\end{sol}
\prob{3}{HMMT November 2014}{Let $f(x) = x^2 + 6x + 7$. Determine the smallest possible value of $f(f(f(f(x))))$ over all real numbers $x$.}

\prob{3}{HMMT February 2014}{Find the sum of all real numbers x such that $5x^4+10x^3 + 10x^2+5x-11 = 0$}

\begin{sol} 
Note that $f(x)=5x^4+10x^3+10x^2+5x-11$ is symmetric about $-\frac{1}{2}$. This is clearer after you rewrite as $5(x^4+2x^3+2x^2+x)+11=5x(x+1)(x^2+x+1)$ and you can see $f(-1-x)=f(x)$. But after a rough sketch and since it is monotonically increasing after $0$, there are only two real roots that sum to $\boxed{-1}$.
\end{sol}

\prob{3}{BmMT 2014}{Consider the graph of $f(x) = x^3 + x + 2014$. A line intersects this cubic at three points, two
of which have x-coordinates 20 and 14. Find the x-coordinate of the third intersection point}

\begin{sol}
Let the line have the equation $y=mx+b$. Then, $mx+b=x^3+x+2014\implies x^2+(1-m)x+2014-b=0$. Then, the sum of the roots are $0$. So, the third intersection point is $\boxed{-34}$.
\end{sol}

\prob{3}{ARML 2017}{ Compute the number of ordered pairs of integers $(a, b)$ such that the polynomials $x^2 - ax + 24$ and $x^2 - bx + 36$ have one root in common}

\begin{sol}
Let $r$ be the common root. Then, $r^2-ar+24=0\implies a = r + \frac{24}{r}$ and $r^2-br+36=0\implies b = r + \frac{36}{r}$. Since $a,b$ are integers, this implies $r|\gcd(24,36)=12$ ($r$ can be both positive and negative divisors). Then, there are $2\cdot 2\cdot 3 =12$ values for $r$ which then determine $(a,b)$. Also, clearly they don't have two roots in common as both are monic and the constant term is different. Our answer is $\boxed{12}$.
\end{sol}

\prob{3}{Reun\rq{}s Thanksgiving 2017 Mock AMC 10}{For integers a and b, the three roots $X_{1},X_{2}$ and $X_{3}$ of the cubic function
$$f(x) = ax^3 - bx^2 + 286x - 120$$
exist such that $X_{i} = \frac{m+i-1}{m+i-2}$ 
for some positive integer constant $m$
What is the sum of the digits of $b$?}

\begin{sol}
$m=4$, $\boxed{11}$
\end{sol}

\prob{3}{AMC 10A 2015/23}{The zeroes of the function $f(x)=x^2-ax+2a$ are integers. What is the sum of the possible values of $a$?}

\begin{sol}
If the roots are integers, then the discriminant must be a perfect square; otherwise, the roots are irrational. Then, $a^2-8a=n^2$ for some positive integer $n$. This gives $(a-4)^2-n^2=16\implies (a-n-4)(a+n-4)=16$. Note that the sum of the roots is $2a-8$. We have factor pairs $(-16,-1),(-8,-2),(-4,-4),(1,16),(2,8),(4,4)$. Note that if the roots are integers, $a$ must also be a integer because $a$ is the sum of the roots by Vieta's. From this, we get $a=-1, 0, 9, 8$. Trying, we get that all of these work. So, the sum is $\boxed{16}$.
\end{sol}

\prob{4}{HMMT February 2013}{Determine all real values of A for which there exist distinct complex numbers x1, x2 such that the
following three equations hold:
$$x_{1}(x_{1}+1)=A$$
$$x_{2}(x_{2}+1)=A$$
$$x_{1}^{4}+3x_{1}^3+5x_{1}=x_{2}^{4}+3x_{2}^3+5x_{2}$$
}

\begin{sol}
$\boxed{-7}$
\end{sol}


\prob{4}{Payback Mock AMC 10}{Find the number of pairs of integers $(a, c)$ such that the function $f(x)=x^2-ax+ac$
has two integer roots(not necessarily distinct) and $1 \leq c \leq 5$.}

\begin{sol}
$\boxed{20}$
\end{sol}

\prob{4}{2018 Memorial Day Mock AMC 10}{The quadratic equations $y = ax^2 + 20x+ 32$ and $y = x^2 +bx+ 16$ share the same vertex in the xy-coordinate plane. If $a$ and $b$ are integers, what is $a + b$?}

\begin{sol}
$\boxed{9}$
\end{sol}
\prob{4}{HMMT Feburary 2017}{Let $Q(x) = a_{0} + a_{1}x + · · · + a_{n}x^n$ be a polynomial with integer coefficients, and $0 \leq ai < 3$ for all $0 \leq i \leq n$.
Given that $Q(\sqrt{3}) = 20 + 17\sqrt{3}$, compute $Q(2)$}

\prob{4}{TAMU 2018}{Suppose $f$ is a cubic polynomial with roots $a,b,c$ such that 
$$a = \frac{1}{3-bc}$$
$$b = \frac{1}{5-ac}$$
$$c = \frac{1}{7-ab}$$
If $f(0)=1$, find $f(abc+1)$.
}

\begin{sol} 
Let the leading coefficient be $k$. We have $f(0)=1\implies abc=\frac{-1}{k}$. Multiply out the expressions to get $3a-abc=1\implies 3a+\frac{1}{k}=1\implies a = \frac{1-\frac{1}{k}}{3}$. Similarily, $b=\frac{1-\frac{1}{k}}{5}, c= \frac{1-\frac{1}{k}}{7}$.  Also, $f(abc)+1=k(abc-a)(abc-b)(abc-c)$.
\end{sol}

\prob{4}{PuMAC 2019}{Let $f(x) = x^2 + 4x + 2$. Let $r$ be the difference between the largest and smallest real solutions
of the equation $f(f(f(f(x)))) = 0$. Then $r = a^{\frac{p}{q}}$ for some positive integers a, p, q so a is
square-free and $p, q$ are relatively prime positive integers. Compute $a + p + q$}

\begin{sol}
Some pattern finding gives $f^{n}(x)=0$ has solutions $x=-2\pm 2^{\frac{1}{2^{n}}}$. Then, $r=2\cdot 2^\frac{1}{2^{n}}$
\end{sol}

\prob{4}{HMMT February 2014}{Find all real numbers $k$ such that $r^4+kr^3+r^2+4kr+16=0$ is true for exactly one real number $r$.}

\begin{sol}
Divide by $r^2$ and subsitute $t=r+\frac{4}{r}$.
\end{sol}

\prob{4}{HMMT February 2015 Team Reworded}
The complex numbers $x, y, z$ satisfy
$$xyz = -4$$
$$(x + 1)(y + 1)(z + 1) = 7$$
$$(x + 2)(y + 2)(z + 2) = -3$$

Find the value of $(x + 3)(y + 3)(z + 3)$.}

\begin{sol}
$\boxed{-28}$
\end{sol}
\prob{4}{Scrabbler94 Mock AMC 10 2020}{Let $f^1(x) = x^2 - 20$ for all real numbers $x$, and let $f^k(x) = f^1(f^{k-1}(x))$ for all integers $k \ge 2$. Let $x_0$ and $x_1$ be the smallest and largest real solutions
to the equation $f^{2020}(x) = 0$, respectively. What is the largest integer less
than or equal to $x^{2}_{0}+ x^{2}_{1}$?}

\begin{sol}
$\boxed{49}$
\end{sol}
\prob{4}{PHS HMMT TST 2020}{Let $a,b,c$ be the distinct real roots of $x^3+2x+5$. Find $(8-a^3)(8-b^3)(8-c^3)$.}

\prob{4}{PuMAC 2019}{Let Q be a quadratic polynomial. If the sum of the roots of $Q^{100}(x)$ (where $Q^{i}(x)$ is defined
by $Q^{1}(x) = Q(x), Q^{i}(x) = Q(Q^{i-1}(x))$ for integers $i \ge 2$) is $8$ and the sum of the roots of $Q$
is $S$, compute $|\log_{2}(S)|$.}

\prob{7}{AIME I 2014/14}{Let $m$ be the largest real solution to the equation

$$ \dfrac{3}{x-3} + \dfrac{5}{x-5} + \dfrac{17}{x-17} + \dfrac{19}{x-19} = x^2 - 11x - 4$$

There are positive integers $a, b,$ and $c$ such that $m = a + \sqrt{b + \sqrt{c}}$. Find $a+b+c$.}

\prob{8}{AoPS Forums}{Given $x_1,x_2,x_3,x_4$ are the roots of $P(x)=2x^4-5x+1$, find the value of $\sum_{i=1}^4 \frac{1}{(1-x_i)^3}$.}

\begin{sol}
$\ln P(x) = \sum \ln (x-x_{i})$
Taking the derivative of both sides,
$\frac{P\rq{}(x)}{P(x)} = \sum \frac{1}{x-x_{i}}$.
We take the derivative more times to get
$\sum_{i=1}^4 \frac{1}{(1-x_i)^3}=\frac{P'''(1)[P(1)]^2-3P''(1)P'(1)P(1)+2[P'(1)]^3}{2[P(1)]^3}$

\href{https://artofproblemsolving.com/community/c728438h2121954}
$\frac{-339}{8}$
\end{sol}

%%-----------------------------------------------------------------------------------------------------------------------------------%%
\subsubsection{Newton's Sums}
\prob{3}{AoPS Forums}{Let $a_1, a_2, a_3$ be the roots to the polynomial $x^3-2x^2-3x+16$; find $$a_1^3(a_2+a_3)+a_2^3(a_1+a_3)+a_3^3(a_1+a_2)$$}

\begin{sol}
$a_{1}+a_{2}=2-a_{3}$
Then, you get $2\sum a_{1}^3 - \sum a_{1}^4 = -3\sum a_{1}^2+16\sum a_{1}=-3(4+6)+16\cdot 2=-30+32=\boxed{2}$
\end{sol}
\prob{3}{BMT 2015}{Let $r, s$, and $t$ be the three roots of the equation $8x^3 + 1001x + 2008 = 0$. Find
$(r + s)^3 + (s + t)^3 + (t + r)^3$}

\begin{sol}
Equal to $-(r^3+s^3+t^3)=-P_{3}$. We have $8P_{3}+1001P_{1}+2008\cdot 3 = 8P_{3}+2008=0\implies P_{3}=-251$. So, the answer is $\boxed{251}$.
\end{sol}
\prob{3}{NanoMath Fall Meet 2020}{If $x + y = 6$ and $x^3 + y^3 = 108$, find $x^5 + y^5$.}

\begin{sol}
$\boxed{2376}$
\end{sol}

\prob{4}{BMT 2019}{
Let $r_1, r_2, r_3$ be the (possibly complex) roots of the polynomial $x^3 + ax^2 + bx + \frac{4}{3}$. How many
pairs of integers a, b exist such that $r_{1}^3 + r_{2}^3 + r_{3}^3 = 0$ ?}

\prob{6}{SLKK AIME 2020}{Let $a, b,$ and $c$ be the three distinct solutions to $x^{3} - 4x^{2} + 5x + 1 = 0$.
Find $$(a^3+b^3)(a^3+c^3)(b^3+c^3).$$
}

\begin{sol}
$\boxed{189}$
\end{sol}
%%-----------------------------------------------------------------------------------------------------------------------------------%%
\subsubsection{Roots of Unity}
\prob{2}{AMC 12B 2017/12}{What is the sum of the roots of $z^{12}=64$ that have a positive real part?} 

\prob{4}{HMMT February 2013}{Let z be a non-real complex number with $z^{23} = 1$. Compute
$$\sum_{k=0}^{22} \frac{1}{1+z^{k}+z^{2k}}.$$}

\begin{sol}
$\boxed{\frac{46}{3}}$
\end{sol}

\prob{5}{BMT 2019}{ Let $a_n$ be the product of the complex roots of $x^{2n} = 1$ that are in the first quadrant of the
complex plane. That is, roots of the form $a + bi$ where $a, b > 0$. Let $r = a_1 \cdot a_2 \cdot ...\cdot a_{10}$. Find
the smallest integer $k$ such that r is a root of $x^k = 1$}

\begin{sol}


\end{sol}

\prob{6}{BMT 2015}{Evaluate $\sum_{k=0}^{37} (-1)^{k} \binom{75}{2k}$.}

\begin{sol}
Roots of Unity Filter
\end{sol}
%%-----------------------------------------------------------------------------------------------------------------------------------%%
\subsubsection{Polynomial Interpolation}

\prob{3}{Mandelbrot}{There is a unique polynomial $P(x)$ of the form 
$$P(x)=7x^7+c_{1}x^6+c_{2}x^{5}+\cdots + c_{6}x+c_{7}$$
such that $P(1)=1,P(2)=2, \ldots,$ and $P(7)=7$. Find $P(0)$.}

\begin{sol}
$\boxed{-35280}$
\end{sol}

\prob{3}{BMT 2020}{The graph of the degree $2021$ polynomial $P(x)$, which has real coefficients and leading coefficient
1, meets the x-axis at the points $(1, 0),(2, 0),(3, 0), \ldots,(2020, 0)$ and nowhere else. The mean
of all possible values of $P(2021)$ can be written in the form $\frac{a!}{b}$, where $a$ and $b $are positive integers and $a$ is as small as possible. Compute $a + b$.}

\begin{sol}
$\boxed{2023}$
\end{sol}

\prob{6}{AMC 12B 2017/23}{The graph of $y=f(x)$, where $f(x)$ is a polynomial of degree $3$, contains points $A(2,4)$, $B(3,9)$, and $C(4,16)$. Lines $AB$, $AC$, and $BC$ intersect the graph again at points $D$, $E$, and $F$, respectively, and the sum of the $x$-coordinates of $D$, $E$, and $F$ is $24$. What is $f(0)$?}

\begin{sol}
$\boxed{\frac{24}{5}}$
\end{sol}

\prob{7}{HMMT Feburary 2017}{A polynomial P of degree $2015$ satisfies the equation $P(n) = \frac{1}{n^2}$ for $n = 1, 2, \ldots , 2016$. Find
$\lfloor 2017P(2017) \rfloor$}

\prob{8}{HMMT Feburary 2020}{Let $P(x)$ be the unique polynomial of degree at most $2020$ satisfying $P(k^2) = k$ for $k = 0, 1, 2, \ldots , 2020$.
Compute $P(2021^2)$.}

\begin{sol}
$\boxed{2021-\binom{4040}{2020}}$
\end{sol}
%%-----------------------------------------------------------------------------------------------------------------------------------%%
\subsubsection{Generating Functions}

\prob{6}{HMMT February 2013}{Let $N$ be a positive integer whose decimal representation contains $11235$ as a contiguous substring, and let $k$ be a positive integer such that $10^k > N$. Find the minimum possible value of
$$\frac{10^k - 1}{\gcd(N, 10^k - 1)}$$}

\begin{sol}
$\boxed{89}$
\end{sol}

\prob{7}{BMT 2018}{Find the value of
$$\frac{1}{\sqrt{2}}+\frac{2^2}{\sqrt{4}}+\frac{3^2}{\sqrt{8}}\cdots$$}

\begin{sol}
Let $f(x)=1^2x+2^2x^2+3^2x^3\cdots$.
Then, we want to find $f(\frac{1}{\sqrt{2}})$.

Now, let the coefficient of $x^{i}$ be $a_{i}$. So, $a_{i}=a_{i-1}+2i-1\implies a_{i-1}=a_{i-2}+2i-3$. Then, $a_{i}-a_{i-1} = a_{i-1}-a_{i-2}+2$. Then, $a_{i-1}-a_{i-2} = a_{i-2}-a_{i-3}+2$ and subtracting, $a_{i}-2a_{i-1}+a_{i-2}=a_{i-1}-2a_{i-2}+a_{i-3}\implies a_{i}=3a_{i-1}-3a_{i-2}+a_{i-3}$.

Then, $f(x)=(3x-3x^2+x^3)f(x)+x^2+x\implies -x^2-x=(x-1)^3f(x)\implies f(x)=\frac{-(x)(x+1)}{(x-1)^3}$
Then, we plug in $x=\frac{1}{\sqrt{2}}$

$\boxed{24+17\sqrt{2}}$
\end{sol}

\prob{7}{BMT 2018}{Let $F_{1} = 0, F_{2} = 1$ and $F_{n} = F_{n-1} + F_{n-2}$. Compute
$$\sum_{n=1}^{\infty} \frac{\sum_{i=1}^{n} F_{i}}{3^{n}}$$}

\begin{sol}
This is equivalent to $\sum_{i=1}^{\infty} \frac{F_{i}}{3^{i}} \sum_{j=0} \frac{1}{3^{j}}$ = \frac{3}{2} \sum_{i=0}^{\infty} \frac{F_{i}}{3^{i}}$ .
Let $f(x)=F_{0}+F_{1}x+F_{2}x^2\cdots$. We want to find $f(\frac{1}{3})$.
Then, $f(x)= (x+x^2)f(x) +x \implies f(x)(1-x-x^2)=x^2\implies f(x)=\frac{x}{1-x-x^2}$. Then, $f(\frac{1}{3}) = \frac{3}{5}$. Then, $\boxed{\frac{9}{10}}$.
\end{sol}
%%-----------------------------------------------------------------------------------------------------------------------------------%%
\subsection{Complex Numbers}

\prob{4}{HMMT Feburary 2016}{Let $z$ be a complex number such that $|z| = 1$ and $|z - 1.45| = 1.05$. Compute the real part of $z$.}

\prob{4}{2019 AMC 12B}{ How many nonzero complex numbers $z$ have the property that $0, z$, and $z^3$, when
represented by points in the complex plane, are the three distinct vertices of an equilateral
triangle?}

\prob{7}{hukilau17 FAIME 2020}{If $x, y$ are real numbers such that
$$x^3=3xy^2+18$$
$$y^3=3x^2y+26$$
Find $x^2+y^2$.
}

\begin{sol}
Multiply second equation by $i$.
$\boxed{10}$
\end{sol}

\prob{8}{2015 AIME I/13}{With all angles measured in degrees, the product 
$$\prod_{k=1}^{45}  \csc^2(2k - 1)^{\circ}=m^n,$$
where $m$ and $n$ are integers greater than $1$. Find $m + n$.}

%%-----------------------------------------------------------------------------------------------------------------------------------%%

\subsection{Manipulation}

\prob{1}{LMT Fall 2019}{If the numerator of a certain simplified fraction is added to the numerator and the denominator of the fraction, the result is $\frac{20}{19}$. What is the fraction?}

\prob{1}{AMC 10A 2020/7}{The $25$ integers from $-10$ to $14,$ inclusive, can be arranged to form a $5$-by-$5$ square in which the sum of the numbers in each row, the sum of the numbers in each column, and the sum of the numbers along each of the main diagonals are all the same. What is the value of this common sum?}

\begin{sol}
Let the common sum be $s$. Note that if we sum each of the columns, we sum each integer in $-10$ to $14$ exactly once. So, $5s=-10-9-8\cdots +13+14=11+12+13+14=50\implies s=\boxed{10}$.
\end{sol}

\prob{1}{2018 AMC 10A/10}{Suppose that real number $x$ satisfies $$\sqrt{49-x^2}-\sqrt{25-x^2}=3.$$ What is the value of $\sqrt{49-x^2}+\sqrt{25-x^2}$?}

\begin{sol}
Multiplying $(\sqrt{49-x^2}+\sqrt{25-x^2})(\sqrt{49-x^2}-\sqrt{25-x^2}) $ gives $$(49-x^2) - (25-x^2) = 24.$$ We know that $\sqrt{49-x^2}-\sqrt{25-x^2}=3$, so $\sqrt{49-x^2}+\sqrt{25-x^2}$ must be $\frac{24}{3} = \boxed{8}$.
\end{sol}

\prob{2}{2018 BmMT}{Let $x$ be a positive real number so that $x - \frac{1}{x} = 1$.  Compute  $x^8 - \frac{1}{x^8}$.}

\begin{sol} 
We square twice to get $x^2+\frac{1}{x^2}=3$ and $x^4+\frac{1}{x^4}=7$. Then, note that $x^8-\frac{1}{x^8}=(x^4-\frac{1}{x^4})(x^4+\frac{1}{x^4})=7(x^2-\frac{1}{x^2})(x^2+\frac{1}{x^2})=21(x-\frac{1}{x})(x+\frac{1}{x})=21(x+\frac{1}{x})=21(x+\frac{1}{x})$. To find $x+\frac{1}{x}$, we set it to a value $n$. $n^2 = x^2 + \frac{1}{x^2} + 2 \implies n^2 = 5$, and we know $x+\frac{1}{x}$ is positive, so $x+\frac{1}{x} = \sqrt{5}$. So, our final answer is $\boxed{21\sqrt{5}}$.
\end{sol}

\prob{2}{LMT Spring 2020}{Let a,b be real numbers satisfying $a^2 +b^2 = 3ab = 75$ and $a > b$. Compute $a^3-b^3$.}

\begin{sol}
$\boxed{500}$
\end{sol}

\prob{2}{HMMT November 2013}{Evaluate 
$$\frac{1}{2-\frac{1}{2-\frac{1}{2- \cdots\frac{1}{2-\frac{1}{2}}}}}$$
where the digit $2$ appears $2013$ times}

\begin{sol}
We can do some pattern finding when $2$ appears $n$ times. When $n=1$, this is $\frac{1}{2}$. When $n=2$, this is $\frac{2}{3}$. We conjecture that when $2$ appears $n$ times, the expresion is $\frac{n}{n+1}$. We can prove this by induction. With $2$ appearing $n+1$ times, we have $\frac{1}{2-\frac{1}{2-\frac{1}{2\cdots}}} = \frac{1}{2-\frac{n}{n+1}}=\frac{n+1}{n+2}$. Our answer is then $\boxed{\frac{2013}{2014}}$.
\end{sol}

\prob{2}{Mandelbrot}{If $\frac{x^2}{y^2}=\frac{8y}{x}=z$, find the sum of all possible $z$.
}

\begin{sol} 
Let $\frac{x}{y}=k$. Then, $k^2=\frac{8}{k}\implies k^3 = 8\implies k =2$. Then, $k^2=4$ is only possible $z$. The sum is $\boxed{4}$ 
\end{sol}

\prob{2}{AMC 10A 2018/14}{What is the greatest integer less than or equal to\[\frac{3^{100}+2^{100}}{3^{96}+2^{96}}?\]}

\begin{sol}
Conider $\frac{3^{100}+2^{100}}{3^{96}+2^{96}}=3^4-\frac{3^{4}2^{96}-2^{100}}{3^{96}+2^{96}}$. Clearly, $3^{4}2^{96}>2^{100}$ but $3^{4}2^{96}-2^{100}=(81-16)2^{96}=65\cdot 2^{96}$ is much less than $3^{96}$. So, the expression is less than $81$ but greater than $80$. Our answer is $\boxed{80}$.
\end{sol}

\prob{2}{HMMT February 2013}{Let $x$ and $y$ be real numbers with $x > y$ such that $x^2 y^2 + x^2 + y^2 + 2xy = 40$ and $xy + x + y = 8$. Find
the value of $x$.}

\begin{sol}
$3+\sqrt{7}$
\end{sol}

\prob{3}{BMT 2016}{Simplify $\frac{1}{\sqrt[3]{81}+\sqrt[3]{72}+\sqrt[3]{64}}$.}

\begin{sol}
Let $\sqrt[3]{8}=a$ and $\sqrt[3]{9}=b$. Then, it is equivalent to $\frac{1}{a^2+ab+b^2}=\frac{a-b}{a^3-b^3}=\boxed{\sqrt[3]{9}-\sqrt[3]{8}}$.
\end{sol}

\prob{3}{MA$\theta$ 2018}{The solutions to $3\sqrt{2x^2-5x-3}+2x^2-5x=7$ can be written in the form $x=\frac{a \pm \sqrt{b}}{c}$ where $a,b,c$ are positive integers and $x$ is in simplest form. Find $a+b+c$.}

\begin{sol}
Subsitute $\sqrt{2x^2-5x-3}=y$. We get $3y+y^2+3=7\implies y^2+3y-4=\implies (y+4)(y-1)$. Since $y$ is positive, $y=1$. Then, $2x^2-5x-3=1\implies 2x^2-5x-4=0$. By the Quadratic Formula, we get $x=\frac{5 \pm \sqrt{57}}{4}$. We get $5+57+4=\boxed{66}$ as our answer.
\end{sol}

\prob{3}{Mandelbrot Nationals 2008}{
Find the positive real number $x$ for which $5\sqrt{1-x}+5\sqrt{1+x}=7\sqrt{2}$.}

\begin{sol}
Let $\sqrt{1-x}=a$ and $\sqrt{1+x}=b$. We get $5a+5b=7\sqrt{2}\implies a+b=\frac{7\sqrt{2}}{5}\implies a^2+2ab+b^2 = \frac{98}{25}$. We also have $a^2+b^2=2$. So, $ab=\frac{24}{25}\implies \sqrt{1-x^2}=\frac{24}{25}\implies 1-x^2=\frac{576}{625}$. This gives $x=\boxed{\frac{7}{25}}$.
\end{sol}

\prob{4}{NEMO 2019}{Suppose $x$ and $y$ are positive real numbers satisfying
$$\sqrt{xy}=x-y=\frac{1}{x+y}=k$$
Determine $k$.}

\begin{sol}
$$x-y=\frac{1}{x+y}\implies x^2-y^2=1$$
$$x-y=k\implies x^2-2xy+y^2=k^2\implies x^2+y^2=3k^2$$
Combining, we get $x^2=\frac{3k^2+1}{2}$ and $y^2=\frac{3k^2-1}{2}$. Then, $\sqrt{xy}=k\implies x^2y^2=k^4\implies \frac{9k^4-1}{4}=k^4\implies k=5^{-\frac{1}{4}}$.
\end{sol}

\prob{4}{HMMT November 2014 }{ Let $a, b, c, x$ be reals with $(a + b)(b + c)(c + a) \neq 0$ that satisfy
$$\frac{a^2}{a+b}=\frac{a^2}{a+c}+20, \frac{b^2}{b+c}=\frac{b^2}{b+a} + 14 \text{, and } \frac{c^2}{c+a}=\frac{c^2}{c+b}+x$$
Compute $x$.}

\begin{sol}
Notice that $\frac{a^2}{a+b}=\frac{a^2}{a+c}+20\implies \frac{a^2(a+c)-a^2(a+b)}{(a+b)(a+c)}=20\implies \frac{a^2(c-b)}{(a+b)(a+c)}=20$. Similarily, $\frac{b^2(a-c)}{(b+c)(b+a)}=14$ and $\frac{c^2(b-a)}{(c+a)(c+b)}=x$. Summing, we get $\frac{a^{2}(c^2-b^2)+b^{2}(a^2-c^2)+c^2(b^2-a^2)}{(a+b)(b+c)(c+a)}=0=34+x\implies x = \boxed{-34}$.
\end{sol}

\prob{4}{HMMT February 2013}{Let $a$ and $b$ be real numbers such that $\frac{ab}{a^2+b^2}=\frac{1}{4}$. Find all possible values of $\frac{|a^2-b^2|}{a^2+b^2}$}

\begin{sol}
$\boxed{\frac{\sqrt{3}}{2}}$
\end{sol}

\prob{5}{HMMT February 2013}{Let $x,y$ be complex numbers such that $\frac{x^2+y^2}{x+y} = 4$ and $\frac{x^4+y^4}{x^3+y^3} = 2$. Find all possible values of 
$\frac{x^6+y^6}{x^5+y^5}$.}

\begin{sol}
$\boxed{10\pm 2\sqrt{17}}$
Sub $x+y=A$ and $xy=b$
\end{sol}
\prob{5}{Math Prizes For Girls 2015}{
Let $S$ be the sum of all distinct real solutions of the equation 
$$\sqrt{x + 2015} = x^2 - 2015.$$
Compute $\lfloor 1/S \rfloor$.  Recall that if $r$ is a real number, then $\lfloor r \rfloor$ (the floor of $r$) is the greatest integer that is less than or equal to $r$}

\begin{sol}
Let $2015=y$. Then, we have $\sqrt{x+y}=x^2-y\implies x+y=x^4-2x^2y+y^2\implies y^2+(-2x^2-1)y+x^4-x=0$. Then, $y=\frac{2x^2+1 \pm (2x+1)}{2}$. Now, we have $2015=x^2+x+1$ or $2015=x^2-x$. These give $x=\frac{-1 \pm \sqrt{8057}}{2}$ and $x=\frac{1 \pm \sqrt{8061}}{2}$. Now, note that we have $x+y\ge 0 \implies x \ge -2015$ and $x^2-y\ge 0\implies |x|\ge \sqrt{2015}$. \\
\\
We can see that $\frac{-1 + \sqrt{8057}}{2}>\sqrt{2015}$ and that $\frac{-1-\sqrt{8057}}{2} < -\sqrt{2015}$. Also, $\frac{1-\sqrt{8061}}{2} > - \sqrt{2015}$ and $\frac{1+\sqrt{8061}}{2} > \sqrt{2015}$. So, we have that our two solutions are $\frac{-1-\sqrt{8057}}{2}$ and $\frac{1+\sqrt{8061}}{2}$. \\
\\
Then, $\frac{1}{S}=\frac{2}{\sqrt{8061}-\sqrt{8057}}=\frac{\sqrt{8061}+\sqrt{8057}}{2}$. So, $89 < \frac{1}{S} < 90$ so our answer is $\boxed{89}$
\end{sol}

\prob{7}{BMT 2020}{Let $a,b$ and $c$ be real numbers such that $a+b+c=\frac{1}{a}+\frac{1}{b}+\frac{1}{c}$ and $abc=5$. The value of
$$(a-\frac{1}{b})^3+(b-\frac{1}{c})^3+(c-\frac{1}{a})^3$$
can be written in the form $\frac{m}{n}$, where $m$ and $n$ are relatively prime positive integeres. Compute $m+n$.}

\begin{sol}
$\boxed{77}$
\end{sol}
%%-----------------------------------------------------------------------------------------------------------------------------------%%
\subsection{Series and Sequences}
\prob{1}{djmathman Mock AMC 2013/9}{ Let $p$ and $q$ be numbers with  $|p| <1$ and $|q| <1$ such that \[p+pq+pq^2+pq^3 + \dots = 2  ~~\text{and} ~~ q + qp + qp^2 + \dots = 3.\] What is $100pq$?}

\begin{sol}
Using the formula for an infinite geometric series, we get $\frac{p}{1-q}=2\implies p=2-2q$ and $\frac{q}{1-p}=3\implies q=3-3p\implies p=\frac{3-q}{3}$. So, $\frac{3-q}{3}=2-2q\implies 3-q=6-6q\implies 5q=3\implies q =\frac{3}{5}\implies p =\frac{4}{5}$. Then $100pq=100\cdot \frac{12}{25}=\boxed{48}$.
\end{sol}

\prob{1}{PHS HMMT TST 2020}{What is the value of $\frac{\frac{1}{1^2}+\frac{1}{2^2}+\frac{1}{3^2} \cdots}{\frac{1}{1^2}+\frac{1}{3^2}+\frac{1}{5^2}\cdots}$? Remember that $\frac{1}{1^2}+\frac{1}{2^2}\cdots = \frac{\pi^2}{6}$}

\begin{sol}
Notice that every positive integer can be uniquely represented as $2^{k}\cdot o$ for some non-negative $k$ and odd integer $o$. Then, $\frac{1}{1^2}+\frac{1}{2^2}\cdots = (\frac{1}{1^2}+\frac{1}{2^2}+\frac{1}{4^2}\cdots)(\frac{1}{1^2}+\frac{1}{3^2}\cdots) = \frac{4}{3}(\frac{1}{1^2} + \frac{1}{3^2}\cdots)$. So, our answer is $\boxed{\frac{4}{3}}$.
\end{sol}

\prob{1}{OMO Spring 2019}{
Compute 
$$\lfloor \sum_{k=2018}^{\infty} \frac{2019!-2018!}{k!}\rfloor.$$
(The notation $\lfloor x \rfloor$ denotes the least integer $n$ such that $n \ge x$.)}

\begin{sol}
$\boxed{2019}$
\end{sol}

\prob{1}{NEMO 2017}{I am thinking of a geometric sequence with $9600$ terms, $a_1, a_2, \ldots, a_{9600}$. The sum of the terms with indices divisible by three (i.e. $a_3 + a_6 + \cdots + a_{9600}$) is $\frac{1}{56}$ times the sum of the other terms (i.e. $a_1 + a_2 + a_4 + a_5 + · · · + a_{9598} + a_{9599}$). Given that the terms with even indices sum to $10$, what is the
smallest possible sum of the whole sequence?}

\begin{sol}
$$(a_3 + a_6 + \cdots + a_{9600}$) \frac{1}{r^2}+\frac{1}{r} =  a_1 + a_2 + a_4 + a_5 + · · · + a_{9598} + a_{9599}$$
$$\frac{1}{r^2}+\frac{1}{r}=56 \implies \frac{1}{r}=7,-8$.

If even add to $10$, then odd add to $\frac{10}{r}$. So, we want to maximize $r$. So, $\frac{1}{r}=7$ and the sum is $70+10=\boxed{80}$
\end{sol}

\prob{3}{DMC Mock AMC 10}{Dene a sequence recursively by T1 = 1 and
$$T_{n}=\frac{n!\cdot T_{n-1}}{(n-1)!+n\cdot T_{n-1}}$$
for all integers $n \ge 2$. The value of $T_{2020}$ can be written as $\frac{2020!}{m}$ , where $m$ is a positive
integer. What is the sum of the distinct prime divisors of $m$?}

\begin{sol}
$\boxed{198}$
\end{sol}

\prob{3}{HMMT February 2013}{ Let $\{a_{n}\}_{n\ge 1}$ be an arithmetic sequence and $\{g_{n}\}_{n\ge 1}$ be a geometric sequence such that the first four
terms of $\{a_{n} + g_{n}\}$ are $0, 0, 1$, and $0$, in that order. What is the $10$th term of $\{a_{n} + g_{n}\}$?}

\begin{sol}
$\boxed{-54}$
\end{sol}

\prob{3}{NanoMath Fall Meet 2020}{Let $a_0, a_1, a_2, \ldots$ be an arithmetic sequence of positive integers. If $a_0 + a_1 + \cdots + a_{10} = 209$ and $a_{a_{0}} +
a_{a_{1}} + \cdots + a_{a_{10}} = 671$, then find $a_0$.}

\begin{sol}
$\boxed{4}$
\end{sol}

\prob{3}{BMT 2015}{Let $\{a_n\}$ be a sequence of real numbers with $a_1 = -1, a_2 = 2$ and for all $n \ge 3$,
$a_{n+1} - a_{n} - a_{n+2} = 0$.
Find $a_{1} + a_{2} + a_{3} + . . . + a_{2015}$.}

\begin{sol} 
Note that the condition is $a_{n+2}=a_{n+1}-a_{n}$ and that $a_{3}=3,a_{4}=1,a_{5}=-2,a_{6}=-3,a_{7}=-1,a_{8}=2$. So, it repeats with period $6$. The sum $a_{1}+a_{2}\cdots + a_{6}=0$. So, $a_{1}+a_{2}\cdots + a_{2015}=0-a_{2016}=0-a_{6}=\boxed{3}$.
\end{sol}

\prob{3}{HMMT Feburary 2004}{Evaluate the sum
$$\frac{1}{2\lfloor \sqrt{1} \rfloor + 1} +\frac{1}{2\lfloor \sqrt{2} \rfloor + 1}+\frac{1}{2\lfloor \sqrt{3} \rfloor + 1} + \cdots + \frac{1}{2\lfloor \sqrt{100} \rfloor + 1}$$ }

\prob{3}{MMATHS 2019}{ Let $F_1 = F_2 = 1$, and let $F_n = F_{n-1} + F_{n-2}$ for all $n \ge 3$. For each positive integer $n$, let $g(n)$ be the minimum possible value of
$|a_1F_1 + a_2F_2 + · · · + a_nF_n|$,
where each $a_i$
is either $1$ or $-1$. Find $g(1) + g(2) + \cdots  + g(100)$}

\begin{sol}
$g(n)=g(n+3)$
Then, $\boxed{34}$
\end{sol}

\prob{3}{BMT 2020}{Let $\phi$ be the positive solution to the equation
$x^2 = x + 1$. For $n \ge 0$, let an be the unique integer such that $\phi^n  - a_{n}\phi$ is also an integer. Compute
$$\sum_{i=0}^{10} a_{n}$$}

\begin{sol}
$\boxed{143}$
\end{sol}
\prob{3}{Reun\rq{}s Thanksgiving Mock AMC 10}{Let $N=3+66+333+6666+33333+\ldots + \underbrace{666\cdots 666}_{\text{2016 6\rq{}s}} + \underbrace{333\cdots 333}_{\text{2017 3\rq{}s}}$. Find the sum of the digits of $N$}

\begin{sol}
$\boxed{4053}$
\end{sol}

\prob{3}{2018 AMC 12A/18}{Let $A$ be the set of positive integers that have no prime factors other than $2$, $3$, or $5$. The infinite sum $$\frac{1}{1} + \frac{1}{2} + \frac{1}{3} + \frac{1}{4} + \frac{1}{5} + \frac{1}{6} + \frac{1}{8} + \frac{1}{9} + \frac{1}{10} + \frac{1}{12} + \frac{1}{15} + \frac{1}{16} + \frac{1}{18} + \frac{1}{20} + \cdots$$of the reciprocals of the elements of $A$ can be expressed as $\frac{m}{n}$, where $m$ and $n$ are relatively prime positive integers. What is $m+n$?}

\begin{sol}
$\boxed{19}$
\end{sol}

\prob{4}{Math Prizes For Girls 2019}{For each integer from 1 through 2019, Tala calculated the product of
its digits. Compute the sum of all 2019 of Tala’s products.}

\begin{sol}
Note that for $2000$ to $2019$, the corresponding product is $0$. Then, by the Distributive Property, what we want is
$$(1+1+2)(1+1+2+3\cdots +9)(1+1+2+3\cdots + 9)(1+1+2+3\cdots + 9)=389344$$
\end{sol}
\prob{4}{HMMT Feburary 2017}{Find the value of
$$\sum_{1\leq a<b<c} \frac{1}{2^{a}3^{b}5^{c}}$$
(i.e the sum of $\frac{1}{2^{a}3^{b}5^{c}}$ over all triples of positive integers $(a, b, c)$ satisfying $a < b < c$)}

\begin{sol}

\end{sol}
\prob{5}{MMATHS 2018}{Compute
$$\sum_{m=1}^{\infty} \sum_{n=1}^{\infty} \frac{m\cos^2(n)+n\sin^2(m)}{3^{m+n}(m+n)}$$}

\begin{sol}
Let the sum be $x$. Then the sum is the same if you swap $m$ and $n$, then things cancel out.
$\frac{\boxed{1}{8}}$
\end{sol}
\prob{5}{HMMT Feburary 2016}{Let $A$ denote the set of all integers $n$ such that $1 \leq n \leq 10000$, and moreover the sum of the decimal digits of $n$ is $2$. Find the sum of the squares of the elements of $A$.}

\prob{5}{Magic Math AMC 10 2020}{Determine the remainder when
$\sum (-1)^{m+n}mn$
is divided by $1009$, where the sum is taken over all pairs of integers $(m, n) $satisfying $1 \leq m <
n \leq 2020$}

\begin{sol}
Notice that by the Distributive Property, $(\sum_{m=1}^{2020} (-1)^{m} m)(\sum_{n=1}^{2020} (-1)^{n} n) = 2\sum_{1\leq m < n\leq 2020} (-1)^{m+n}mn + \sum_{i=1}^{2020} i^2$.  Note that $(\sum_{m=1}^{2020} (-1)^{m} m)(\sum_{n=1}^{2020} (-1)^{n} n)=(\sum_{m=1}^{2020} (-1)^{m} m)^2=(-1+2-3+4\cdots -2019+2020)=1010^2=1\pmod{1009}$. Also, $\sum_{i=1}^{2020} i^2=\frac{2020\cdot 2021\cdot 4041}{6} =\frac{2\cdot 3\cdot 5}{6} \pmod{1009}=5$. 

Then, letting $x=\sum_{1\leq m < n\leq 2020} (-1)^{m+n}mn$,we have 
$$1=2x+5\pmod{1009}$$
$$-2=x\pmod{1009}$$
$$x=\boxed{1007} \pmod{1009}$$
\end{sol}

\prob{5}{HMMT Feburary 2016}{Determine the remainder when
$$\sum_{i=0}^{2015} \lfloor \frac{2^{i}}{25} \rfloor $$
is divided by $100$, where $\lfloor x \rfloor$ denotes the largest integer not greater than $x$.}
%%-----------------------------------------------------------------------------------------------------------------------------------%%
\subsubsection{Telescoping}

\prob{3}{Purple Comet 2015 HS}{
$$\left(1 + \frac{1}{1+2^1}\right)\left(1+\frac{1}{1+2^2}\right)\left(1 + \frac{1}{1+2^3}\right)\cdots\left(1 + \frac{1}{1+2^{10}}\right)= \frac{m}{n},$$ where $m$ and $n$ are relatively prime positive integers. Find $m + n$.}

\begin{sol}
Telescope to  get $boxed{\frac{2048}{1025}}$
\end{sol}

\prob{5}{BMT 2020}{Given that $\binom{n}{k}=\frac{n!}{k!(n-k)!}$ , the value of

$$\sum_{n=3}^{10} \frac{\binom{n}{2}}{\binom{n}{3}\binom{n+1}{3}}$$

can be written in the form $\frac{m}{n}$ , where $m$ and $n$ are relatively prime positive integers. Compute $m + n$}

\begin{sol}
$\binom{n}{2}=\binom{n+1}{3}-\binom{n}{3}$
So, $\frac{\binom{n}{2}}{\binom{n}{3}\binom{n+1}{3}} = \frac{1}{\binom{n}{3}}-\frac{1}{\binom{n+1}}{3}$
$\frac{164}{165}\implies \boxed{329}$
\end{sol}

\prob{5}{MMATHS 2020}{Let $p(x)$ be the monic cubic polynomial with roots $\sin^2(1^{\circ})$, $\sin^2(3^{\circ})$, and $\sin^2(9^{\circ})$. Suppose that $p\left(\frac{1}{4}\right)=\frac{\sin(a^{\circ})}{n\sin(b^{\circ})}$, where $0 <a,b \le 90$ and $a,b,n$ are positive integers.  What is $a+b+n$?}

\begin{sol}
$f(x)=(x-\sin^2(1))(x-\sin^2(3))(x-\sin^2(9))$
so, $f(\frac{1}{4}) = (\frac{1}{4}-1+\cos^2(1))(\frac{1}{4}-1+\cos^2(3))(\frac{1}{4}-1+\cos^2(9)) = (\frac{4\cos^2(1)-3)}{4})(\frac{4\cos^2(3)-3)}{4})(\frac{4\cos^2(9)-3)}{4}) = \frac{\cos(3)}{4\cos(1)}\frac{\cos(9)}{4\cos(3)}\frac{\cos(27)}{4\cos(9)=\frac{\cos(27)}{64\cos(1))=\frac{\sin(63)}{64\sin(89)}$. Then, $63+64+89=\boxed{216}$
\end{sol}
%%-----------------------------------------------------------------------------------------------------------------------------------%%
\subsection{Trigonometry}

\prob{2}{ARML Local 2020}{A real number $x$ is selected uniformly at random between $0$ and $\pi$. Compute the probability that $\sin(x)\cos(x) < \frac{1}{4}$}

\prob{4}{MA$\theta$ 1992}{If $A$ and $B$ are both in $[0,2\pi)$ and $A$ and $B$ satisfy the equations
$$\sin A + \sin B = \frac{1}{3}$$
$$\cos A + \cos B  = \frac{4}{3}$$
find $\cos(A-B)$}

\prob{4}{TAMU 2019}{Simplify $\arctan \frac{1}{1+1+1^2}+ \arctan \frac{1}{1+2+2^2} + \arctan \frac{1}{1+3+3^2} \cdots + \arctan \frac{1}{1+n+n^2}$}

\begin{sol}
Use induction to show this is just $\boxed{\arctan \frac{n}{n+2}}$.
\end{sol}

\prob{6}{Purple Comet 2015 HS}{
Let $x$ be a real number between 0 and $\tfrac{\pi}{2}$ for which the function  $3\sin^2 x + 8\sin x \cos x + 9\cos^2 x$ obtains its maximum value, $M$. Find the value of $M + 100\cos^2x$.}

\prob{6}{ARML 2009}{If $6 \tan^{-1} x + 4 \tan^{-1}(3x) = \pi$, compute $x^2$}

\begin{sol}
$\boxed{\frac{15-8\sqrt{3}}{33}}$
\end{sol}

%%-----------------------------------------------------------------------------------------------------------------------------------%%
\subsection{Logarithms}
\prob{2}{AMC 12A 2018/14}{The solutions to the equation $\log_{3x} 4 = \log_{2x} 8$, where $x$ is a positive real number other than $\tfrac{1}{3}$ or $\tfrac{1}{2}$, can be written as $\tfrac {p}{q}$ where $p$ and $q$ are relatively prime positive integers. What is $p + q$?}

\begin{sol}
$\boxed{31}$
\end{sol}

\prob{2}{ARML 2005}{Let $A,R,M,L$ be positive reals such that \[\log_{10}(A\cdot L)+\log_{10}(A\cdot M)=2, \quad \log_{10}(M\cdot L)+\log_{10}(M\cdot R)=3,\quad \text{and}\quad\log_{10}(R\cdot A)+\log_{10}(R\cdot L)=4.\]Compute the value of the product $A\cdot R\cdot M\cdot L$. }

\prob{2}{MA$\theta$ 1991}{Given that $\log_{10}2 \approx 0.3010$, how many digits are in $5^{44}$?}

\begin{sol}
Note $\log_{10}(10)-\log_{10}(2)=\log_{10}(5)\implies \log_{10}(5) \approx 0.699\approx \frac{7}{10}$. Then, $5^{44}\approx 10^{44\cdot \frac{7}{10}}=10^{30.8}$. So, it has $\boxed{31}$ digits.
\end{sol}

\prob{3}{ARML 2009}{Compute all real values of $x$ such that $\log_{2}(\log_{2} x) = \log_{4}(\log_{4} x)$.}

\prob{3}{FARML 2008}{Given that $\log_a{16}-\log_2{a} = 6,$ compute $(\log_a{4})^2+(\log_4{a})^2.$}

\prob{3}{PuMAC 2019}{ If $x$ is a real number so $3^x = 27x$, compute $\log_{3}(\frac{3^{3^x}}{x^{3^{3}}})$.}

\begin{sol}
Note that $3^{x}=27x\implies x=3+\log_{3}(x)$. Also, $\log_{3}(\frac{3^{3^x}}{x^{3^{3}}}) = 3^{x}-27\log_{3}(x)=27x-27(x-3)=\boxed{81}$.
\end{sol}

\prob{3}{AMC 12B 2017/20}{Real numbers $x$ and $y$ are chosen independently and uniformly at random from the interval $(0,1)$. What is the probability that $\lfloor\log_2x\rfloor=\lfloor\log_2y\rfloor$?}

\prob{4}{CountofMC's Last-Minute Mock AMC 12}{ Let $x, y$, and $z$ be positive real numbers with $1 < x < y < z$ such that
$$log_{x} y + \log_{y}z + log_{z}x = 8 \text{ and}$$
$$\log_{z}x + \log_{z}y + \log_{y}x =\frac{25}{2}$$
The value of $\log_{y}(z)$ can then be written as $\frac{a+\sqrt{b}}{c}$ for positive integers $a, b$, and $c$ such that $b$ is not divisible by the square of any prime. What is the value of
$a + b + c$?}

\begin{sol}
$\boxed{42}$
\end{sol}
\prob{4}{MA$\theta$ 1991}{Suppose $a$ and $b$ are positive numbers for which $\log_{4n} 40\sqrt{3} = \log_{3n} 45$, find $n^3$}

\begin{sol}
Note $\log_{4n} 40\sqrt{3} = \log_{3n} 45\implies \frac{\log 40\sqrt{3}}{\log 4n} = \frac{\log 45}{\log 3n}$ for an arbitary base. Then, we also get $\frac{\log 40\sqrt{3}}{\log 4 + \log n} = \frac{\log 45}{\log 3 + \log n}\implies \log 120\sqrt{3} + \log 40n\sqrt{3} = \log 180 + \log 45n \implies \log \frac{8n\sqrt{3}}{9} = \log \frac{3}{2\sqrt{3}}= \log \frac{\sqrt{3}}{2}\implies n = \frac{9}{16}$. So, $n^3=\boxed{\frac{729}{4096}}$.
\end{sol}

\prob{4}{MA$\theta$ 1992}{Suppose $a$ and $b$ are positive numbers for which
 $$\log_{9} a  =\log_{15} b =\log_{25}(a+2b)$$
What is the value of $\frac{b}{a}$?}

\begin{sol}
$\boxed{-1+\sqrt{2}}$
\end{sol}

\prob{4}{PHS ARML TST 2017}{Positive real numbers $x, y$, and $z$ satisfy the following system of equations:
$$x^{\log(yz)} = 100$$
$$y^{\log(xz)} = 10$$
$$z^{\log(xy)} = 10\sqrt{10}$$
Compute the value of the expression $(\log(xyz))^2$}

\begin{sol}
Let $x=10^{a}, y=10^{b}, z = 10^{c}$. Let $\log(xyz)=a+b+c=k$. Then, we have $a(b+c)=2$ and $b(a+c)=1$ and $c(a+b)=\frac{3}{2}$. Summing, we get $2\sum_{cyc} ab = \frac{9}{2}\implies \sum_{cyc} ab =\frac{9}{4}$. Then, subtraction from each of the equations gives $ab=\frac{3}{4}, ac=\frac{5}{4},bc=\frac{1}{4}$. Then, multiplying all together, we get $abc=\frac{\sqrt{15}}{8}$. Solving, we get $a=\frac{\sqrt{15}}{2}, b= \frac{\frac{\sqrt{15}}{5}}{2}, c=\frac{\frac{\sqrt{15}}{3}}{2}$. Then, $a+b+c=\frac{23\sqrt{15}}{30}$. So, $(a+b+c)^2=\frac{529\cdot 15}{30^2}=\boxed{\frac{529}{60}}$.
\end{sol}

\prob{4}{}{If $60^{a}=3$ and $60^{b} = 5$, then find $12^{\frac{1-a-b}{2-2b}}$.}

\begin{sol}
$\boxed{2}$
\end{sol}

\prob{4}{NanoMath Fall Meet 2020}{Let a be a positive integer. If
$$(\log_{(\log_{2}(a^{\log_{2} a^4}))} a)(\log_{2} a^{\log_{4} a^{\frac{3}{2}}}) = 8$$,
what is the value of $a$?}

\begin{sol}
$\boxed{16}$
\end{sol}

\prob{5}{SLKK AIME 2020}{Let $x$ be a real number in the interval $(0,\frac{\pi}{2})$ such that
$\log_{\sin^2(x)} \cos(x) + \log_{\cos^2(x)} \sin(x) = \frac{5}{4}$.
If $\sin^2(2x)$ can be expressed as $m\sqrt{n} -p$, where $m, n$, and $p$ are positive integers such
that $n$ is not divisible by the square of a prime, find $m + n + p$.
}

\begin{sol}
$\boxed{17}$
\end{sol}

\prob{5}{}{Let $a = 256$. Find the unique real number $x > a^2$
such that
$$\log_{a} \log_{a} \log_{a} x = \log_{a^2} \log_{a^2} \log_{a^2} x$$.}

\begin{sol}
$\boxed{2^{32}}$
\end{sol}

\prob{6}{PUMaC 2017}{
Let $\Gamma$ be the maximum possible value of $a+3b+9c$ among all triples $(a,b,c)$ of positive real numbers such that
\[ \log_{30}(a+b+c) = \log_{8}(3a) = \log_{27} (3b) = \log_{125} (3c) .\]If $\Gamma = \frac{p}{q}$ where $p$ and $q$ are relatively prime positive integers, then find $p+q$.}

\prob{7}{AIME I 2017/14}{
Let $a > 1$ and $x > 1$ satisfy $\log_a(\log_a(\log_a 2) + \log_a 24 - 128) = 128$ and $\log_a(\log_a x) = 256$. Find the remainder when $x$ is divided by $1000$.}

\prob{7}{CIME 2021}{For distinct real numbers $k>1$ and $x>1$, the function $f_k(x)$ is defined as
\[ f_k(x)=\left | \dfrac{\log_x(\log_xk)}{\log_k(\log_kx)} \right |.\]For some positive integer $n$, it is given that
\[ \sum_{\substack{d \mid n \\ 1 < d < n}} f_d(n) = 9. \]Find the smallest possible value of $n$.}
%%-----------------------------------------------------------------------------------------------------------------------------------%%
\subsection{Distance, Rate, Work}
\prob{2}{AlcumusGuy Mock AMC 10 2016-2017}{Two cars simultaneously leave Lincoln at noon along the same route, with the first traveling at a constant rate of $50$ miles per hour, and the second at a constant rate of $40$ miles per hour. A third car leaves Lincoln half an hour later, traveling along the same route at constant speed $s$ miles per hour. It passes the first car $t$ minutes after noon, which was precisely an hour and a half after passing the second car.
What is $s + t$?}
\begin{sol}
$\boxed{240}$
\end{sol}
%%-----------------------------------------------------------------------------------------------------------------------------------%%
\subsection{Arbitary Functions}

\prob{2}{LMT Spring 2020}{A function $f(x)$ is such that for any integer $x, f (x)+ x f (2- x) = 6$. Compute $-2019f (2020)$.}

\begin{sol}
Plug in $x=2020$ and $x=-2018$ and solve the system of linear equations.
\end{sol}

\prob{3}{Mandelbrot Nationals 2009}{Let $f(x)$ be a function defined for all positive real numbers satisfying the conditions $f(x) > 0$ for all $x > 0$ and $f(x-y) = \sqrt{f(xy) + 1}$ for all $x > y > 0$. Determine $f(2009)$.}

\prob{4}{BMT 2020}{Let $f:\mathbb{R^{+}} \to \mathbb{R^{+}}$ be a function such that for all $x,y \in \mathbb{R^{+}}$, $f(x)f(y)=f(xy)+f(\frac{x}{y})$ where $\mathbb{R^{+}}$ represents the positive real numbers. Given that $f(2)=3$, compute the last two digits of $f(2^{2^{2020}})$.}

\begin{sol}
$\boxed{47}$
\end{sol}

\prob{5}{HMMT Feburary 2017}{Let $f : \mathbb{R} \to \mathbb{R}$ be a function satisfying $f(x)f(y) = f(x-y)$. Find all possible values of $f(2017)$.}

\prob{7}{MMATHS 2019}{The continuous function $f(x)$ satisfies $9f(x + y) = f(x)f(y)$ for all real numbers $x$ and $y$. If $f(1) = 3$, what is $f(-3)$?}

\begin{sol}
$f(x) is always nonnegative and can\rq{}t have any real zeros because then $f(0)=0$. So, $f(x)=3^{g(x)}$ for some continuous function $g(x)$. This then gives $2+g(2x)=2g(x)$ which implies $g(x)$ is linear and in the form $ax+b$. Then, $b=2$ and $a=-1$. So, $g(-3)=5$ and $f(-3)=\boxed{243}$
\end{sol}
%%-----------------------------------------------------------------------------------------------------------------------------------%%
\subsection{Inequalities}

\prob{1}{HMMT February 2013}{
The real numbers $x, y, z$ satisfy $0 \leq x \leq y \leq z \leq 4$. If their squares form an arithmetic progression
with common difference $2$, determine the minimum possible value of $|x - y| + |y - z|$.}

\begin{sol}
$\boxed{4-2\sqrt{3}}$
\end{sol}
\prob{1}{PuMAC 2019}{ Let a, b be positive integers such that $a+b = 10$. Let $\frac{p}{q}$
be the difference between the maximum and minimum possible values of $\frac{1}{a}+\frac{1}{b}$, where $p$ and $q$ are relatively prime positive integers.
Compute $p + q$.}

\begin{sol}
Note that $\frac{1}{a}+\frac{1}{b}=\frac{a+b}{ab}=\frac{10}{ab}$. To minimize this, we have to maximize $ab$ which happens when $a,b$ are equal or very close together (you can rigorously show this using a discrete smoothing arugment, i.e if they are not as close together as possible, then you can make the expression bigger by moving them together). That means $a=b=5\implies ab=25$ so we have a minimium of $\frac{2}{5}$. The maximium of the expression occurs when $a,b$ are very far apart (again by a discrete smoothing argument) when $a=1, b=9$, giving a maximium of $\frac{10}{9}$. Then, the difference is $\frac{10}{9}-\frac{2}{5}=\frac{50-18}{45}=\frac{32}{45}$, giving $\boxed{77}$.
\end{sol}

\prob{2}{Math Prizes For Girls 2019}{The degree measures of the six interior angles of a convex hexagon form
an arithmetic sequence (not necessarily in cyclic order). The common
difference of this arithmetic sequence can be any real number in the
open interval $(-D, D)$. Compute the greatest possible value of $D$.}

\begin{sol}
Let the angles be $a,a+d\cdots a+5d$.  Notice that any negative common difference can be turned into a positive common difference by simply starting at the maximium instead of the minimium. This same applies vice versa. So, let $d\ge 0$. Then, they sum to $180(4)=720$ giving $a+(a+d)\cdots + (a+5d)=720\implies (2a+5d)(3)=720\implies 2a+5d=240\implies a =\frac{240-5d}{2}$. So, any $d<\frac{240}{5}=48$ would corresond to a valid $a$. Notice that since the hexagon is convex, $a+5d<180\implies \frac{240+5d}{2} < 180 \implies \frac{48+d}{2} < 36\implies d < 24$. Now, it is clear that any $D\leq 24$ would work. So, the greatest possible value of $D$ is $\boxed{24}$.
\end{sol}

\prob{3}{Math Prizes For Girls 2019}{Find the least real number $K$ such that for all real numbers x
and y, we have $(1+20x^2)(1+19y^2)\ge Kxy$. Express your answer in simplified
radical form.}

\begin{sol}
By AM-GM, $(1+20x^2)\ge 2\sqrt{5}\cdot |x|$ and $(1+19y^2)\ge \sqrt{19}\cdot |y|$. Then, we get $(1+20x^2)(1+19y^2)\ge 2\sqrt{95} \cdot |x|\cdot |y|$ as $|x|,|y|$ are positive. Now, if both $x,y$ have the same sign, then $|x||y|=xy$ so $K=2\sqrt{95}$ for that case. But, if $x,y$ have opposite signs, $|x||y|=-xy$ so $(1+20x^2)(1+19y^2)\ge -19\sqrt{95}xy$. If $K$ was any larger in this case, it would not work. So, the least $K$ is $\boxed{-19\sqrt{95}}$.
\end{sol}

\prob{3}{PMC Mock AMC 10 2020}{An infinite geometric series with initial term $a$ and common ratio $b$ sums to $b - a$.
What is the maximum possible value of $a$?}

\begin{sol}
Isolate $a$ and use AM-GM, $\boxed{3-2\sqrt{2}}$
\end{sol}

\prob{4}{CountofMC's Last Minute Mock AMC 12}{Let $a,b$ and $c$ be positive real numbers such that
$$\frac{a^2}{2018-a}+\frac{b^2}{2018-b}+\frac{c^2}{2018-c}=\frac{1}{168}$$
What is the largest possible value of $a+b+c$?}

\begin{sol}
Cauchy-Schwarz, $\boxed{6}$
\end{sol}

\prob{4}{HMMT November 2013}{ Find the largest real number $\lambda$ such that $a^2 + b^2 + c^2 + d^2 \geq ab + \lambda bc + cd$ for all real numbers $a, b, c, d$.}

\prob{5}{HMMT February 2014}{Suppose that $x$ and $y$ are positive real numbers such that $x^2-xy+2y^2=8$.  Find the maximum possible value of $x^2+xy+2y^2$.}


\prob{5}{BMT 2019}{Find the number of ordered integer triplets $x, y, z$ with absolute value less than or equal to $100$
such that $2x^2 + 3y^2 + 3z^2 + 2xy + 2xz - 4yz < 5$}

\begin{sol}
 $(x+y)^2+(x+z)^2+2(y-z)^2<5$
\end{sol}
%%-----------------------------------------------------------------------------------------------------------------------------------%%
\subsection{Floors, Ceilings, Fractional Parts}

\prob{3}{AMC 10A 2020/22}{For how many positive integers $n \le 1000$ is$$\left\lfloor \dfrac{998}{n} \right\rfloor+\left\lfloor \dfrac{999}{n} \right\rfloor+\left\lfloor \dfrac{1000}{n}\right \rfloor$$not divisible by $3$? (Recall that $\lfloor x \rfloor$ is the greatest integer less than or equal to $x$.)}

\begin{sol}
$\boxed{22}$
\end{sol}

\prob{3}{HMMT February 2013}{ Find the number of integers $n$ such that
$$1+\lfloor \frac{100n}{101} \rfloor = \lceil \frac{99n}{100} \rceil$$}

\begin{sol}
Consider $f(n)= \lceil \frac{99n}{100} \rceil - \lfloor \frac{100n}{101} \rfloor$$
$\boxed{10100}$
\end{sol}

\prob{3}{CMC 10A 2021}{Let $a, b$ be not necessarily distinct positive integers chosen uniformly at random from
the set $\{1, 2, ..., 100\}$. Find the probability that
$$\lfloor \sqrt{a} \rfloor + \lceil \sqrt{b} \rceil = \lceil \sqrt{a} \rceil + \lfloor \sqrt{b} \rfloor$$
,where for real $x$, $\lfloor x\rfloor$ denotes the greatest integer less than or equal to $x$ and $\lceil x\rceil$ denotes
the least integer greater than or equal to $x$.}

\prob{5}{AMC 10B 2020/24}{
How many positive integers $n$ satisfy
$$\dfrac{n+1000}{70} = \lfloor \sqrt{n} \rfloor?$$
(Recall that $\lfloor x\rfloor$ is the greatest integer not exceeding $x$.)}

\begin{sol}
$\boxed{6}$
\end{sol}

\prob{4}{HMMT February 2013}{ Let $a_1,a_2,a_3,a_4,a_5$ be real numbers whose sum is $20$. Determine the smallest possible
value of
$$\sum_{1\leq i<j\leq 5} \lfloor a_{i}+a_{j}\rfloor$$}

\begin{sol}
$\boxed{72}$
\end{sol}

\prob{4}{Czech And Slovak MO}{
Solve the equation $x\cdot  \lfloor x\cdot \lfloor x \cdot \lfloor x \rfloor \rfloor \rfloor = 88$ in the set of real numbers.}

\begin{sol}
$x\ge 4$ gives that the expression is at least $4^4=256$ and $x\leq 3$ gives it is at most $27$.
So $x$ is in $[3,4]$. Since the expression is monotonically increasing, there is one solution.
Then, we have $x=\frac{88}{k}$ for some integer. So, we need to test $k=23,24\ldots 28,29$.

We get $x=\boxed{\frac{88}{28}=\frac{22}{7}}$
\end{sol}

\prob{5}{AMC 10B 2018/25}{Let $\lfloor x \rfloor$ denote the greatest integer less than or equal to $x$. How many real numbers $x$ satisfy the equation $x^2 + 10,000\lfloor x \rfloor = 10,000x$?}

\begin{sol}
$\boxed{199}$
\end{sol}

\prob{6}{AIME II 2020/14}{For real number $x$ let $\lfloor x\rfloor$ be the greatest integer less than or equal to $x$, and define $\{x\} = x - \lfloor x \rfloor$ to be the fractional part of $x$. For example, $\{3\} = 0$ and $\{4.56\} = 0.56$. Define $f(x)=x\{x\}$, and let $N$ be the number of real-valued solutions to the equation $f(f(f(x)))=17$ for $0\leq x\leq 2020$. Find the remainder when $N$ is divided by $1000$.}

\begin{sol}
$\boxed{10}$
\end{sol}

%%-----------------------------------------------------------------------------------------------------------------------------------%%
\subsection{Fake Algebra}
\prob{3}{BMT 2019}{Find the maximum value of $\frac{x}{y}$ if x and y are real numbers such that $x^2 +y^2-8x-6y + 20 = 0$.}

\begin{sol}
Let $\frac{x}{y}=k$. Then, $y=kx$. Plugging this in, we get $(k^2+1)x^2-(8+6k)x+20=0$. Clearly, the maximium $k$ occurs when the determinant is $0$. This implies $(8+6k)^2-4(20)(k^2+1)=(36k^2+96k+64)-(80k^2+80)=-44k^2+96k-16=0\implies 11k^2-24k+4=0\implies (11k-2)(k-2)=0$. Then, we get $k=2$ and $k=\frac{2}{11}$. Our answer is $\boxed{2}$.
\end{sol}
\prob{3}{BMT 2015}{Let x and y be real numbers satisfying the equation $x^2 - 4x + y^2 + 3 = 0$. If the maximum
and minimum values of $x^2 + y^2$
are $M$ and $m$ respectively, compute the numerical value of
$M - m$.}

\begin{sol}
We rewrite to get $(x-2)^2+y^2=1$ which is the equation of a circle. Let $P=(x,y)$ and $O=(0,0)$. Then, $x^2+y^2=PO^2$. So, we want to find the minimium of $PO$ and maximium of $PO$. Let $C=(2,0)$ which is the center of the circle. Clearly, the maximium and minimium occur where $PC$ intersects the circle. So, the minimium is $2-1=1$ and the maximium is $2+1=3$. So, $M-m=9-1=\boxed{8}$.
\end{sol}

\prob{3}{Magic Math AMC 10 2020}{Find the number of ordered pairs of real numbers $(x, y)$ satisfying $x =\frac{2y}{y^2-1}$ and $y =\frac{2x}{x^2-1}$.
}

\begin{sol}
Let $x=\tan  \alpha$ and $y=\tan \beta. $\boxed{5}$
\end{sol}

\prob{4}{ARML 2014}{Compute the area of the region defined by $x^2 + y^2 \leq |x| + |y|$.}
\begin{sol}
$\boxed{2+\pi}$
\end{sol}
\prob{4}{PuMAC 2019}{ Let x and y be positive real numbers that satisfy $(\log x)^2+(\log y)^2 = \log x^2 + \log y^2$.
Compute the maximum possible value of $(\log xy)^2$.
}

\begin{sol} 
Subsitute $\log x =a$, $\log y = b$. You get the equation of a circle $(a-1)^2+(b-1)^2=2$. You want to find the y-intersect of tangent line with slope $-1$ on "top" of the circle. Draw a perpendicular to the line from the center to find the tangency point. This has slope $1$ and it is $\sqrt{2}$ long. So, the coordinates of this tangency is $(2,2)$ and $a+b=4$. We get $(a+b)^2=\boxed{16}$.
\end{sol}

\prob{5}{PersonPyschopath\rq{}s 2nd 2015-2016 Mock AMC 10}{What is the square root of the value of 
$$(\sqrt{85}+\sqrt{205}+\sqrt{218})(-\sqrt{85}+\sqrt{205}+\sqrt{218})(\sqrt{85}-\sqrt{205}+\sqrt{218})(\sqrt{85}+\sqrt{205}-\sqrt{218})$$}

\begin{sol}
Heron\rq{}s Formula, triangle inscribed in a square. Then, $\boxed{254}$
\end{sol}

\prob{7}{HMMT February 2014}{Given that $a, b,$ and $c$ are complex numbers satisfying
$$a^2 + ab + b^2 = 1+ i$$
$$b^2 + bc + c^2 = 2$$
$$c^2 + ca + a^2 = 1,$$
compute $(ab + bc + ca)^2$}
%%-----------------------------------------------------------------------------------------------------------------------------------%%
\subsection{Graphing}

\prob{4}{Reun\rq{}s Thanksgiving 2017 Mock AMC 10}{Two functions $f$ and $g$ exist such that
$$f(x)=-\frac{x-15}{x+1}$$
and
$$g(x)=-\frac{x+17}{x+1}$$
intersect the circle $x^2+y^2=c$ for some constant $c$ at exactly $8$ points.
What is the least possible integer value of $c$?}

\begin{sol}
Furthest point from graph of $f(x)$ and $g(x)$ to origin is $\sqrt{50}$, $\boxed{51}$
\end{sol}
%%-----------------------------------------------------------------------------------------------------------------------------------%%
\subsection{Miscellaneous}

\prob{2}{NEMO 2017}{
Find all ordered 8-tuples of positive integers $(a, b, c, d, e, f, g, h)$ such that:
$$\frac{1}{a^3}+\frac{1}{b^3}+\frac{1}{c^3}+\frac{1}{d^3}+\frac{1}{e^3}+\frac{1}{f^3}+\frac{1}{g^3}+\frac{1}{h^3}=1$$
}

\begin{sol}
Bounding argument, WLOG $a$ largest, then $a=1$ and $a\ge 3$ don\rq{}t work. $\boxed{(2,2,2,2,2,2,2,2)}$.
\end{sol}

\prob{3}{PersonPsychopath\rq{}s 2nd 2015-2016 Mock AMC 10}{A certain post office only uses rectangular prism shaped packages that come in two sizes: small and large. All small packages have a surface area of 200 and a volume of 160. Doubling the height in a small package will result in a large package. If the surface area of each large package is equal
to its volume, what is the sum of the length, width, and height of the small package?}

\begin{sol}
$\boxed{19}$
\end{sol}

\prob{3}{Payback Mock AMC 10}{A $20 \times 20$ multiplication table has twenty numbers representing each individual row
and column, namely $2^{0},2^{1},2^{2}\ldots 2^{19}$ A number is formed by multiplying its row-
number and column-number. What is the remainder when the sum of all numbers in the multiplication table is divided by 5?}

\begin{sol}
$\boxed{0}$
\end{sol}

\prob{4}{NEMO 2017}{The value of the expression
$$\sqrt{1+\sqrt{\sqrt[3]{32}-\sqrt[3]{16}}} + \sqrt{1-\sqrt{\sqrt[3]{32}-\sqrt[3]{16}}}$$
can be written as $\sqrt[m]{n}$, where $m$ and $n$ are positive integers. Compute the smallest possible value of
$m + n$.}

\begin{sol}
Let the left term be $a$ and the right term be $b$.
Then, $a^2+b^2=2$. 
$ab=\sqrt{1-\sqrt[3]{32}+\sqrt[3]{16}}=\sqrt[3]{4}-1$.
Then, $(a+b)^2=2\sqrt[3]{4}=2^{\frac{5}{3}}$. So, $a+b=2^{\frac{5}{6}}=32^{\frac{1}{6}}$. So, $\boxed{38}$.
\end{sol}
%%-----------------------------------------------------------------------------------------------------------------------------------%%
%%%%%%%%%%%%%%%%%%%%%%%%%%%%%%%%%%%%%%%%%%%%%%%%%%%%%
%%-----------------------------------------------------------------------------------------------------------------------------------%%
\setcounter{problem}{0}
\section{Geometry}
%%-----------------------------------------------------------------------------------------------------------------------------------%%
\subsection{Coordinate Geometry}

\prob{1}{BmMT 2014}{Find the area of the convex quadrilateral with vertices at the points $(-1, 5),(3, 8),(3, -1)$,
and $(-1, -2)$.}

\begin{sol} 
Direct application of Shoelace.
\end{sol}

\prob{2}{CRMT Individuals 2019}{
Let $S$ be the set of all distinct points in the coordinate plane that form an acute isosceles triangle
with the points $(32, 33)$ and $(63, 63)$. Given that a line $L$ crosses $S$ a finite number of times, find
the maximum number of times $L$ can cross $S$.
}

\begin{sol}
Replace $(32,33)$ and $(63,63)$ by $A$ and $B$. Then, we do casework on $AC=BC$ or $CB=AB$ or $CA=BA$. We get a line and two semicircles. A line can intersect a semicircle two times and a line one time.  So, our answer is $4+1=\boxed{5}$.
\end{sol}

\prob{2}{HMMT November 2013}{Plot points A, B, C at coordinates (0, 0), (0, 1), and (1, 1) in the plane, respectively. Let S denote
the union of the two line segments AB and BC. Let X1 be the area swept out when Bobby rotates
S counterclockwise 45 degrees about point A. Let X2 be the area swept out when Calvin rotates S
clockwise 45 degrees about point A. Find $\frac{X_{1}+X_{2}}{2}$}

\prob{2}{AMC 12B 2017/9}{A circle has center $(-10,-4)$ and radius $13$. Another circle has center $(3,9)$ and radius $\sqrt{65}$. The line passing through the two points of intersection of the two circles has equation $x + y = c$. What is $c$?}

\prob{3}{Math Prizes For Girls 2019}{Two ants sit at the vertex of the parabola $y = x^2$. One starts walking
northeast (i.e., upward along the line $y = x$) and the other starts
walking northwest (i.e., upward along the line $y = -x$). Each time they
reach the parabola again, they swap directions and continue walking.
Both ants walk at the same speed. When the ants meet for the eleventh
time (including the time at the origin), their paths will enclose $10$
squares. What is the total area of these squares?}

\begin{sol}
$\boxed{770}$
\end{sol}

\prob{4}{AMC 12A 2018/22}{The solutions to the equations $z^2=4+4\sqrt{15}i$ and $z^2=2+2\sqrt 3i,$ where $i=\sqrt{-1},$ form the vertices of a parallelogram in the complex plane. The area of this parallelogram can be written in the form $p\sqrt q-r\sqrt s,$ where $p,$ $q,$ $r,$ and $s$ are positive integers and neither $q$ nor $s$ is divisible by the square of any prime number. What is $p+q+r+s?$}

\begin{sol}
DeMovire, Half-Angle, and then Shoelace
$\boxed{20}$
\end{sol}

\prob{5}{BMT 2019}{
A regular hexagon has positive integer side length. A laser is emitted from one of the hexagon’s
corners, and is reflected off the edges of the hexagon until it hits another corner. Let $a$ be the
distance that the laser travels. What is the smallest possible value of $a^2$ such that $a > 2019$?
You need not simplify/compute exponents.}

\prob{5}{SLKK AIME 2020}{Mr. Duck draws points $ A = (a, 0), B = (0, b), C = (3, 5)$ and $O = (0, 0)$
such that $a, b > 0$ and $\angle ACB = 45^{\circ}$. If the maximum possible area of $\triangle AOB$ can be
expressed as $m-n\sqrt{p}$ where $m, n,$ and $p$ are positive integers such that p is not divisible
by the square of a prime, find $m + n + p$
}

\begin{sol}
$\boxed{85}$
\end{sol}

%%-----------------------------------------------------------------------------------------------------------------------------------%%
\subsection{3D Geometry}
\prob{1}{AMC 12A 2009/5}{One dimension of a cube is increased by $1$, another is decreased by $1$, and the third is left unchanged. The volume of the new rectangular solid is $5$ less than that of the cube. What was the volume of the cube?}

\begin{sol}
Let $x$ be the original side length of the cube. Then, we have $(x+1)(x-1)(x)+5=x^3\implies x^3-x+5=x^3\implies x=5$. So, the original volume is $x^3=\boxed{125}$.
\end{sol}

\prob{1}{AMC 12A 2008/8}{What is the volume of a cube whose surface area is twice that of a cube with volume 1?}

\begin{sol}
The surface area of a cube with volume $1$ is $6\cdot 1^2=6$. Then, the surface area of the larger cube is $12$ and so each face has area $2$. This implies the side length is $\sqrt{2}$. The volume is $2\sqrt{2}$.
\end{sol}

\prob{1}{Reun\rq{}s Thanksgiving 2017 Mock AMC 10}{A silo is a three-dimensional structure, made up of a cylinder and a cone
with equal radii of 3, that stores fodder. The height of the conic part of
the silo is 3 units. When the entire silo is half full of fodder, the cylindrical
part of the silo is $\frac{2}{3}$ full of fodder. What is the volume of the silo?}

\begin{sol}
$\boxed{36\pi}$
\end{sol}

\prob{1}{BMT 2018}{A cube has side length $5$. Let $S$ be its surface area and $V$ its volume. Find $\frac{S^3}{V^2}$.}

\begin{sol}
Note $S=6\cdot 5^2=150$ and $V=5^3$. Then, $\frac{S^3}{V^2}=\frac{6^3\cdot 5^6}{5^6}=\boxed{216}$.
\end{sol}

\prob{2}{AMC 12B 2008/18}{On a sphere with a radius of $2$ units, the points $A$ and $B$ are $2$ units away from each other. Compute the distance from the center of the sphere to the line segment $AB.$}

\begin{sol}
Consider the great circle that goes through both $A$ and $B$. Also, let the center of the sphere be $O$. Then, drop the perpendicular from $O$ to $AB$ with foot $C$. Then, $C$ bisects $AB$ with $AC=CB=1$. Also, $OB=2$. Using the Pythagorean Theorem, $OC=\boxed{\sqrt{3}}$.
\end{sol}

\prob{2}{DMC Mock AMC 10}{A plane cuts into a sphere of radius $11$ such that the area of the region of the plane
inside the sphere is $108\pi$. A perpendicular plane cuts into the sphere such that the
area of the region of the plane inside the sphere is $94\pi$. Given that the two planes
intersect at a line, what is the length of the segment of the line inside the sphere?}

\begin{sol}
$\boxed{18}$
\end{sol}

\prob{2}{AMC 12B 2005/16}{Eight spheres of radius 1, one per octant, are each tangent to the coordinate planes. What is the radius of the smallest sphere, centered at the origin, that contains these eight spheres?}

\begin{sol}
We are basically finding the distance from the origin to the furtherest point on these spheres and that will be our radius. It is clear that a sphere with this radius will work and that if this distance is any smaller, the sphere will not completely contain all the other spheres. It is also clear that the spheres are symmetric so we only need to consider one of the spheres to find this furthest distance.  

We consider the sphere in the positive x,y,z direction. Then the center $C$ of this sphere is $(1,1,1)$. So, $OC=\sqrt{3}$. Now, it is clear that the furthest point $P$ is the other intersection of $OC$ with the sphere. 

So, the furthest distance is $\boxed{1+\sqrt{3}}$.
\end{sol}
\prob{3}{AIME 1984/9}{In tetrahedron $ABCD$, edge $AB$ has length 3 cm. The area of face $ABC$ is $15\mbox{cm}^2$ and the area of face $ABD$ is $12 \mbox { cm}^2$. These two faces meet each other at a $30^\circ$ angle. Find the volume of the tetrahedron in $\mbox{cm}^3$.}

\prob{3}{Scrabber94 Mock AMC 10 2020}{A right triangular prism has all edges of length 2. An ant crawls on the
exterior of the prism from point A to point B, where B is the midpoint of
the edge opposite A as shown. What is the shortest possible distance the
ant crawls?}

\begin{sol}
Make a net. Then, $\boxed{\sqrt{7+2\sqrt{3}}$.
\end{sol}

\prob{3}{HMMT February 2014}{Let $C$ be a circle in the $xy$ plane with radius 1 and center $(0, 0, 0)$, and let P be a point in space
with coordinates $(3, 4, 8)$. Find the largest possible radius of a sphere that is contained entirely in the
slanted cone with base C and vertex P.}

\prob{3}{AMC 12B 2017/14}{An ice-cream novelty item consists of a cup in the shape of a 4-inch-tall frustum of a right circular cone, with a 2-inch-diameter base at the bottom and a 4-inch-diameter base at the top, packed solid with ice cream, together with a solid cone of ice cream of height 4 inches, whose base, at the bottom, is the top base of the frustum. What is the total volume of the ice cream, in cubic inches?}

\prob{4}{ARML 2009}{A cylinder with radius $r$ and height $h$ has volume $1$ and total surface area $12$. Compute $\frac{1}{r}+\frac{1}{h}$}

\begin{sol}
$\boxed{6}$
\end{sol}

\prob{6}{BMT 2018}{A rectangular prism has three distinct faces of area 24, 30, and 32. The diagonals of each distinct
face of the prism form sides of a triangle. What is the triangle’s area?}

\begin{sol}
Use 3D cross-product, $\boxed{25}$
\end{sol}

%%-----------------------------------------------------------------------------------------------------------------------------------%%
\subsection{Inequalties}
\prob{1}{AlcumusGuy Mock AMC 10 2016-2017}{Triangle AMC has area 2017 with $\angle M = 90^{\circ}$. What is the closest integer to the minimum possible length of AC?}

\begin{sol}
$\boxed{90}$
\end{sol}

\prob{1}{Math Prizes for Girls 2019}{The degree measures of the six interior angles of a convex hexagon form
an arithmetic sequence (not necessarily in cyclic order). The common
difference of this arithmetic sequence can be any real number in the
open interval $(-D, D)$. Compute the greatest possible value of $D$.}

\begin{sol}
$\boxed{24}$
\end{sol}
\prob{2}{Math Prizes For Girls 2019}{A paper equilateral triangle with area 2019 is folded over a line parallel
to one of its sides. What is the greatest possible area of the overlap of
folded and unfolded parts of the triangle?}

\begin{sol}
$\boxed{673}$
\end{sol}
\prob{3}{BMT 2018}{If $A$ is the area of a triangle with perimeter $1$, what is the largest possible value of $A^2$?}

\begin{sol}
Using Heron's Formula and a smoothing argument gives that an equilaterial triangle gives the largest area.  So, we have $s=\frac{1}{3}$ and $A=\frac{s^2\sqrt{3}}{4}=\frac{\sqrt{3}}{36}$. So, $A^2=\frac{3}{36^2}=\Fefrac{1}{12\cdot 36}=\boxed{\frac{1}{432}}$.
\end{sol}

\prob{4}{PersonPsychopath\rq{}s 2nd 2015-2016 Mock AMC 10}{Two unit circles are externally tangent to each other at point $P$. A pair of perpendicular lines is drawn such that each line is tangent to a different circle. Let the intersection of these lines be $I$. What is the minimum possible length of $PI$?}

\begin{sol}
$\boxed{\sqrt{2}-1}$
\end{sol}

\prob{4}{MMATHS 2018}{
$A, B, C, D$ all lie on a circle with $AB = BC = CD$. If the distance between any two of these points is a positive integer, what is the smallest possible perimeter of quadrilateral $ABCD$}

\begin{sol}
$\boxed{17}$
\end{sol}

%%-----------------------------------------------------------------------------------------------------------------------------------%%
\subsection{Transformations}
\prob{3}{HMMT February 2014}{
In quadrilateral ABCD, $\angle DAC = 98, \angle DBC = 82, \angle BCD = 70$, and $BC = AD$. Find $\angle ACD$.}

\begin{sol}
Reflect.
\end{sol}

\prob{6}{LMT Spring 2020}{Let ABC be a triangle such that such that $AB = 14,BC = 13,$ and $AC = 15$. Let $X$ be a point inside triangle $ABC$. Compute the minimum possible value of $(\sqrt{2}AX +B X +C X)^2$.}

\prob{8}{SLKK AIME 2020}{Squares $ABCD$ and $DEFG$ are drawn in the plane with both sets of
vertices $A, B, C, D$ and $D, E, F, G$ labeled counterclockwise. Let P be the intersection of
lines $AE$ and $CG$. If $DA = 35, DG = 20$, and $BF = 25\sqrt{2}$, find $DP^2$.}

\begin{sol}
Spiral Similarity from ABCD to DEFG. $\boxed{768}$
\end{sol}

%%-----------------------------------------------------------------------------------------------------------------------------------%%
\subsection{Trigonometry}
\prob{3}{AMC 12B 2017/15}{Let $ABC$ be an equilateral triangle. Extend side $\overline{AB}$ beyond $B$ to a point $B'$ so that $BB'=3 \cdot AB$. Similarly, extend side $\overline{BC}$ beyond $C$ to a point $C'$ so that $CC'=3 \cdot BC$, and extend side $\overline{CA}$ beyond $A$ to a point $A'$ so that $AA'=3 \cdot CA$. What is the ratio of the area of $\triangle A'B'C'$ to the area of $\triangle ABC$?}

\begin{sol}
$\sqrt{37}$
\end{sol}

\prob{3}{ARML 2014}{In triangle $ABC, a = 12, b = 17,$ and $c = 13$. Compute $b cos C - c cos B$.}

\begin{sol}
Find the expression in terms of segments and use Pythaogrean Theorem and Difference of Squares to easily compute.
$\boxed{10}$
\end{sol}

\prob{4}{AMC 12A 2018/20}{Triangle $ABC$ is an isosceles right triangle with $AB=AC=3$. Let $M$ be the midpoint of hypotenuse $\overline{BC}$. Points $I$ and $E$ lie on sides $\overline{AC}$ and $\overline{AB}$, respectively, so that $AI>AE$ and $AIME$ is a cyclic quadrilateral. Given that triangle $EMI$ has area $2$, the length $CI$ can be written as $\frac{a-\sqrt{b}}{c}$, where $a$, $b$, and $c$ are positive integers and $b$ is not divisible by the square of any prime. What is the value of $a+b+c$?}

\begin{sol}
$\boxed{12}$
\end{sol}

\prob{4}{ARML 2014}{In triangle $\triangle ABC$, $BC = 2$. Point $D$ is on $AC$ such that $AD = 1$ and $CD = 2$. If
$m\angle BDC = 2m\angle A$, compute $\sin A$.}

\begin{sol}
Note that $\angle DBA = \angle A$ so $BD=DA=1$. We then apply Law of Cosines on $\triangle BDC$ and bash to get $\boxed{\frac{\sqrt{6}}{4}}$.
\end{sol}

\prob{5}{AMC 12A 2018/23}{In $\triangle PAT,$ $\angle P=36^{\circ},$ $\angle A=56^{\circ},$ and $PA=10.$ Points $U$ and $G$ lie on sides $\overline{TP}$ and $\overline{TA},$ respectively, so that $PU=AG=1.$ Let $M$ and $N$ be the midpoints of segments $\overline{PA}$ and $\overline{UG},$ respectively. What is the degree measure of the acute angle formed by lines $MN$ and $PA?$}

\begin{sol}
Sum to product formula, $\boxed{80^{\circ}}$.
\end{sol}

%%-----------------------------------------------------------------------------------------------------------------------------------%%
\subsection{General}

\prob{1}{HMMT February 2013}{
Jarris the triangle is playing in the $(x,y)$ plane. Let his maximum $y$ coordinate be $k$. Given that he
has side lengths $6, 8$, and $10$ and that no part of him is below the x-axis, find the minimum possible
value of $k$.}

\begin{sol}
Altitude, $\boxed{\frac{24}{5}}$
\end{sol}

\prob{1}{MA$\theta$ 2018}{A parallelograms has diagonals of length $10$ and $20$. Find the area inclosed by the circle inscribed in the parallelogram.
}

\prob{1}{TAMU 2019}{An acute isosceles triangle ABC is inscribed in a circle. Through B and C, tangents to
the circle are drawn, meeting at D. If $\angle ABC = 2\angle CDB$, then find the radian measure of
$\angle BAC$.}

\prob{1}{PHS PuMAC TST 2017}{In triangle $ABC$, let $D$ and $E$ be the midpoints of $BC$ and $AC$. Suppose $AD$ and $BE$ meet at $F$. If
the area of $\triangle DEF$ is $50$, then what is the area of $\triangle CDE$?}

\prob{1}{PersonPsychopath\rq{}s 2nd Mock AMC 10}{In convex pentagon $ABCDE, AB = 40, BC = 20, CD = 5$, and $DE = 37$. If $\angle A, \angle B$, and $\angle C$ are all right angles, what is the length of $EA$?}

\begin{sol}
$\boxed{88}$
\end{sol}

\prob{1}{AMC 10A 2019/13}{Let $\triangle ABC$ be an isosceles triangle with $BC = AC$ and $\angle ACB = 40^{\circ}$. Construct the circle with diameter $\overline{BC}$, and let $D$ and $E$ be the other intersection points of the circle with the sides $\overline{AC}$ and $\overline{AB}$, respectively. Let $F$ be the intersection of the diagonals of the quadrilateral $BCDE$. What is the degree measure of $\angle BFC?$}

\begin{sol}
$\boxed{110}$
\end{sol}

\prob{2}{AMC 10B 2011/17}{In the given circle, the diameter $\overline{EB}$ is parallel to $\overline{DC}$, and $\overline{AB}$ is parallel to $\overline{ED}$. The angles $AEB$ and $ABE$ are in the ratio $4 : 5$. What is the degree measure of angle $BCD?$}

\begin{sol}
$\boxed{130}$
\end{sol}

\prob{2}{Brazil 2007}{Let $ABC$ be a triangle with circumcenter $O$. Let $P$ be the intersection of straight lines $BO$ and $AC$ and $\omega$ be the circumcircle of triangle $AOP$. Suppose that $BO = AP$ and that the measure of the arc $OP$ in $\omega$, that does not contain $A$, is $40^{\circ}$. Determine the measure of the angle $\angle OBC$.}

\begin{sol}
$\boxed{30^{\circ}}$
\end{sol}

\prob{2}{BMT 2018}{A $1$ by $1$ square ABCD is inscribed in the circle $m$. Circle $n$ has radius $1$ and is centered around
$A$. Let $S$ be the set of points inside of $m$ but outside of $n$. What is the area of $S$?}

\prob{2}{AMC 10B 2011/18}{Rectangle $ABCD$ has $AB = 6$ and $BC = 3$. Point $M$ is chosen on side $AB$ so that $\angle AMD = \angle CMD$. What is the degree measure of $\angle AMD?$}

\begin{sol}
$\boxed{75^{\circ}}$
\end{sol}

\prob{2}{TAMU 2019}{Let $AA_{1}$ be an altitude of triangle $\triangle ABC$, and let $A_{2}$ be the midpoint of the side $BC$.
Suppose that $AA_{1}$ and $AA_{2}$ divide angle $\angle BAC$ into three equal angles. Find the product of the angles of 
$\triangle ABC$ when the angles are expressed in degrees.}

\begin{sol}
Let $\angle BAA_{1}=\angle A_{1}AA_{2}=\angle A_{2}AC = \alpha$. We have that $\triangle ABA_{2}$ is an isosceles triangle as $\angle ABA_{1}=\angle AA_{2}A_{1}=90-\alpha$. Then, as $AA_{1}$ is an altitude, $BA_{1}=A_{1}A_{2}$. Let $BA_{1}=A_{1}A_{2}=x$, then $A_{2}C=BA_{2}=2x$. Consider triangles $BAA_{1}$ and $\triangle A_{1}AC$. We have that $\tan \alpha = \frac{x}{AA_{1}}$ and $\tan 2\alpha = \frac{3x}{AA_{1}}$. So, $\frac{\tan 2\alpha}{\tan \alpha}= 3 \implies \tan \alpha = \frac{1}{\sqrt{3}}\implies \alpha = 30$. So, $\angle BAC = 90$, $\angle ABC = 60$, and $\angle ACB=30$. The product is $\boxed{162000}$.
\end{sol}

\prob{2}{PHS PuMAC TST 2017}{A trapezoid has area 32, and the sum of the lengths of its two bases and altitude is 16. If one of the
diagonals is perpendicular to both bases, then what is the length of the other diagonal?}

\prob{2}{Autumn Mock AMC 10}{Equilateral triangle ABC has side length $6$. Points $D, E, F$ lie within the lines $AB, BC$ and $AC$ such that $BD=2AD$, $BE=2CE$, and $AF=2CF$. Let $N$ be the numerical value of the area of triangle $DEF$. Find $N^2$.}

\begin{sol}
$\boxed{12}$
\end{sol}

\prob{2}{OMO Spring 2019}{In triangle ABC, side AB has length 10, and the A- and B-medians have length 9 and 12, respectively. Compute the area of the triangle.}

\begin{sol}
$\boxed{72}$
\end{sol}

\prob{2}{LMT Spring 2020}{
Three mutually externally tangent circles are internally tangent to a circle with radius $1$. If two of the inner circles have
radius $\frac{1}{3}$, the largest possible radius of the third inner circle can be expressed in the form $\frac{a+b\sqrt{c}}{d}$ where $c$ is squarefree
and $\gcd(a,b,d) = 1$. Find $a +b +c +d$.}

\prob{2}{2018 AMC 12A/17}{Farmer Pythagoras has a field in the shape of a right triangle. The right triangle's legs have lengths 3 and 4 units. In the corner where those sides meet at a right angle, he leaves a small unplanted square $S$ so that from the air it looks like the right angle symbol. The rest of the field is planted. The shortest distance from $S$ to the hypotenuse is 2 units. What fraction of the field is planted?}

\begin{sol}
$\boxed{\frac{145}{147}}$
\end{sol}

\prob{2}{2018 AMC 12A/18}{Triangle $ABC$ with $AB=50$ and $AC=10$ has area $120$. Let $D$ be the midpoint of $\overline{AB}$, and let $E$ be the midpoint of $\overline{AC}$. The angle bisector of $\angle BAC$ intersects $\overline{DE}$ and $\overline{BC}$ at $F$ and $G$, respectively. What is the area of quadrilateral $FDBG$?}

\begin{sol}
$\boxed{75}$
\end{sol}

\prob{3}{AHSME 1984/28}{Triangle ABC has area $10$. Points $D, E$, and $F$, all distinct from $A, B$, and $C$, are on sides $AB, BC$, and $CA$, respectively, and $AD = 2, DB = 3$. Triangle ABE and quadrilateral
$DBEF$ have equal areas $s$. Find $s$.}

\prob{3}{FARML 2012/6}{In triangle $ABC,$ $AB=7,$ $AC=8,$ and $BC=10.$ $D$ is on $AC$ and $E$ is on $BC$ such that $\angle AEC=\angle BED=\angle B+\angle C.$ Compute the length $AD.$}

\begin{sol}
$\frac{13}{8}$
\end{sol}

\prob{3}{HMMT November 2013}{Let ABC be an isosceles triangle with $AB = AC$. Let D and E be the midpoints of segments $AB$ and $AC$, respectively. Suppose that there exists a point F on ray $\overrightarrow{DE}$ outside of $ABC$ such that triangle $BFA$ is similar to triangle $ABC$. Compute $\frac{AB}{BC}$ .}

\prob{3}{Magic Math AMC 10 2020}{Two concentric circles have radii $\sqrt{2}$ and $2$. Three points $A, B, C$ are chosen on the larger circle such that $\triangle ABC$ is equilateral. Find the area of the region in the smaller circle that is also inside the triangle.}

\begin{sol}
$3+\frac{\pi}{2}$
\end{sol}

\prob{3}{CNCM Online Round 2}{There is a rectangle $ABCD$ such that $AB = 12$ and $BC = 7$. $E$ and $F$ lie on sides $AB$
and CD respectively such that $\frac{AE}{EB} = 1$ and $\frac{CF}{FD} =\frac{1}{2}$. Call $X$ the intersection of $AF$ and $DE$. What is
the area of pentagon $BCFXE$?
}

\prob{3}{LMT Spring 2020}{
Let $\triangle ABC$ be a triangle such that $AB = 6,BC = 8$, and $AC = 10$. Let $M$ be the midpoint of $BC$. Circle $\omega$ passes through $A$ and is tangent to $BC$ at $M$. Suppose $\omega$ intersects segments $AB$ and $AC$ again at points $X$ and $Y$, respectively. If the area of $\triangle AXY$ can be expressed as $\frac{p}{q}$ where $p,q$ are relatively prime integers, compute $p + q$.}

\begin{sol}
Power of a Point, then Sine Triangle Area formula.
\end{sol}

\prob{3}{JMC 10 Year 1}{Consider triangle $ABC$ with medians $BE$ and $CF$ that intersect at $G$. If $AG = BC = 8$ and $CG = 6$, what is the length of $\overline{GE}$?}

\begin{sol}
$\boxed{\sqrt{7}}$
\end{sol}
\prob{3}{AMC 12B 2017/18}{The diameter $AB$ of a circle of radius $2$ is extended to a point $D$ outside the circle so that $BD=3$. Point $E$ is chosen so that $ED=5$ and line $ED$ is perpendicular to line $AD$. Segment $AE$ intersects the circle at a point $C$ between $A$ and $E$. What is the area of $\triangle 
ABC$?}

\prob{3}{MMATHS 2018}{In rectangle $ABCD$, let $E$ lie on $CD$, and let $F$ be the intersection of $AC$ and $BE$. If the area of $\triangle ABF$
is $45$ and the area of $\triangle CEF$ is $20$, find the area of the quadrilateral $ADEF$.}

\begin{sol}
$\boxed{\frac{505}{4}}$
\end{sol}
\prob{3}{PHS ARML TST 2017}{An algorithm starts with an equilateral triangle $A_{0}B_{0}C_{0}$ of side length 1. At step k, points $A_k, B_k,
$and $C_k$ are chosen on line segments $B_{k-1}C_{k-1}, C_{k-1}A_{k-1}$ and $A_{k-1}B_{k-1}$ respectively, such that

$$B_{k-1}A_{k}:A_{k}C_{k-1}=1:1$$
$$C_{k-1}B_{k}:B_{k}A_{k-1}=1:2$$ 
$$A_{k-1}C_{k}:C_{k}B_{k-1}=1:3$$

What is the value of the infinite series:
$$\sum_{i=0}^{\infty} \text{Area}[\triangle A_{k}B_{k}C_{k}]$$}

\prob{3}{PersonPsychopath\rq{}s 2nd 2015-2016 Mock AMC 10}{In $\triangle ABC$, $\angle A = 90^{\circ}$. Points $X$ and $Y$ lie on $AB$ and $AC$ respectively such that $\frac{AX}{AB} = \frac{1}{2}$ and $\frac{AY}{AC} = \frac{1}{3}$. $BY$ and $CX$ intersect at $I$. What is the ratio of the area of $AXIY$ to $\triangle ABC$?}

\begin{sol}
$\boxed{\frac{7}{30}}$
\end{sol}

\prob{3}{Reun's Pre-real Mock AMC 10}{All diagonals of hexagon $ABCDEF$ are drawn. If $AE = DB = 13$, $EC = BF = 14$, $CA = FD = 15$ and $AD = BE = CF =\frac{65}{4}$. what is the sum of the digits of the area of hexagon $ABCDEF$?}
%%circumcircle
\begin{sol}
$\boxed{15}$
\end{sol}

\prob{4}{Mandelbrot Nationals 2009}{Triangle ABC has sides of length $AB =\sqrt{41}, AC = 5$,and $BC = 8$. Let O be the center of the circumcircle of $\triangle ABC$, and let $A'$ be the point diametrically opposite $A$, as shown. Determine the area of $\triangle A'BC$.
}

\prob{4}{DMC Mock AMC 10}{In trapezoid $ABCD$ with $\overline{AD}|| \overline{BC}$ and side lengths $AD = 18$, $BC = 20$, and
$AB = CD = 8$, let $X$ be the intersection of line $AB$ and the bisector of $\angle ADC$, and
let $Y$ be the intersection of line $CD$ and the bisector of $\angle DAB$. What is $XY$ ?}

\begin{sol}
$\boxed{24}$
\end{sol}
\prob{4}{HMMT February 2014}{Triangle ABC has sides $AB = 14,BC = 13,$ and $CA = 15$. It is inscribed in circle, which has center $O$. Let $M$ be the midpoint of $AB$, let $B'$ be the point on  diametrically opposite $B$, and let $X$ be the intersection of $AO$ and $MB'$. Find the length of $AX$.}

\begin{sol}
AX is the centroid of $ABB'$.
\end{sol}

\prob{4}{AIME 1989}{Triangle $ABC$ has an right angle at $B$ and contains a point $P$ such that $AP=10$, $BP=6$, and $\angle APC = \angle CPB =\angle BPA$. Find $CP$.
}
\begin{sol}
Law of Cosines and Pythagorean Theorem gives $CP=33$.
\end{sol}

\prob{4}{106 Geometry Problems}{In triangle $ABC$, medians $BB_{1}$ and $CC_{1}$ are perpendicular. Given that $AC=19$ and $AB=22$, find $BC$.}

\begin{sol}
Let $BG=2x, GB_{1}=x$ and $CG=2y, GC_{1}=y$. Set systems of equations and solve.
\end{sol}

\prob{4}{AIME 2005}{In quadrilateral $ABCD$, let $BC=8, CD=12, AD=10$ and $\angle A = \angle B = 60^{\circ}$. Given that $AB = p + \sqrt{q},$ where $p$ and $q$ are positive integers, find $p+q.$}

\begin{sol}
Extend $AD$ and $BC$ to make an equilateral triangle and then Law of Cosines.
\end{sol}

\prob{4}{HMMT November 2013}{ Let ABC be a triangle and D a point on BC such that $AB=\sqrt{2}, BC=\sqrt{3}, \angle BAD = 30^{\circ}$, and$\angle CAD = 45^{\circ}$. Find AD.}

\prob{4}{CMIMC 2020}{
Let $ABC$ be a triangle with centroid $G$ and $BC = 3$. If $ABC$ is similar to $GAB$, compute the area of $ABC$.
}

\prob{4}{Magic Math AMC 10 2020}{Isosceles triangle $\triangle ABC$ has $AB = AC$ and circumcenter O. The circle with diameter $AO$ intersects segment $BC$ at $P$ and $Q$, with $P$ closer to $B$, such that $BP = QC = 4$ and $PQ = 6$.
Compute the area of $\triangle A$}

\prob{5}{PHS HMMT TST 2020}{$\triangle$ ABC has side lengths $AB = 11, BC = 13, CA = 20$. A circle is drawn with diameter $AC$. Line $AB$
intersects the circle at $D \neq A$, and line BC intersects the circle at $E \neq B$. Find the length of $DE$.}

\prob{5}{NEMO 2017}{In convex equilateral hexagon $ABCDEF$, $AC = 13$, $CE = 14$, and $EA = 15$. It is given that the area of $ABCDEF$ is twice the area of triangle $ACE$. Compute $AB$.}

\begin{sol}
Reflect $B,D,F$ over $AC,CE,EA$ respectively. Then $[AB\rq{}C] + [CD\rq{}E] + [E\rq{}FA] = [ABC]$. Note that LHS monotonically increases as $ AB$ increases. Then, note that this is satisified when $AB=R$ which implies the $B\rq{}, D\rq{}, F\rq{}$ are circumcenter. 
$\boxed{\frac{65}{8}}$
\end{sol}

\prob{5}{CMIMC 2020}{
Let $ABC$ be a triangle with centroid $G$ and $BC = 3$. If $ABC$ is similar to $GAB$, compute the area of $ABC$.
}

\prob{5}{ARML Local 2020}{Triangle $ABC$ has side lengths $AB=8, BC =5,$ and $AC=7$. Point $P$ lies inside $\triangle ABC$ so that $\angle APB = \angle BPC = \angle CPA = 120^{\circ}$. Compute $AP+BP+CP$.}

\begin{sol}
$\sqrt{129}$
\end{sol}

\prob{5}{Scrabber94 Mock AMC 10 2020}{Parallelogram $ABCD$ has $AB = CD = 10$, $BC = AD = 6$, and $BD = 8$.
Let $O_1, O_2, O_3$, and $O_4$ be the circumcenters of $\triangle ABC, \triangle ABD, \triangle ACD$,
and $\triangle BCD$, respectively. What is the area of quadrilateral $O_{1}O_{2}O_{3}O_{4}$?}

\begin{sol}
$\boxed{27}$
\end{sol}
\prob{5}{Magic Math AMC 10 2020}{Six points $A, B, C, D, E, F$ are selected on a circle with center $O$ such that $ABCDEF$ is an equiangular hexagon with $AB = CD = EF < BC = DE = F A$. Diagonals $AD, BE, F C$ are
drawn, intersecting at points $X, Y, Z$. The three circles passing through $O$ and two of $X, Y, Z$
are each internally tangent to $O$. Find $BC/AB$.}

\prob{6}{AMC 12B 2017/24}{Quadrilateral $ABCD$ has right angles at $B$ and $C$, $\triangle ABC \sim \triangle BCD$, and $AB > BC$. There is a point $E$ in the interior of $ABCD$ such that $\triangle ABC \sim \triangle CEB$ and the area of $\triangle AED$ is $17$ times the area of $\triangle CEB$. What is $\frac{AB}{BC}$?}

\begin{sol}
Coordinate bashable by setting $BC=1$ and letting $B=(0,0), A=(0,x), C=(1,\frac{1}{x})$. Then, $E=(\frac{1}{x^2+1}, \frac{x}{x^2+1})$. We apply Shoelace and find $x^2=9+2\sqrt{10}\implies x=\boxed{2+\sqrt{5}}$.
\end{sol}

\prob{6}{SLKK AIME 2020}{Cyclic quadrilateral $AXBY$ is inscribed in circle $\omega$ such that $AB$ is a
diameter of $\omega$. $M$ is the midpoint of $XY$ and $AM = 13, BM = 5$, and $AB = 16$. If the
area of AXBY can be expressed as $m\sqrt{p}+n$, where $m, n$, and $p$ are positive integers such
that $m$ and $n$ are relatively prime and $p$ is not divisible by the square of a prime, find
the remainder when $m + n + p$ is divided by $1000$.}

\begin{sol}
$\boxed{058}$
\end{sol}
\setcounter{problem}{0}

\section{Misc}
These are problems that don't really fall into any other category at all.

\subsection{General}
\prob{1}{Math Prizes For Girls 2019}{Let $a_{1}, a_{2}, \ldots , a_{2019}$ be a sequence of real numbers. For every five
indices $i, j, k$, and $m$ from $1$ through $2019$, at least two of the
numbers $a_{i},a_{j},a_{k}$ and $a_{m}$ have the same absolute value. What
is the greatest possible number of distinct real numbers in the given
sequence?}

\begin{sol}
There can't be five or more distinct absolute values because then we can just take those five. It's clear that only having four distinct absolute values work. By the Pigeonhole Principle, at least two are the same. Then, from those four, we can have at most $4\cdot 2=\boxed{8}$ distinct real numbers.
\end{sol}

\prob{1}{AlcumusGuy Mock AMC 10 2016-2017}{How many of the following four statements are true?}

\begin{enumerate}
	\item It is possible to find an even number of even integers such that the sum of their reciprocals is 1.

	\item It is possible to find an even number of odd integers such that the sum of their reciprocals is 1.

	\item It is possible to find an odd number of even integers such that the sum of their reciprocals is 1.
	
	\item It is possible to find an odd number of odd integers such that the sum of their reciprocals is 1.
\end{enumerate}

\begin{sol}
$\boxed{3}$
\end{sol}

\prob{2}{ARML 2014}{The sequence of words $\{a_{n}\}$ is defined as follows: $a_{1} = X, a_{2} = O$, and for $n \ge 3$, an is $a_{n-1}$ followed by the reverse of $a_{n-2}$. For example, $a_{3} = OX, a_{4} = OXO, a_{5} = OXOXO$, and
$a_{6} = OXOXOOXO$. Compute the number of palindromes in the first $1000$ terms of this
sequence.}

\begin{sol}
Only $a_{3n}$ are not palindromes. We can prove this by induction. Let $\overline{a}$ be the reverse of $a$. Let $a_{3n+1}=x,a_{3n+2}=y$ and assume those are palindromes. Then, $a_{3n+3}=y\overline{x}$ which cannot be a palindrome (the length is less than twice of $y$) and $a_{3n+4}=y\overline{x}y$ which is a palindrome as $y$ is a palindrome. Also, $a_{3n+5}=y\overline{x}yx\overline{y}$. This also a palindrome.
Then, our answer is $1000-333=\boxed{667}$.
\end{sol}

\prob{3}{MMATHS 2019}{S is a set of positive integers with the following properties:
\begin{enumerate}
\item There are exactly 3 positive integers missing from S.

\item If a and b are elements of S, then a + b is an element of S. (We allow a and b to be the same.)
\end{enumerate}
How many possibilities are there for the set S?}

\begin{sol}
1 must be missing. Then, do casework on second missing number.
$\boxed{4}$
\end{sol}
\subsection{Games}

\prob{1}{HMMT February 2013}{Rahul has ten cards face-down, which consist of five distinct pairs of matching cards. During each
move of his game, Rahul chooses one card to turn face-up, looks at it, and then chooses another to
turn face-up and looks at it. If the two face-up cards match, the game ends. If not, Rahul flips both
cards face-down and keeps repeating this process. Initially, Rahul doesn’t know which cards are which.
Assuming that he has perfect memory, find the smallest number of moves after which he can guarantee
that the game has ended.}

\begin{sol}
$\boxed{4}$
\end{sol}

\prob{1}{HMMT February 2013}{On a game show, Merble will be presented with a series of 2013 marbles, each of which is either red or
blue on the outside. Each time he sees a marble, he can either keep it or pass, but cannot return to a
previous marble; he receives 3 points for keeping a red marble, loses 2 points for keeping a blue marble,
and gains 0 points for passing. All distributions of colors are equally likely and Merble can only see
the color of his current marble. If his goal is to end with exactly one point and he plays optimally,
what is the probability that he fails?}

\begin{sol}
Only lose if all red or blue
$\boxed{\frac{1}{2^{2012}}}$
\end{sol}

\prob{2}{2018 Memorial Day Mock AMC 10}{Ashley and Ben are playing a game. At the beginning of the game, $n = 1$. Starting with Ashley, the players then take turns multiplying $n$ by an integer between $2$ and $9$,
inclusive. The first player to get a product of at least $100$ wins. If neither player makes
any mistakes, how many different integers can Ashley multiply n by on her first turn
in order to win?}

\begin{sol}
$\boxed{4}$
\end{sol}

\prob{2}{BmMT 2016}{
Suppose you have a $20 \times 16$ bar of chocolate squares. You want to break the bar into smaller
chunks, so that after some sequence of breaks, no piece has an area of more than 5. What is the
minimum possible number of times that you must break the bar?}

\begin{sol} 
By the Pigeon Hole Principle, at the end, there should be at least $\frac{320}{5}=64$ pieces. It takes at least $63$ moves to break it into $64$ pieces as each move creates one more piece. We can also construct a sequence of moves by noticing that this is equalty case of the Pigeon Hole Principle so each piece must have exactly $5$ squares. Break it into $16$ of $20\times 1$ pieces. This uses $15$ moves. Then, break each of those pieces into $4$ of $5\times 1$ pieces. This uses $3\cdot 16=48$ moves. So, our answer is $\boxed{63}$
\end{sol}

\prob{3}{BMT 2015}{Two players play a game with a pile with N coins is on a table. On a player’s turn, if there
are n coins, the player can take at most $\frac{n}{2} + 1$ coins, and must take at least one coin. The
player who grabs the last coin wins. For how many values of N between 1 and 100 (inclusive)
does the first player have a winning strategy?}

\subsection{Logic}
\prob{2}{BmMT 2014}{Alice, Bob, Carl, and Dave are either lying or telling the truth. If the four of them make the
following statements, who has the coin?
\begin{center}
Alice: I have the coin.

Bob: Carl has the coin.

Carl: Exactly one of us is telling the truth.

Dave: The person who has the coin is male.
\end{center}}

\begin{sol}
Analyzing, Carl must be lying since if he was telling the truth, everyone else are lying which means Alice can't have the coin but the person who has the coin isn't male which is a contradiction. Also note that all of them can't be lying by similar reasoning. So, either two, three, or four are telling the truth. Four telling the truth is impossible as Carl's statement is false. Three telling the truth means everyone but Carl is telling the truth which is impossible as Alice and Dave's statements conflict. So, two must be telling the truth. Dave's and Alice's statements are true if and only if the other is false. If Dave is false and Alice is true, then Bob must also be false which is a contradiction to the fact two are telling the truth. If Dave is true and Alice is false, then Bob must be telling the truth and \boxed{Carl} has the coin. This is the only possible case.
\end{sol}

\prob{2}{Berkeley Math Circle 2013}{
Ten people sit side by side at a long table, all facing the same direction. Each of them is either a knight (and always
tells the truth) or a knave (and always lies). Each of the people announces: “There are more knaves on my left than
knights on my right.” How many knaves are in the line?
}

\begin{sol}
$5$ knaves on the left, $5$ knights on the right gives a valid solution. So, $\boxed{5}$.
\end{sol}

\prob{2}{Scrabbler94 Mock AMC 10 2020}{If today is rainy, then Erin wears a raincoat. If today is rainy and Erin
is wearing a raincoat, then Erin will not get wet outside. Which of the
following statement(s) is logically implied from the above two statements?

I: If today is not rainy, then Erin is not wet outside.

II: If Erin is not wet outside, then today is not rainy.

III: If Erin is wet outside, then today is not rainy.

IV: If Erin is wet outside, then Erin is not wearing a raincoat.
}

\begin{sol}
$\boxed{\text{only }III}$
\end{sol}

\end{document}